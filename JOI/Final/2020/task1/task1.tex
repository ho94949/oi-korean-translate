\begin{problem}{길 뿐인 목걸이}
	{standard input}{standard output}
	{1초}{256MB}{}
	
	당신은 Just Odd Inventions라는 회사를 아는가? 이 회사는 ``그저 기묘한 발명"을 하는 회사다. 여기서는 줄여서 JOI 사라고 한다.
	
	JOI 사는 신상품 ``길 뿐인 목걸이"를 개발했다. 목걸이는 $N+1$ 종류가 있어서, 각 종류에는 1번부터 $N+1$번까지의 번호가 붙어있다. $i$ 번째 ($1 \le i \le N+1$) 종류 넥타이의 길이는 $A_i$이다.
	
	JOI 사는 사원을 모아서 넥타이의 테스트를 하기로 했다. 테스트에는 $N$ 명의 사원이 참가했고, $j$ 번째 ($1 \le j \le N$) 사원이 처음에 매고 있는 넥타이의 길이는 $B_j$이다.
	
	테스트는 다음과 같은 순서로 진행될 예정이다.
	
	\begin{enumerate}
		\item 우선 테스트에 사용하지 않을 넥타이를 한 종류 골라 제외한다.
		\item 각 사원은 제외된 넥타이를 뺀 넥타이 중 한 종류를 고른다. 단, 어떤 두 명도 같은 종류의 넥타이를 고를 수는 없다.
		\item 마지막으로 각 사원은 매고 있던 넥타이를 벗어서, 처음에 선택한 넥타이를 맨다.
	\end{enumerate}

	길이 $b$의 넥타이를 매고 있던 사원이 길이 $a$의 넥타이를 맬 때 $\max\{a-b, \ 0\}$의 해당하는 \textbf{기묘함}을 느낀다(여기서, $\max\{a-b, \ 0\}$은 $a-b$와 0중 작지 않은 값을 의미한다.) 테스트에서 각 사원이 느낀 기묘함의 최댓값이 그 테스트의 기묘함이다.
	
	테스트에서 제외한 넥타이가 $k$ 번째 넥타이일 경우 테스트의 기묘함이 될 수 있는 값 중 최솟값을 $C_k$라 하자.
	
	각 종류의 넥타이의 길이와 각 사원이 처음에 매고 있는 넥타이의 길이가 주어졌을 때, $C_1$, $C_2$, $\cdots$, $C_{N+1}$의 값을 구하는 프로그램을 작성하여라.
	
	\InputFile
	
	표준 입력에서 다음과 같은 형식으로 주어진다. 모든 값은 정수이다.

	$N$
	
	$A_1$ $\cdots$ $A_{N+1}$
	
	$B_1$ $\cdots$ $B_{N}$
	
	\OutputFile
	
	$C_1$, $C_2$, $\cdots$, $C_{N+1}$의 값을 공백으로 구분하여 표준 출력으로 첫째 줄에 출력하여라.
	\Constraints
	
	\begin{itemize}
	\item $1 \le N \le 200\ 000$.
	\item $1 \le A_i \le 1\ 000\ 000\ 000$ ($1 \le i \le N+1$).
	\item $1 \le B_j \le 1\ 000\ 000\ 000$ ($1 \le j \le N$).
	\end{itemize}
	
	
	\SubtaskWithCost{1}{1}
	\begin{itemize}
		\item $N \le 10$.
	\end{itemize}

	\SubtaskWithCost{2}{8}
	\begin{itemize}
		\item $N \le 2\ 000$.
	\end{itemize}
	
	\SubtaskWithCost{3}{91}
	
	추가 제한조건이 없다.
	
	\Examples
		
	\begin{example}
	\exmp{
	3
	4 3 7 6
	2 6 4
		}{%
	2 2 1 1
		}%
	\end{example}

	예를 들어, 테스트가 다음과 같이 진행된다고 하자.
	
	\begin{itemize}
		\item 4번째 종류의 넥타이를 제외한다.
		\item 1번째 사원이 1번째 넥타이를, 2번째 사원이 2번째 넥타이를, 3번째 사원이 3번째 넥타이를 고른다.
		\item 각 사원이 넥타이를 맨다.
	\end{itemize}

	이때, 각 사원이 느낀 기묘함은 차례로 2, 0, 3이 된다. 이 테스트의 기묘함은 3이다.
	
	사원이 고른 넥타이를 바꾸는 것으로 테스트의 기묘함을 1로 만들 수 있다. 예를 들어, 테스트가 다음과 같이 진행된다고 하자.
	
	\begin{itemize}
		\item 4번째 종류의 넥타이를 제외한다.
		\item 1번째 사원이 2번째 넥타이를, 2번째 사원이 3번째 넥타이를, 3번째 사원이 1번째 넥타이를 고른다.
		\item 각 사원이 넥타이를 맨다.
	\end{itemize}

	이때, 각 사원이 느낀 기묘함은 차례로 1, 1, 0이 된다. 이 테스트의 기묘함은 1이다.
	
	이것이 4번째 종류의 넥타이를 제외했을 때 테스트의 기묘함의 최솟값이므로, $C_4 = 1$이다.
	

	\begin{example}
	\exmp{
		5
		4 7 9 10 11 12
		3 5 7 9 11
	}{%
		4 4 3 2 2 2
	}%
	\end{example}

\end{problem}

