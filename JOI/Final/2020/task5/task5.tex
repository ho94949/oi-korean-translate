\begin{problem}{화재}
	{standard input}{standard output}
	{1.5초}{256MB}{}
	
	JOI 마을에는 $N$ 개의 구역이 있고, 1번부터 $N$번까지의 번호가 붙어있다. 이 구역은 번호순으로 일렬로 나열되어 있다. 지금 각 구역에는 화재가 발생했고 시각 0에 $i$번 ($1 \le i \le N$) 구역에 일어난 화재의 세기는 $S_i$이다. ($S_i > 0$)
	
	시각 0에 1번 구역에서 $N$번 구역까지의 방향으로 바람이 불기 시작했다. 시각 $t$에 ($ 0 \le t$) 인접한 두 구역에 대해 위쪽 구역의 화재가 아래쪽 구역의 화재보다 더 강할 경우, 시각 $t+1$에 아래쪽 구역의 화재의 세기는 시각 $t$의 위쪽 구역의 화재의 세기와 같아진다. 그렇지 않을 경우, 시각 $t+1$에 아래쪽 구역의 화재의 세기는 시각 $t$와 같다. 즉, 시각 $t$ ($0 \le t$)의 $i$번 구역의 화재의 세기를 $S_i(t)$라고 하자. $1 \le t$일 경우, $S_i(t) = \max\{S_{i-1}(t-1), S_i(t-1)\}$이 된다. 단, 임의의 $t$ ($0 \le t$)에 대해 $S_0(t) = 0$이라고 하고 임의의 $i$ ($1 \le i \le N$) 에 대해, $S_i(0) = S_i$라고 하자.
	
	소방관인 당신은 $Q$ 개의 계획을 정했다. $Q$ 개의 계획 중 하나를 실행할 예정이다. $j$ 번째 ($1 \le j \le Q$) 계획은 시각 $T_j$에 $L_j \le k \le R_j$를 만족하는 모든 $k$번 구역의 불을 끄는 것이다. 화재의 세기가 $s$인 구역을 소화하는 데에는 $s\ell$의 물이 필요하다. 즉, $j$ 번째 계획에 사용되는 물의 양은 $S_{L_j}(T_j) + S_{L_j+1}(T_j) + \cdots + S_{R_j} (T_j) \ell$이다.
	
	어떤 계획을 실행할지 정하기 위하여 각 계획에 필요한 물의 양을 알고 싶다.
	
	시각 0의 화재의 세기와 계획의 정보가 주어졌을 때, 각 계획에 필요한 물의 양을 구하는 프로그램을 작성하여라.
	
	
	\InputFile
	
	표준 입력에서 다음과 같은 형식으로 주어진다. 모든 값은 정수이다.

	$N$ $Q$
	
	$S_1$ $\cdots$ $S_N$
	
	$T_1$ $L_1$ $R_1$
	
	$\vdots$
	
	$T_Q$ $L_Q$ $R_Q$
	
	\OutputFile
	
	표준 출력으로 $Q$개의 줄을 출력하여라. $j$ 번째 ($1 \le j \le Q$) 줄에는 $j$ 번째 계획에 필요한 물의 양을 출력하여라.
	
	\Constraints
	
	\begin{itemize}
	\item $1 \le N \le 200\ 000$.
	\item $1 \le Q \le 200\ 000$.
	\item $1 \le S_i \le 1\ 000\ 000\ 000$ ($1 \le i \le N$).
	\item $1 \le T_j \le N$ ($1 \le j \le Q$).
	\item $1 \le L_j \le R_j \le N$ ($1 \le j \le Q$).
	
	\end{itemize}
	
	
	\SubtaskWithCost{1}{1}
	\begin{itemize}
		\item $N \le 200$.
		\item $Q \le 200$.
	\end{itemize}

	\SubtaskWithCost{2}{6}
	\begin{itemize}
		\item $T_1 = T_2 = \cdots = T_Q$.
	\end{itemize}

	\SubtaskWithCost{3}{7}
	\begin{itemize}
		\item $L_j = R_j$ ($1 \le j \le Q$).
			\end{itemize}


	\SubtaskWithCost{4}{6}
	\begin{itemize}
		\item $S_i \le 2$ ($1 \le i \le N$).
	\end{itemize}
	
	\SubtaskWithCost{5}{80}
	
	추가 제한조건이 없다.
	
	\Examples
		
	\begin{example}
	\exmp{
	5 5 
	9 3 2 6 5
	1 1 3
	2 1 5
	3 2 5
	4 3 3
	5 3 5
		}{%
	21
	39
	33
	9
	27
		}%
	\end{example}

	\begin{itemize}
		\item 시각 0의 각 구역의 화재의 세기는 1번 구역부터 차례로 9, 3, 2, 6, 5이다.
		\item 시각 1의 각 구역의 화재의 세기는 1번 구역부터 차례로 9, 9, 3, 6, 6이다. 또한, 첫 번째의 계획에 필요한 물의 양은 9+9+3 = 21$\ell$이다.
		\item 시각 2의 각 구역의 화재의 세기는 1번 구역부터 차례로 9, 9, 9, 6, 6이다. 또한, 첫 번째의 계획에 필요한 물의 양은 9+9+9+6+6 = 39$\ell$이다.
		\item 시각 3의 각 구역의 화재의 세기는 1번 구역부터 차례로 9, 9, 9, 9, 6이다. 또한, 첫 번째의 계획에 필요한 물의 양은 9+9+9+6 = 33$\ell$이다.
		\item 시각 4의 각 구역의 화재의 세기는 1번 구역부터 차례로 9, 9, 9, 9, 9이다. 또한, 첫 번째의 계획에 필요한 물의 양은 9$\ell$이다.
		\item 시각 5의 각 구역의 화재의 세기는 1번 구역부터 차례로 9, 9, 9, 9, 9이다. 또한, 첫 번째의 계획에 필요한 물의 양은 9+9+9=27$\ell$이다.
	\end{itemize}

	이 입력 예제는 서브태스크 1, 5의 조건을 만족한다.

	\begin{example}
	\exmp{
		10 10 
		3 1 4 1 5 9 2 6 5 3
		1 1 6
		2 8 10
		4 2 7
		8 3 3
		6 1 10
		3 2 8
		5 1 9
		7 4 5
		9 7 9
		10 10 10
	}{%
		28
		21
		34
		4
		64
		43
		55
		9
		27
		9
	}%
	\end{example}

	이 입력 예제는 서브태스크 1, 5의 조건을 만족한다.


\begin{example}
	\exmp{
		10 10 
		3 1 4 1 5 9 2 6 5 3
		1 6 6
		2 8 8
		4 2 2
		8 3 3
		6 1 1
		3 4 4
		5 5 5
		7 10 10
		9 8 8
		10 7 7
	}{%
		9
		9
		3
		4
		3
		4
		5
		9
		9
		9
	}%
\end{example}

	이 입력 예제는 서브태스크 1, 3, 5의 조건을 만족한다.


\begin{example}
	\exmp{
		10 10
		3 1 4 1 5 9 2 6 5 3
		7 1 6
		7 8 10
		7 2 7
		7 3 3
		7 1 10
		7 2 8
		7 1 9
		7 4 5
		7 7 9
		7 10 10
	}{%
		28
		27
		34
		4
		64
		43
		55
		9
		27
		9
	}%
\end{example}

이 입력 예제는 서브태스크 1, 2, 5의 조건을 만족한다.

\begin{example}
	\exmp{
		20 20
		2 1 2 2 1 1 1 1 2 2 2 1 2 1 1 2 1 2 1 1
		1 1 14
		2 3 18
		4 10 15
		8 2 17
		9 20 20
		4 8 19
		7 2 20
		11 1 5
		13 2 8
		20 1 20
		2 12 15
		7 1 14
		12 7 18
		14 2 17
		9 19 20
		12 12 12
		6 2 15
		11 2 15
		19 12 17
		4 1 20
	}{%
		25
		30
		12
		32
		2
		24
		38
		10
		14
		40
		8
		28
		24
		32
		4
		2
		28
		28
		12
		40
	}%
\end{example}

이 입력 예제는 서브태스크 1, 4, 5의 조건을 만족한다.


\end{problem}

