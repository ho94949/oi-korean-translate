\begin{problem}{스탬프 수집 3}
	{standard input}{standard output}
	{2초}{1024MB}{}
	
	JOI 군이 사는 IOI 나라는 거대한 호수가 있는 것으로 유명하다. 오늘 호수 둘레에서 스탬프 수집 대회가 개최된다.
	
	호수 둘레에는 $N$ 개의 도장이 있어서, 시계방향으로 1번부터 $N$번까지의 번호가 붙어있다. 호수 둘레의 길이는 $L$m이고, $i$번 ($1 \le i \le N$) 스탬프는 출발 지점으로부터 시계방향으로 $X_i$m 앞에 있다.
	
	스탬프 수집의 각 참가자는 스탬프 수집 대회를 시작할 때 출발 지점에 모이고, 시작한 후에는 호수 둘레를 시계방향 혹은 반시계방향으로 돌 수 있다. 참가자는 도장이 있는 위치에 도착하면 그 도장을 찍어서 스탬프를 남길 수 있다. 같은 도장을 여러 번 찍을 수는 없다. 하지만 스탬프 수집 대회가 시작한 지 $T_i$초가 지나기 전에 도착해야 한다(대회가 시작한 지 정확히 $T_i$초 후에 도장이 있는 위치에 도착해도 도장을 찍을 수 있다.)
	
	JOI 군은 이 스탬프 수집 대회의 참가자이다. JOI 군은 1m를 걷는 데 1초가 걸린다. 또한 JOI는 도장 찍기의 숙련자이기 때문에, 도장을 찍는 데 걸리는 시간은 무시해도 좋다. 
	
	도장의 개수, 호수 둘레의 길이, 각 도장이 있는 지점과 각 도장을 찍을 수 있는 시간이 주어졌을 때, JOI군이 찍을 수 있는 스탬프 개수의 최댓값을 구하는 프로그램을 작성하여라.
	
	
	\InputFile
	
	표준 입력에서 다음과 같은 형식으로 주어진다. 모든 값은 정수이다.

	$N$ $L$
	
	$X_1$ $\cdots$ $X_N$
	
	$T_1$ $\cdots$ $T_N$
	
	\OutputFile
	
	JOI군이 찍을 수 있는 스탬프 개수의 최댓값을 표준 출력으로 첫째 줄에 출력하여라.
	
	
	\Constraints
	
	\begin{itemize}
	\item $1 \le N \le 200$.
	\item $2 \le L \le 1\ 000\ 000\ 000$.
	\item $1 \le X_i < L$ ($1 \le i \le N$).
	\item $X_i < X_{i+1}$ ($1 \le i \le N-1$).
	\item $0 \le T_i \le 1\ 000\ 000\ 000$ ($1 \le i \le N$).
	\end{itemize}
	
	
	\SubtaskWithCost{1}{5}
	\begin{itemize}
		\item $N \le 12$.
		\item $L \le 200$.
		\item $T_i \le 200$ ($1 \le i \le N$).
	\end{itemize}

	\SubtaskWithCost{2}{10}
	\begin{itemize}
		\item $N \le 15$.
	\end{itemize}


	\SubtaskWithCost{3}{10}
	\begin{itemize}
		\item $L \le 200$.
		\item $T_i \le 200$ ($1 \le i \le N$).
	\end{itemize}
	
	\SubtaskWithCost{4}{75}
	
	추가 제한조건이 없다.
	
	\Examples
		
	\begin{example}
	\exmp{
	6 25
	3 4 7 17 21 23
	11 7 17 10 8 10
		}{%
	4
		}%
	\end{example}

	다음 방법으로 JOI 군은 4개의 스탬프를 모을 수 있다.

	\begin{enumerate}
		\item 반시계방향으로 2m 간다. 대회 시작 2초 후이기 때문에, 6번 도장을 찍을 수 있다.
		\item 또 반시계방향으로 2m 간다. 대회 시작 4초 후이기 때문에, 5번 도장을 찍을 수 있다.
		\item 시계방향으로 7m 간다. 대회 시작 11초 후이기 때문에, 1번 도장을 찍을 수 있다.
		\item 또 시계방향으로 1m 간다. 대회 시작 12초 후이기 때문에, 2번 도장을 찍을 수 없다.
		\item 또 시계방향으로 3m 간다. 대회 시작 15초 후이기 때문에, 3번 도장을 찍을 수 없다.
	\end{enumerate}

	어떤 방법으로도 JOI 군이 5개 이상의 스탬프를 모을 수는 없기 때문에, 4를 출력한다.
	

	\begin{example}
	\exmp{
		5 20
		4 5 8 13 17
		18 23 15 7 10
	}{%
		5
	}%
	\end{example}

	JOI군은 스탬프 수집이 시작한 후, 호수 둘레를 반시계방향으로만 돌면서 모든 스탬프를 모을 수 있다.


	\begin{example}
		\exmp{
			4 19
			3 7 12 14
			2 0 5 4
		}{%
			0
		}%
	\end{example}

	유감이지만, JOI군은 어떻게 이동해도 스탬프를 하나도 모을 수 없다.


\begin{example}
	\exmp{
		10 87
		9 23 33 38 42 44 45 62 67 78
		15 91 7 27 31 53 12 91 89 46
	}{%
		5
	}%
\end{example}

	

\end{problem}

