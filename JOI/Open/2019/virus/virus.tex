\begin{problem}{3단 점프}
	{standard input}{standard output}
	{2 seconds}{256 megabytes}{}
	
	당신은 Just Odd Invention 유한회사를 아는가? 이 회사의 일은 ``그저 기묘한 발명"을 하는 것이다. 우리는 JOI회사라고 부른다.
	
	JOI회사는 새로운 바이러스 ``JOI 바이러스"를 발명했다. JOI회사는 IOI섬의 주민을 JOI 바이러스로 감염시키는 실험을 하고 싶다.
	
	IOI섬은 직사각형 모양이다. $R-1$개의 동쪽에서 서쪽으로 뻗은 평행한 도로들과, $C-1$개의 북쪽에서 남쪽으로 뻗은 평행한 도로가 있다. 이들은 섬을 $RC$개의 구간으로 나눈다. 각 구간에는 1명의 주밍시 살 고 있다. 우리는 북쪽에서 $i$번째, 서쪽에서 $j$번째 ($1 \le i \le R, \ 1 \le j \le C$)주민을 ``\textbf{주민} $(i, j)$"라고 부를 것이다.
	
	IOI 섬에는, 하루를 $M$개의 시한(시간 구간)들로 나눈다. 이 중 $k$번째 시한을 ``\textbf{시한} $k$" 라고 한다. 바람은 언제나 동, 서, 남, 북 중 하나의 방향으로만 분다. 방향의 부는 방향은 시한이 바뀌면 바뀔 수 있으나, 같은 시한 안에서는 한 방향으로만 분다.
	
	각 주민은 ``\textbf{저항력}"을 가지고 있다. 주민 $(i, j)$ ($1 \le i \le R, \ 1 \le j \le C$)의 저항력은 음이 아닌 정수 $U_{i, j}$로 표현된다.
	
	
	\begin{itemize}
		\item $U_{i, j}$가 0이면, 주민 $(i, j)$는 높은 저항력을 가지고 있어서 JOI 바이러스에 감염되지 않는 다는 것을 말한다.
		\item $U_{i, j}$가 0이면, 주민 $(i, j)$는 바이러스에 감염될 수도 있다. 다음 조건이 $U_{i, j}$ 시한 동안 계속 된다면, 그 시민은 다음 시한에 바이러스에 감염된다.
		\begin{itemize}
			\item 바람이 불어오는 쪽에 인접한 주민이 JOI 바이러스에 감염되어 있다.
		\end{itemize}
	\end{itemize}

	어떤 날의 마지막 시한과, 다음 날의 첫째 시한이 인접해 있음을 유의하여라.
	
	실험 목적으로, 적어도 한 명의 시민을 감염 시키고 싶지만, 많은 시민을 감염 시키고 싶지는 않다. 처음에, 한 명의 시민을 골라 그 사람을 감염 시킨다. 저항력이 0인 시민은 고를 수 없다.
	
	각 시간 마다 바람이 불어오는 방향과 각 시민의 저항력이 주어졌을 때, 어떤 사람을 감염 시켜야 $10^100$일이 지난 후에 감염되는 사람의 총 수가 최소가 되는지, 그리고 그 때 감염된 사람 수를 구하여라.
	
	
	\InputFile
	
	표준 입력에서 다음과 같은 형식으로 주어진다. 
	
	$M$ $R$ $C$
	
	$D$
	
	$U_{1, 1}$ $\cdots$ $U_{1, C}$

	\ $\vdots$ \ \ \ \ \ \ \ \ \ \ $\vdots$	
	
	$U_{R, 1}$ $\cdots$ $U_{R, C}$
	
	$D$는 IOI 섬에서 불어오는 바람의 방향을 의미하는 문자열이다. $D$는 4가지의 문자 \texttt{N}, \texttt{S}, \texttt{W} 혹은 \texttt{E} 로 구성되어 있다. $k$번째 ($1 \le k \le M$) 문자는 시한 $k$에서 불어오는 바람의 방향을 나타낸다. 바람이 부는 방향을 나타내는게 아니라는 것에 유의하여라. \texttt{N}은 북쪽, \texttt{S}는 남쪽, \texttt{W}은 서쪽, \texttt{E}는 동쪽을 의미한다.
	
	\OutputFile
	
	표준 출력으로 2개의 줄을 출력하여라.
	첫째 줄은, $10^{100}$일이 지난 이후에 감염되는 사람의 최소 수 이다. 둘째 줄은 최소 수의 사람을 감염 시키기 위해서 고를 수 있는 사람들의 수를 출력하여라.
	
	\Constraints
	
	\begin{itemize}
		
		\item $1 \le M \le 100\ 000$.
		\item $1 \le R \le 800$.
		\item $1 \le C \le 800$.
		\item $D$는 \texttt{N}, \texttt{S}, \texttt{W}, \texttt{E}로만 구성된 길이 $M$의 문자열이다.
		\item $0 \le U_{i, j} \le 100\ 000$ ($1 \le i \le R, \ 1 \le j \le C$)
		\item $1 \le U_{i, j}$ 인 $(i, j)$가 적어도 하나 존재한다. ($1 \le i \le R, \ 1 \le j \le C$)
	\end{itemize}
	
	
	\SubtaskWithCost{1}{14}
	\begin{itemize}
		\item $D$는 \texttt{W} 혹은 \texttt{E}로만 구성되어 있다.
	\end{itemize}
	
	\SubtaskWithCost{2}{6}
	\begin{itemize}
		\item $1 \le R \le 50$.
		\item $1 \le C \le 50$.
	\end{itemize}
	
	
	\SubtaskWithCost{3}{80}
	
	추가 제한조건이 없다.
	
	\Examples
	
	\begin{example}
		\exmp{
			6 3 4
			SWNEES
			2 1 1 2
			1 0 1 3
			1 1 2 2
		}{%
			8
			8
		}%
	\exmp{
		4 4 4
		EWWE
		1 2 1 2
		1 1 1 1 
		0 0 0 0
		2 2 2 4
	}{%
		3
		3
	}%
	\end{example}
	
	\Note
	
	첫째 예제에서, 시민 (3, 1)을 첫번째 사람으로 감염시켰다고 하자.
	
	\begin{itemize}
		\item 시민 (2, 1)에 대해서, 첫번째 날의 시한 1에 남쪽에서 바람이 불어오고, 남쪽에 있는 사람이 이미 감염되어 있으므로, 첫번째 날의 시한 2에 이 시민은 감염된다.
		\item 시민 (3, 2)에 대해서, 첫번째 날의 시한 2에 서쪽에서 바람이 불어오고, 서쪽에 있는 사람이 이미 감염되어 있으므로, 첫번째 날의 시한 3에 이 시민은 감염된다.
		\item 시민 (1, 1)에 대해서, 첫번째 날의 시한 6에 남쪽에서 바람이 불어오고, 남쪽에 있는 사람이 이미 감염되어 있고, 두번째 날의 시한 1에 남쪽에서 바람이 불어오고, 남쪽에 있는 사람이 이미 감염되어 있으므로, 두번째 날의 시한 2에 이 시민은 감염된다.
		\item 시민 (1, 2)에 대해서, 두번째 날의 시한 2에 서쪽에서 바람이 불어오고, 서쪽에 있는 사람이 이미 감염되어 있으므로, 두번째 날의 시한 3에 이 시민은 감염된다.
		\item 시민 (1, 3)에 대해서, 세번째 날의 시한 2에 서쪽에서 바람이 불어오고, 서쪽에 있는 사람이 이미 감염되어 있으므로, 세번째 날의 시한 3에 이 시민은 감염된다.
		\item 시민 (2, 3)에 대해서, 세번째 날의 시한 3에 북쪽에서 바람이 불어오고, 북쪽에 있는 사람이 이미 감염되어 있으므로, 세번째 날의 시한 4에 이 시민은 감염된다.
		\item 시민 (2, 1)에 대해서, 첫번째 날의 시한 1에 남쪽에서 바람이 불어오고, 남쪽에 있는 사람이 이미 감염되어 있으므로, 첫번째 날의 시한 2에 이 시민은 감염된다.
		\item 시민 (3, 3)에 대해서, 네번째 날의 시한 2에 서쪽에서 바람이 불어오고, 서쪽에 있는 사람이 이미 감염되어 있고, 네번째 날의 시한 3에 북쪽에서 바람이 불어오고, 북쪽에 있는 사람이 이미 감염되어 있으므로, 네번째 날의 시한 4에 이 시민은 감염된다.
	\end{itemize}

	더 이상 JOI바이러스에 감염되는 시민은 없을 것이다. 즉 우리가 시민 (3, 1)을 처음 감염되는 사람으로 골랐으면, 8명의 시민이 $10^{100}$일이 지난 후에 JOI바이러스에 감염되는 시민이 될 것이다.
	
	어떤 시민을 고르든, 감염되는 시민의 수를 8명 보다 적게 할 수는 없으므로, 8을 첫번째 줄에 출력한다. 우리가 처음 감염되는 시민을 (1, 1), (1, 2), (1, 3), (2, 1), (2, 3), (3, 1), (3, 2), (3, 3)중 하나로 고르면, $10^{100}$일이 지난 후에 감염되는 사람의 수는 8명이므로, 처음에 고를 수 있는 사람 수인 8을 두번째 줄에 출력한다.
	
	둘째 예제는, 서브태스크 1의 조건을 만족한다.
	
\end{problem}

