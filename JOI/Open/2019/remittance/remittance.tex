\begin{problem}{송금}
	{standard input}{standard output}
	{1 second}{256 megabytes}{}
	
	JOI왕국의 비버호 근처에는 $N$개의 집이 있고, 1번부터 $N$번까지 반시계방향으로 번호가 붙어 있다.
	
	각 집은 호수에서 바라봤을 때 왼쪽에 있는 집으로 송금 서비스를 사용하여 돈을 보낼 수 있다. ($i$번 집 ($1 \le i \le N-1$)은 $i+1$번 집이, $N$번 집은 1번이, 왼쪽에 있는 집이다.) 하지만, 돈을 보내려고 할 때, 보내는 돈의 양과 같은 만큼의 돈을 이용료로 지불해야 한다. 보내는 돈은 1엔 단위여야 한다. 돈을 보낼때에는, 이용료를 무조건 지불해야 하고, 이용료와 보내는 돈의 양을 합쳐서 집에 있는 돈의 양을 넘을 수는 없다.
	
	현재 $i$번 집 $(1 \le i \le N$)에는 $A_i$엔이 있다. 반면, 세금 기준에 따르면 $i$번 집에 $B_i$엔이 있는 것을 원한다. 송금 서비스를 이용 하여, $i$번 집의 돈의 양을 $B_i$로 만들고 싶다. 송금 서비스의 이용료를 제외하고는 돈을 사용할 수는 없다.
	
	현재 집이 가지고 있는 돈과 만들고 싶은 돈의 양이 주어졌을 때, 돈의 양을 원하는 대로 맞출 수 있는지를 판단하는 프로그램을 작성하여라.
	
	\InputFile
	
	표준 입력에서 다음과 같은 형식으로 주어진다.
	
	$N$
	
	$A_1$ $B_1$
	
	$\vdots$
	
	$A_N$ $B_N$
	
	\OutputFile
	
	송금 서비스를 이용하여 돈의 양을 원하는 대로 맞출 수 있으면 \texttt{Yes}, 아니면 \texttt{No}를 출력하여라.
	
	\Constraints

	\begin{itemize}
		
		\item $2 \le N \le 1\ 000\ 000$.
		\item $0 \le A_i \le 1\ 000\ 000\ 000$ ($1 \le i \le N$).
		\item $0 \le B_i \le 1\ 000\ 000\ 000$ ($1 \le i \le N$).
		
	\end{itemize}
	
	
	\SubtaskWithCost{1}{15}
	\begin{itemize}
		\item $N \le 7$
		\item $A_i \le 5$ ($1 \le i \le N$)
		\item $B_i \le 5$ ($1 \le i \le N$)
	\end{itemize}
	
	\SubtaskWithCost{2}{40}
	\begin{itemize}
		\item $N \le 20$
	\end{itemize}
	
	\SubtaskWithCost{3}{45}
	
	추가 제한조건이 없다.
		
	\Examples
		
	\begin{example}
	\exmp{
5
0 0
1 0
2 3
3 3
4 0
	}{%
Yes
	}%
	\exmp{
5
0 0
1 2
2 4
3 2
4 0
	}{%
No
	}%
	\exmp{
	2
	1 1
	2 1
}{%
	No
}%
	\exmp{
	2
	1 1
	2 2
}{%
	Yes
}%
	\end{example}
	
	\Notes
	
	첫 번째 예제에서는, 송금 서비스를 다음과 같은 방법으로 이용하면 돈의 양을 원하는 대로 맞출 수 있다.
	
	\begin{itemize}
		\item 2엔을 5번 집에서 1번 집으로 보낸다. 2엔을 이용료로 낸다.
		\item 1엔을 1번 집에서 2번 집으로 보낸다. 1엔을 이용료로 낸다.
		\item 1엔을 2번 집에서 3번 집으로 보낸다. 1엔을 이용료로 낸다.
	\end{itemize}

	두 번째 예제에서는, 송금 서비스를 이용하여 돈의 양을 원하는 대로 맞출 수 없다.
	
	세 번째 예제에서는, 보내는 돈의 단위가 1엔 단위임에 주의하여라.
	
	네 번째 예제에서는, 송금 서비스를 아예 사용할 필요가 없다.
	
	
	
\end{problem}

