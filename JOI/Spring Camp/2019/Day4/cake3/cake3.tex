\begin{problem}{케이크 3}
	{standard input}{standard output}
	{4 seconds}{256 megabytes}{}
	
	오늘은 IOI양의 생일이다. 이 날을 위해 JOI군은 생일 케이크를 예약했다. 원형 케이크 하나를 통채로 예약할 생각이었지만 착오가 있어서 $N$조각의 케이크를 예약해 버렸다. 각 조각에는 1번 부터 $N$번까지 번호가 붙어 있고, $i$ 번째 ($1 \le i \le N$) 조각의 가치는 $V_i$이고, 색의 짙음은 $C_i$이다.
	
	JOI군은 서로 다른 $M$개의 케이크 골라, 원하는 순서대로 배열해 합쳐서 원형 케이크를 만들기로 결심했다. 케이크 조각들이 $k_1$번, $\cdots$, $k_M$번 조각의 순서로 나열되어 있을 때, 이 케이크의 아름다움은 
	
	$$
	\sum_{j=1}^{M} {V_{k_j}} - \sum_{j=1}^{M} {\left| C_{k_j} - C_{k_{j+1}}\right|}
	$$
	
	으로 정의된다. (단, $k_{M+1} = k_1$ 이다.) 즉, 아름다움은, 사용된 케이크 조각의 가치의 합으로 부터 인접한 두 케이크의 색의 짙음에 차의 절댓값의 합계로 정의된다. JOI군은 되도록이면 원형 케이크의 가치의 합을 최대로 하고싶다.
	
	케이크 조각의 갯수, 각 케이크 조각의 가치와 색의 짙음, 원형 케이크를 만들기 위해 필요한 조각의 갯수가 주어졌을 때, JOI군이 만들 수 있는 원형 케이크의 아름다움의 최댓값을 구하는 프로그램을 작성하여라.
	
	\InputFile
	
	표준 입력에서 다음과 같은 형식으로 주어진다. 모든 입력은 정수이다.
	
	$N$ $M$

	$V_1$ $C_1$
	
	$\vdots$
	
	$V_N$ $C_N$
	
	\OutputFile
	
	표준 출력으로 한 개의 줄에 하나의 수를 출력하여라. 이는 JOI군이 만들 수 있는 원형 케이크의 아름다움의 최댓값이다.
		
	\Constraints
	
	\begin{itemize}
		\item $3 \le N \le 200\ 000$.
		\item $3 \le M \le N$.
		\item $1 \le V_i \le 1\ 000\ 000\ 000$ ($1 \le i \le N$).
		\item $1 \le C_i \le 1\ 000\ 000\ 000$ ($1 \le i \le N$).
	\end{itemize}
	
	
	\SubtaskWithCost{1}{5}
	\begin{itemize}
		\item $N \le 100$.
	\end{itemize}


	\SubtaskWithCost{2}{19}
	\begin{itemize}
		\item $N \le 2\ 000$.
	\end{itemize}


	\SubtaskWithCost{3}{76}
	추가 제한조건이 없다.
	
	\Examples
	
	\begin{example}
		\exmp{
			5 3
			2 1
			4 2
			6 4
			8 8
			10 16
		}{%
			6
		}%
	\end{example}
	
	JOI군이 조각 1번, 3번, 2번 조각을 골라서 순서대로 붙이면, 조각의 가치의 합은 $2+6+4=12$ 이고, 케이크의 짙음의 차이 합은 $|1-4|+|4-2|+|2-1|=6$이다. 그래서 원형 케이크의 아름다움은 $12-6=6$이다.
	
	또한, 2번, 3번, 4번 조각을 골라서 순서대로 붙여서 아름다움이 6인 원형 케이크를 만들 수도 있다.
	
	더 아름다운 원형 케이크를 만들 수는 없기 때문에, 6을 출력해야 한다.
	\begin{example}
		\exmp{
8 4
112103441 501365808
659752417 137957977
86280801 257419447
902409188 565237611
965602301 689654312
104535476 646977261
945132881 114821749
198700181 915994879
}{%
2323231661
}%
	\end{example}
	
	
\end{problem}

