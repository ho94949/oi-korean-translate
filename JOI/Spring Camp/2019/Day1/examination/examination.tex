\begin{problem}{시험}
	{standard input}{standard output}
	{3 seconds}{1024 megabytes}{}
	
	$N$명의 학생이 수학 부문과 정보 부문이 있는 시험을 쳤다. $i$번째 ($1 \le i \le N$) 학생은 수학에서는 $S_i$점을, 정보에서는 $T_i$점을 받았다. T교수와 I교수는 각 학생이 시험을 통과할지 말지를, 점수를 기반으로 정하려고 한다.
	
	\begin{itemize}
		\item T 교수는 두 과목을 모두 중요하게 본다. 수학에서 $A$점, 정보에서 $B$점을 받아야만 통과한 것으로 생각한다.
		\item I 교수는 총점만 중요하게 본다. 수학과 정보를 합쳐서 $C$점 받아야 통과한 것으로 생각한다.
		\item 두 교수의 기준을 모두 통과한 학생만 시험을 통과할 수 있다.
	\end{itemize}

	당신은 기준인 $A$, $B$, $C$를 모른다. 하지만, $Q$가지의 세 정수 $(X_j,\ Y_j,\ Z_j)$ ($1 \le j \le Q$) 가 주어져서 몇 명의 학생들이 $A=X_j,\ B=Y_j,\ C=Z_j$일 때 시험을 통과하는지 알고 싶다.
	
	학생들의 수, 점수 정보와 점수 기준이 주어졌을 때, 이 조건 하에서 시험을 통과하는 학생의 수를 구하여라.

	\InputFile
	
	표준 입력에서 다음과 같은 형식으로 주어진다. 모든 값은 정수이다.

	$N$ $Q$
	
	$S_1$ $T_1$
	
	$\vdots$
	
	$S_N$ $T_N$

	$X_1$ $Y_1$ $Z_1$

	$\vdots$
	
	$X_Q$ $Y_Q$ $Z_Q$

	
	\OutputFile
	
	표준 출력으로 $Q$개의 줄을 출력하여라. $j$번째 ($1 \le j \le Q$)줄은 몇 명의 학생들이 $A=X_j,\ B=Y_j,\ C=Z_j$일 때 시험을 통과하는 학생 수이다.
	
	\Constraints
	
	\begin{itemize}
	
	\item $1 \le N \le 100\ 000$.
	\item $1 \le Q \le 100\ 000$
	\item $0 \le S_i \le 1\ 000\ 000\ 000$ ($1 \le i \le N$).
	\item $0 \le T_i \le 1\ 000\ 000\ 000$ ($1 \le i \le N$).
	\item $0 \le X_j \le 1\ 000\ 000\ 000$ ($1 \le j \le Q$).
	\item $0 \le Y_j \le 1\ 000\ 000\ 000$ ($1 \le j \le Q$).
	\item $0 \le Z_j \le 2\ 000\ 000\ 000$ ($1 \le j \le Q$).
		
	\end{itemize}
	
	
	\SubtaskWithCost{1}{2}
	\begin{itemize}
		\item $N \le 3\ 000$
		\item $Q \le 3\ 000$
	\end{itemize}
	
	\SubtaskWithCost{2}{20}
	\begin{itemize}
		\item $S_i \le 100\ 000$ ($1 \le i \le N$).		
		\item $T_i \le 100\ 000$ ($1 \le i \le N$).
		\item $X_j \le 100\ 000$ ($1 \le j \le Q$).
		\item $Y_j \le 100\ 000$ ($1 \le j \le Q$).
		\item $Z_j = 0$ ($1 \le j \le Q$).
	\end{itemize}
	
	
	\SubtaskWithCost{3}{21}
	\begin{itemize}
		\item $S_i \le 100\ 000$ ($1 \le i \le N$).		
		\item $T_i \le 100\ 000$ ($1 \le i \le N$).
		\item $X_j \le 100\ 000$ ($1 \le j \le Q$).
		\item $Y_j \le 100\ 000$ ($1 \le j \le Q$).
		\item $Z_j \le 200\ 000$ ($1 \le j \le Q$).
	\end{itemize}
	
	\SubtaskWithCost{4}{57}
	
	추가 제한조건이 없다.
	
	\Examples
		
	\begin{example}
	\exmp{
5 4
35 100
70 70
45 15
80 40
20 95
20 50 120
10 10 100
60 60 80
0 100 100
	}{%
2
4
1
1
	}%
\exmp{
10 10
41304 98327
91921 28251
85635 59191
30361 72671
28949 96958
99041 37826
10245 2726
19387 20282
60366 87723
95388 49726
52302 69501 66009
43754 45346 3158
25224 58881 18727
7298 24412 63782
24107 10583 61508
65025 29140 7278
36104 56758 2775
23126 67608 122051
56910 17272 62933
39675 15874 117117
}{%
1
3
5
8
8
3
3
3
5
6
}%
	\end{example}

	\Note
	
	첫째 예제에서
	
	\begin{itemize}
		\item $A=20,\ B=50,\ C=120$일 때, 첫 번째와 두 번째 학생만 수학 부문에서 최소 20점, 정보 시험에서 최소 50점, 그리고 총점 120점을 넘길 수 있다. 그래서 시험을 통과하는 학생들의 수는 2이다.
		\item $A=10,\ B=10,\ C=100$일 때, 첫 번째, 두 번째, 네 번째 그리고 다섯 번째 학생만 수학 부문에서 최소 10점, 정보 시험에서 최소 10점, 그리고 총점 100점을 넘길 수 있다. 그래서 시험을 통과하는 학생들의 수는 4이다.
		\item $A=60,\ B=60,\ C=80$일 때, 두 번째 학생만 수학 부문에서 최소 60점, 정보 시험에서 최소 60점, 그리고 총점 80점을 넘길 수 있다. 그래서 시험을 통과하는 학생들의 수는 1이다.
		\item $A=0,\ B=100,\ C=100$일 때, 첫 번째 학생만 수학 부문에서 최소 0점, 정보 시험에서 최소 100점, 그리고 총점 100점을 넘길 수 있다. 그래서 시험을 통과하는 학생들의 수는 1이다.
	\end{itemize}
	
	
	
	
\end{problem}

