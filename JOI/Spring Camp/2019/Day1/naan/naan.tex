\begin{problem}{난}
	{standard input}{standard output}
	{4초}{256MB}{}
	
	JOI 카레 매점은 매우 긴 난(인도의 납작한 빵)을 판매하는 것으로 유명하다. 난에는 $L$ 개의 맛이 있으며, 1번부터 $L$번까지 번호가 붙어 있다. 난 중에서 ``JOI 스페셜 난"이 제일 인기가 있다. 길이가 $L$ cm 이고, 왼쪽에서 $j-1$ cm부터 $j$ cm까지의 부분에는 $j$번 ($1\le j \le L$) 맛으로 되어 있다.
	
	$N$명의 사람이 JOI 카레 매점에 왔다. 그들의 취향은 다른 사람과 다르다. 구체적으로, $i$ 번째 ($1 \le i \le N$) 사람이 $j$번 ($1 \le j \le L$) 맛의 난을 먹었을 경우에는, 1cm당 $V_{i, j}$의 행복도를 얻을 것이다.
	그들은 하나의 JOI 스페셜 난을 주문했다. 그들은 난을 다음과 같은 방법으로 나누어 가질 것이다.
	
	\begin{enumerate}
		\item $0 < X_1 < X_2 < \cdots < X_{N-1} < L$을 만족하는 $N-1$개의 분수 $X_1,\ \cdots,\ X_{N-1}$를 고른다.
		\item $N$ 개의 정수 $P_1,\ \cdots, \ P_N$을 고른다. 이는 $1, \ \cdots, \ N$의 순열이어야 한다.
		\item 각 $k$ ($1 \le k \le N-1$)에 대해서, 난을 $X_k$ cm 지점에서 자른다. 난은 $N$개의 조각으로 나누어질 것이다.
		\item 각 $k$ ($1 \le k \le N$)에 대해서, $P_k$ 번째 사람에게 $X_{k-1}$ cm와 $X_k$ cm 사이의 조각을 준다. 우리는 $X_0$을 0, $X_N$을 $L$이라고 생각할 것이다.
	\end{enumerate}

	우리는 난을 공평하게 나누고 싶다. 우리는 각 사람이 혼자 JOI 스페셜 난을 모두 먹었을 때 얻는 행복도의 $1/N$ 이상을 얻었을 경우, 분배 방식이 \textbf{공평하다}고 할 것이다.
	
	$N$ 명의 사람의 선호가 주어졌을 때, 난을 공평하게 나누는 방법이 있는가를 출력하여라. 있는 경우, 난을 공평하게 나누는 방법에 대해 출력하여라.
	
	\InputFile
	
	표준 입력에서 다음과 같은 형식으로 주어진다. 모든 수는 정수이다.
	
	$N$ $L$
	
	$V_{1,1}$ $V_{1, 2}$ $\cdots$ $V_{1, L}$
	
	$\vdots$
	
	$V_{N,1}$ $V_{N, 2}$ $\cdots$ $V_{N, L}$
	
	\OutputFile
	
	난을 공평하게 나누는 방법이 없다면, \texttt{-1}을 첫째 줄에 출력하여라. 공평하게 나눌 수 있다면, 나누는 방법을 나타내는 $N-1$ 개의 분수 $X_1,\ \cdots,\ X_{N-1}$과 $N$ 개의 정수 $P_1, \cdots, P_N$을 다음 형식으로 출력하여라.
	
	$A_1$ $B_1$

	$A_2$ $B_2$

	$\vdots$ 
	
	$A_{N-1}$ $B_{N-1}$
	
	$P_1$ $P_2$ $\cdots$ $P_N$
	
	$A_i$, $B_i$는 $X_i = \dfrac{A_i}{B_i}$ ($1 \le i \le N$)를 만족하는 정수 쌍이다. 이 정수는 출력 제한을 따라야 한다.
	
	\Constraints


	{
	\large
	\textbf{입력 제한}
	}

	\begin{itemize}
		
		\item $1 \le N \le 2000$.
		\item $0 \le L \le 2000$.
		\item $1 \le V_{i, j} \le 100\ 000$ ($1 \le i \le N,\ 1 \le j \le L$).
	\end{itemize}

	{
	\large
	\textbf{출력 제한}
	}

	
	난을 공평한 방식으로 나눈 방법이 존재한다면, 출력은 다음 제한을 따라야 한다.
	
	\begin{itemize}
		\item $1 \le B_i \le 1\ 000\ 000\ 000$. ($1 \le i \le N$)
		\item $0 \le \dfrac{A_1}{B_1} < \dfrac{A_2}{B_2} \cdots < \dfrac{A_{N-1}}{B_{N-1}} < L$.
		\item $P_1, \ \cdots, \ P_N$은 $1, \ \cdots, \ N$의 순열이다.
		\item 분배에서, $i$ 번째 사람이 가지는 행복도의 양은 $\dfrac{V_{i, 1}+V_{i,2}+\cdots+V_{i,L}}{N}$ 이상이어야 한다.
	\end{itemize}
	
	$A_i$와 $B_i$는 서로소일 필요는 없다.
	공평한 분배가 존재하는 경우 $1 \le B_i \le 1\ 000\ 000\ 000$을 만족하는 출력이 존재함을 증명할 수 있다.
	
	\SubtaskWithCost{1}{5}
	\begin{itemize}
		\item $N = 2$
	\end{itemize}
	
	\SubtaskWithCost{2}{24}
	\begin{itemize}
		\item $N \le 6$ 
		\item $V_{i, j} \le 10$ ($1 \le i \le N,\ 1 \le j \le L$)
	\end{itemize}
	
	\SubtaskWithCost{3}{71}
	
	추가 제한조건이 없다.
		
	\Examples
		
	\begin{example}
	\exmp{
2 5
2 7 1 8 2
3 1 4 1 5
	}{%
14 5
2 1
	}%
\end{example}

	이 예제에서, 모든 난을 먹었을 때 첫 번째 사람은 2 + 7 + 1 + 8 + 2 = 20의 행복도를 가지고 두 번째 사람은 3 + 1 + 4 + 1 + 5 = 14의 행복도를 가진다. 즉, 첫 번째 사람이 $\dfrac{20}{2} = 10$ 이상의 행복도를 가지고 둘째 사람이 $\dfrac{14}{2} = 7$ 이상의 행복도를 가지면 분배는 공평하다.
	
	난을 $\dfrac{14}{5}$cm 에서 나누면, 첫 번째 사람은 $1 \times \dfrac{1}{5} + 8 + 2 = \dfrac{51}{5}$의 행복도를 얻고, 두 번째 사람은 $3 + 1 + 4 \times \dfrac{4}{5} = \dfrac{36}{5}$의 행복도를 얻는다. 그러므로 이것은 공평한 분배이다.
	
	\begin{example}
	\exmp{
		7 1
		1
		2
		3
		4
		5
		6
		7
	}{%
		1 7
		2 7 
		3 7
		4 7
		5 7
		6 7
		3 1 4 2 7 6 5
	}%
\end{example}

	이 예제에서는 맛이 하나 뿐이다. 난을 크기가 같은 7개의 부분으로 자르면 $P_1, \ \cdots, \  P_N$과 관계 없이 분배가 공정하다.
	

	\begin{example}
	\exmp{
		5 3
		2 3 1
		1 1 1
		2 2 1
		1 2 2
		1 2 1
	}{%
		15 28
		35 28
		50 28
		70 28
		3 1 5 2 4
	}%
\end{example}

	$A_i$와 $B_i$가 서로소 일 필요는 없다. ($1 \le i \le N$)
	
\end{problem}

