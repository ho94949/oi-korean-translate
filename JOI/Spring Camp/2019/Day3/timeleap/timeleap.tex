\begin{problem}{시간을 달리는 비타로}
	{standard input}{standard output}
	{3 seconds}{512 megabytes}{}
	
	비버랜드에는 $N$개의 도시가 있다. 이 도시들은 1번부터 $N$번까지 번호가 붙어있다. $i$번째 ($1 \le i \le N-1$) 도로는 $i$번 도시와 $i+1$번 도시를 양방향으로 잇는다. 또한, 비버랜드의 하루는 1 000 000 000개의 단위시간으로 분열되어 있고, 이 단위시간을 \textit{쵸}라고 부른다. 하루가 시작하고 나서 $x$쵸가 지난 시간을 시각 $x$라 부른다. 한 도로를 통과하는 데에는 1쵸가 걸리고, $i$번째 도로는 시각 $L_i$와 시각 $R_i$ 사이에만 통과할 수 있다. 구체적으로, $i$번째 도로를 통과하기 위해서 우리는 도시 $i$나 $i+1$을 $L_i \le x \le R_i -1$ 을 만족하는 시각 $x$에 떠나야 하고, 다른 도시에 시각 $x+1$에 도착해야 한다.
	
	비타로는 비버랜드에 사는 평범한 비버다. 아니, 비버였다 라고 하는게 옳은 것일까. 지각을 자주한 비타로는 이를 개선하려고 한 결과로 시간을 거슬러 올라가는게 가능해 졌다. 이 능력을 한 번 사용하면 1쵸 뒤로 갈 수 있다. 하지만, 어제로 갈 수는 없다. 만약 그가 능력을 시각 0과 시각 1 사이에 사용했다면, 그는 시각 0으로 돌아갈 것이다. 그는 이 기술을 도시에 있을 때 사용할 수 있다. 비타로의 위치는 능력을 사용해도 변하지 않는다.
	
	비타로는 기술을 사용하면 피곤해 진다. 최소한의 기술을 사용하여 이동하는 방법을 찾기 위한 비타로는 $Q$개의 사고실험을 진행했다. 사고 실험의 $j$ 번째 단계에서는, 그는 다음 중 한 행동을 한다:
	
	\begin{itemize}
		\item $P_j$ 번째 도로가 여행될수 있는 시각을 바꾼다. 바뀐 이후에는, 시각 $S_j$와 시각 $E_j$ 사이에만 $P_j$ 번째 도로를 통과할 수 있다.
		\item 그가 $A_j$번 도시, 시각 $B_j$에 있다고 할 때, $C_j$번 도시, 시각 $D_j$로 이동하기 위해 사용해야하는 능력의 수의 최솟값을 구하여라.
	\end{itemize}

	그는 사고실험의 결과를 궁금해한다.
	
	비버랜드의 도시의 수, 도로의 정보, 사고실험의 방법이 주어졌을 때, 사고 실험의 결과를 계산하는 프로그램을 작성하여라.
	
	
	\InputFile
	
	표준 입력에서 다음과 같은 형식으로 주어진다. 모든 값은 정수이다.
	
	$N$ $Q$
	
	$L_1$ $R_1$
	
	$\vdots$
	
	$L_{N-1}$ $R_{N-1}$
	
	(Query 1)
	
	$\vdots$
	
	(Query $Q$)
	
	여기서, (Query $j$)는 공백으로 구분된 4개나 5개의 정수로 이루어져 있다. $T_j$가 첫번째 정수라고 하자. 그러면,
	
	\begin{itemize}
		\item $T_j=1$인 경우, (Query $j$)는 4개의 정수 $T_j$, $P_j$, $S_j$, $E_j$로 이루어져 있다. 이것은, 사고 실험의 $j$번째 단계에서, $P_j$번째 도로를 지날수 있는 시간이 시각 $S_j$와 시각 $E_j$ 사이로 바뀐다는 것을 의미한다.
		\item $T_j=2$인 경우, (Query $j$)는 5개의 정수 $T_j$, $A_j$, $B_j$, $C_j$, $D_j$로 이루어져 있다. 이는, $j$번째 사고 실험에서, 당신의 프로그램이 비타로가 $A_j$번 도시, 시각 $B_j$에 있다고 할 때, $C_j$번 도시, 시각 $D_j$로 이동하기 위해 사용해야하는 능력의 수의 최솟값을 구해야 한다는 것을 의미한다.
	\end{itemize}
	
	\OutputFile
	
	$T_j=2$인 각 단계에 대해서, 사용해야 하는 능력의 수의 최솟값을 한 줄에 하나씩 차례로 출력하여라.
	
	\Constraints
	
	\begin{itemize}
		
		\item $1 \le N \le 300\ 000$.
		\item $1 \le Q \le 300\ 000$.
		\item $0 \le L_i < R_i \le 999\ 999\ 999$ ($q \le i \le N-1$).
		\item $1 \le T_j \le 2$ ($1 \le j \le Q$).
		\item $1 \le P_j \le N-1$ ($1 \le j \le Q$, $T_j = 1$).
		\item $1 \le S_j \le E_j \le 999\ 999\ 999$ ($1 \le j \le Q$, $T_j = 1$).
		\item $1 \le A_j \le N$ ($1 \le j \le Q$, $T_j = 2$).
		\item $1 \le B_j \le 999\ 999\ 999$ ($1 \le j \le Q$, $T_j = 2$).
		\item $1 \le C_j \le N$ ($1 \le j \le Q$, $T_j = 2$).
		\item $1 \le D_j \le 999\ 999\ 999$ ($1 \le j \le Q$, $T_j = 2$).
		
	\end{itemize}
	
	
	\SubtaskWithCost{1}{4}
	\begin{itemize}
		\item $N \le 1000$
		\item $Q \le 1000$
	\end{itemize}
	
	\SubtaskWithCost{2}{30}
	\begin{itemize}
		\item $T_j = 2$ ($1 \le j \le Q$).
	\end{itemize}

	\SubtaskWithCost{3}{66}
	
	추가 제한조건이 없다.
	
	\Examples
	
	\begin{example}
		\exmp{
			3 3
			0 5
			0 5
			2 1 3 3 3 
			1 2 0 1
			2 1 3 3 3 
		}{%
			2
			4
		}%
	\end{example}
	
	사고 실험의 첫 번째 단계에서, 비타로는 1번 도시에서 2번 도시로 1쵸만에 이동하고, 2번 도시에서 3번 도시로 1쵸만에 이동하여 3번 도시, 시각 5에 위치하여 있다. 능력을 두번 사용하면, 그는 3번 도시, 시각 3에 위치할 수 있다.
	
	사고 실험의 두 번째 단계에서, 2번 도시를 통과할 수 있는 시간이 시각 0부터 시각 1까지로 바뀐다.
	
	사고 실험의 세 번째 단계에서, 1번 도시에서 2번 도시로 1쵸 만에 이동하여, 2번 도시, 시각 4에 위치해 있다. 여기서 능력을 네 번 사용여, 3번 도시로 1쵸만에 이동하여 2쵸를 기다리면 3번 도시, 시각 3에 위치할 수 있다.
	
	
	\begin{example}
		\exmp{
5 5
3 5
4 8
2 6
5 10
2 5 3 1 10
2 2 6 5 6
1 3 4 6
2 3 3 4 3
2 4 5 1 5
		}{%
4
3
2
3
		}%
		\exmp{
7 7
112103440 659752416
86280800 902409187
104535475 965602300
198700180 945132880
137957976 501365807
257419446 565237610
2 4 646977260 7 915994878
2 1 221570340 6 606208433
2 7 948545948 4 604273995
2 7 247791098 5 944822313
2 7 250362511 2 50167280
2 3 364109400 4 555412865
2 7 33882587 7 186961394
}{%
145611455
0
447180143
0
207252171
0
0
}%
\exmp{
7 7
535825574 705426142
964175291 996597835
481817391 649559926
4519006 410772613
74521477 274584126
256535565 899389890
1 6 511428966 602601933
1 1 69986642 201421232
2 3 636443425 4 625975977
1 6 235225515 405336399
2 3 866680458 3 701821857
1 6 180606048 900533151
1 6 612564160 720179605
}{%
10467449
164858601
}%
	\end{example}

\end{problem}

