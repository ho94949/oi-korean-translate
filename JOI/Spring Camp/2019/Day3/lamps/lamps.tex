\begin{problem}{램프}
	{standard input}{standard output}
	{1 second}{256 megabytes}{}
	
	긴 복도에 $N$개의 램프가 일렬로 나열되어 있다. 램프는 왼쪽부터 차례로 1번부터 $N$번까지의 번호가 붙어있다. 각 램프는 off또는 on중 하나의 상태이다.
	
	램프의 상태를 바꾸는 특별한 기작이 있어서, 한 번의 작업으로 다음 셋 중 한 가지 동작을 할 수 있다.
	
	\begin{itemize}
		\item $1 \le p \le q \le N$을 만족하는 정수 $p$와 $q$를 골라서 $p$, $p+1$, $\cdots$, $q$를 off 상태로 만든다.
		\item $1 \le p \le q \le N$을 만족하는 정수 $p$와 $q$를 골라서 $p$, $p+1$, $\cdots$, $q$를 on 상태로 만든다.
		\item $1 \le p \le q \le N$을 만족하는 정수 $p$와 $q$를 골라서 $p$, $p+1$, $\cdots$, $q$의 상태를 바꾼다. (off를 on으로, on을 off로)
	\end{itemize} 

	처음에 램프의 상태는 길이 $N$의 문자열 $A$로 표현된다. $A$의 $i$ 번째 ($1 \le i \le N$) 문자가 \texttt{0}이면 $i$ 번째 램프가 off 상태인 것이고, \texttt{1}이면 on 상태인 것이다. 우리는 만들고 싶은 상태가 길이 $N$의 문자열 $B$로 표현 되어 있고, 작업의 수를 최소한으로 하여 만들고 싶다. $B$의 $i$ 번째 ($1 \le i \le N$) 문자가 \texttt{0}이면 $i$ 번째 램프가 off 상태인 것이고, \texttt{1}이면 on 상태인 것이다. 
	
	램프의 수와, 현재 상태와 만들고 싶은 상태가 주어졌을 때, 만들고 싶은 상태로 바꾸는 데에 드는 연산의 수의 최솟값을 출력하여라.

	\InputFile
	
	표준 입력에서 다음과 같은 형식으로 주어진다.
	
	$N$

	$A$

	$B$
	
	
	\OutputFile
	
	표준 출력으로 한 개의 줄을 출력하여라. 이는 원하는 상태를 만들기 위한 연산의 수의 최솟값이다.
		
	\Constraints
	
	\begin{itemize}
		\item $1 \le N \le 1\ 000\ 000$.
		\item $A$와 $B$는 길이 $N$의 문자열이다.
		\item $A$와 $B$를 이루는 문자들은 \texttt{0} 혹은 \texttt{1}이다.
	\end{itemize}
	
	
	\SubtaskWithCost{1}{6}
	\begin{itemize}
		\item $N \le 18$
	\end{itemize}


	\SubtaskWithCost{2}{41}
	\begin{itemize}
		\item $N \le 2000$
	\end{itemize}


	\SubtaskWithCost{3}{4}
	\begin{itemize}
		\item $A$를 이루는 각 문자는 \texttt{0}이다.
	\end{itemize}
	
	\SubtaskWithCost{4}{49}
	
	추가 제한조건이 없다.
	
	\Examples
	
	\begin{example}
		\exmp{
			8
			11011100
			01101001
		}{%
			4
		}%
	\end{example}
	
	이 입력에서 우리는 원하는 상태를 다음과 같은 방법으로 네 번의 작업으로 만들 수 있다.
	
	\begin{enumerate}

		\item 1, 2, 3, 4번 램프의 상태를 바꾼다. 램프의 상태는 \texttt{00101100}이 된다.
		\item 2번 램프를 off 상태로 만든다. 램프의 상태는 \texttt{01101100}이 된다.
		\item 6, 7, 8번 램프의 상태를 바꾼다. 램프의 상태는 \texttt{01101011}이 된다.
		\item 6, 7번 램프를 on 상태로 만든다. 램프의 상태는 \texttt{01101001}이 된다.
	\end{enumerate}
	
	네 번보다 더 적은 작업으로 원하는 상태를 만들 수 있는 방법은 없으므로, 4를 출력한다.
	
	\begin{example}
		\exmp{
13
1010010010100
0000111001011
		}{%
3
		}%
		\exmp{
18
001100010010000110
110110001000100101
}{%
5
}%
	\end{example}
	
	
\end{problem}

