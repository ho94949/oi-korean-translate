\begin{problem}{두 요리}
	{standard input}{standard output}
	{5 seconds}{1024 megabytes}{}
	
	요리사 비타로는 요리 대회에 참여했다. 이 대회에서 참가자는 IOI 돈부리와 JOI 카레를 요리해야 한다.
	
	IOI 돈부리를 요리하는 방법은 $N$단계로 이루어져 있다. $i$ 번째 ($1 \le i \le N$) 단계는 정확히 $A_i$분이 걸린다. 처음에, 그는 첫 번째 단계만 실행할 수 있다. $i$번째 ($2 \le i \le N$) 단계를 실행하려면, $(i-1)$번째 단계를 끝마쳐야 한다.
	
	JOI 카레를 요리하는 방법은 $M$단계로 이루어져 있다. $j$ 번째 ($1 \le j \le M$) 단계는 정확히 $B_j$분이 걸린다. 처음에, 그는 첫 번째 단계만 실행할 수 있다. $j$번째 ($2 \le j \le M$) 단계를 실행하려면, $(j-1)$번째 단계를 끝마쳐야 한다.
	
	각 단계를 집중해야 하기 때문에, 한 단계를 시작하면, 그 단계를 끝날 때 까지 다른 단계를 실행할 수 없다. 한 단계가 끝난 이후에는 다른 요리의 단계를 시작해도 상관 없다. 대회가 시작하면 두 요리가 끝나기 까지의 쉬는 시간은 없다.
	
	이 대회에서는, 각 참가자는 \textbf{예술 점수}를 다음 기준에 따라 받는다.
	
	\begin{itemize}
		\item IOI 돈부리를 만드는 $i$번째 ($1 \le i \le N$) 단계를 처음부터 $S_i$시간 안에 끝냈을 경우 $P_i$점을 얻는다. $P_i$는 음수 일 수도 있다.
		\item JOI 카레를 만드는 $j$번째 ($1 \le j \le M$) 단계를 처음부터 $T_j$시간 안에 끝냈을 경우 $Q_j$점을 얻는다. $Q_j$는 음수 일 수도 있다.
	\end{itemize} 

	비타로는 예술 점수를 최대화 하고 싶다.
	
	요리 단계의 수와, 각 단계에 걸리는 시간과, 예술 점수의 정보가 주어졌을 때, 비타로가 얻을 수 있는 예술 점수의 최댓값을 구하여라.
	
	\InputFile
	
	표준 입력에서 다음과 같은 형식으로 주어진다. 모든 값은 정수이다.
	
	$N$ $M$
	
	$A_1$ $S_1$ $P_1$
	
	$\vdots$
	
	$A_N$ $S_N$ $P_N$
		
	$B_1$ $T_1$ $Q_1$
	
	$\vdots$
	
	$B_M$ $T_M$ $Q_M$
	
	\OutputFile
	
	표준 출력으로 한 개의 줄을 출력하여라. 이는 비타로가 얻을 수 있는 예술 점수의 최댓값이다.
		
	\Constraints
	
	\begin{itemize}
		\item $1 \le N \le 1\ 000\ 000$.
		\item $1 \le M \le 1\ 000\ 000$.
		\item $1 \le A_i \le 1\ 000\ 000\ 000$ ($1 \le i \le N$).
		\item $1 \le B_j \le 1\ 000\ 000\ 000$ ($1 \le j \le M$).
		\item $1 \le S_i \le 2\ 000\ 000\ 000\ 000\ 000 = 2$ × $10^{15}$ ($1 \le i \le N$).
		\item $1 \le T_j \le 2\ 000\ 000\ 000\ 000\ 000 = 2$ × $10^{15}$ ($1 \le j \le M$).
		\item $-1\ 000\ 000\ 000 \le P_i \le 1\ 000\ 000\ 000$ ($1 \le i \le N$).
		\item $-1\ 000\ 000\ 000 \le Q_j \le 1\ 000\ 000\ 000$ ($1 \le j \le M$).
	\end{itemize}
	
	
	\SubtaskWithCost{1}{5}
	\begin{itemize}
		\item $N \le 200\ 000$
		\item $M \le 200\ 000$
		\item $S_1 = \cdots = S_N$
		\item $T_1 = \cdots = T_N$
	\end{itemize}


	\SubtaskWithCost{2}{3}
	\begin{itemize}
		\item $N \le 12$
		\item $M \le 12$
		\item $P_i = 1$ ($1 \le i \le N$)
		\item $Q_j = 1$ ($1 \le j \le M$)
	\end{itemize}



	\SubtaskWithCost{3}{7}
	\begin{itemize}
		\item $N \le 2\ 000$
		\item $M \le 2\ 000$
		\item $P_i = 1$ ($1 \le i \le N$)
		\item $Q_j = 1$ ($1 \le j \le M$)
	\end{itemize}



	\SubtaskWithCost{4}{39}
	\begin{itemize}
		\item $N \le 200\ 000$
		\item $M \le 200\ 000$
		\item $P_i = 1$ ($1 \le i \le N$)
		\item $Q_j = 1$ ($1 \le j \le M$)
	\end{itemize}



	\SubtaskWithCost{5}{11}
	\begin{itemize}
		\item $N \le 200\ 000$
		\item $M \le 200\ 000$
		\item $1 \le P_i$ ($1 \le i \le N$)
		\item $1 \le Q_j$ ($1 \le j \le M$)
	\end{itemize}

	\SubtaskWithCost{6}{9}
	\begin{itemize}
		\item $1 \le P_i$ ($1 \le i \le N$)
		\item $1 \le Q_j$ ($1 \le j \le M$)
	\end{itemize}

	\SubtaskWithCost{7}{17}
	\begin{itemize}
		\item $N \le 200\ 000$
		\item $M \le 200\ 000$
	\end{itemize}

	
	\SubtaskWithCost{8}{9}
	
	추가 제한조건이 없다.
	
	\Examples
	
	\begin{example}
		\exmp{
			5
			10 2 4
			1 1 1
			2 1 3
			1 1 1
			100 1 1
			5
			1 2
			2 3
			1 3
			1 4
			1 5
		}{%
			-1
			1
			8
			8
			99
		}%
	\end{example}
	
	이 입력은 서브태스크 2의 조건을 만족한다.
	
	이 입력에서 비타로는 다음과 같은 방식으로 요리 할 수 있다.
	
	\begin{enumerate}
		\item JOI 돈부리의 첫 번째 단계를 요리한다. 그는 대회가 시작한 후 3분 후에 단계를 끝낸다. 6분 안이므로, 1점을 얻는다.
		\item IOI 카레의 첫 번째 단계를 요리한다. 그는 대회가 시작한 후 5분 후에 단계를 끝낸다. 1분 밖이므로, 아무 점수도 얻지 못한다.
		\item IOI 돈부리의 두 번째 단계를 요리한다. 그는 대회가 시작한 후 8분 후에 단계를 끝낸다. 8분 안이므로, 1점을 얻는다.
		\item JOI 카레의 두 번째 단계를 요리한다. 그는 대회가 시작한 후 10분 후에 단계를 끝낸다. 11분 안이므로, 1점을 얻는다.
		\item IOI 돈부리의 세 번째 단계를 요리한다. 그는 대회가 시작한 후 12분 후에 단계를 끝낸다. 13분 안이므로, 1점을 얻는다.
		\item IOI 돈부리의 네 번째 단계를 요리한다. 그는 대회가 시작한 후 13분 후에 단계를 끝낸다. 13분 안이므로, 1점을 얻는다.
		\item JOI 커리의 세 번째 단계를 요리한다. 그는 대회가 시작한 후 15분 후에 단계를 끝낸다. 15분 안이므로, 1점을 얻는다.
	\end{enumerate}
	
	총 예술점수는 6점이다. 6점 보다 더 많은 점수를 얻을 수는 없으므로 6을 출력해야 한다.
	
	\begin{example}
		\exmp{
5 7
16 73 16
17 73 10
20 73 1
14 73 16
18 73 10
3 73 2
10 73 7
16 73 19
12 73 4
15 73 15
20 73 14
15 73 8
		}{%
		63
		}%
	\end{example}
	
	이 입력은 서브태스크 1의 조건을 만족한다.
	
		\begin{example}
		\exmp{
9 11
86 565 58
41 469 -95
73 679 28
91 585 -78
17 513 -63
48 878 -66
66 901 59
72 983 -70
68 1432 11
42 386 -87
36 895 57
100 164 10
96 812 -6
23 961 -66
54 193 51
37 709 82
62 148 -36
28 853 22
15 44 53
77 660 -19
		}{%
		99
		}%
	\end{example}
	
\end{problem}

