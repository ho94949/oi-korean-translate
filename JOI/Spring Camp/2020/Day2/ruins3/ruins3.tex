\begin{problem}{유적 3}
	{standard input}{standard output}
	{4초}{512MB}{}
	
	JOI 교수는 IOI 왕국 역사 연구의 일인자이다. IOI 왕국에 관련된 신사를 조사하는 중 신사에 건설된 $N$ 개의 기둥을 발견했다. 또한, IOI 왕국의 고대인들이 쓴 것으로 추정되는 고문서를 발견했다. 이 문서에는 기둥에 대한 정보가 쓰여 있었다. 문서에 쓰인 내용은 다음과 같다.
	
	\begin{itemize}
		\item 기둥 건설 직후에 $2N$개의 기둥이 있었고, 1번부터 $2N$번까지의 번호가 붙어있다.
		\item 기둥 건설 직후에 각 $k$에 ($1 \le k \le N$) 대해, 높이가 $k$인 기둥이 정확히 두 개 있었다.
		\item 지진이 $N$번 일어났다. 지진 이후에 몇몇 기둥은 무너졌고 높이가 1 감소했다. 다른 기둥들은 고대인들이 보호했고 기둥이 무너지지 않았기 때문에 높이도 변하지 않았다.
		\item 지진이 일어났을 때 각 $k$에 ($1 \le k \le N$) 대해, 높이가 $k$인 기둥이 정확히 하나가 보호되었다. 만약에 높이가 $k$인 기둥이 두 개 이상 있었을 경우, 가장 번호가 높은 기둥이 보호되었다. 다른 말로, 지진 전의 $i$번째 ($1 \le i \le 2N$) 기둥의 높이를 $h_i$라 할 때, $h_i \ge 1$이고, 모든 $j>i$에 대해 $h_j \ne h_i$를 만족한 경우 $i$ 번째 기둥이 보호되었다.
		\item $N$번의 지진이 일어난 이후 $N$개의 기둥이 남았다. (즉 기둥의 높이가 1 이상인 기둥이 정확히 $N$개 있다.)
	\end{itemize}

	$2N$개 기둥의 높이를 복원할 수 있다면 세기의 대발견이 될 수 있다고 생각한 JOI 교수는 기둥을 좀 더 면밀히 조사했다. 그는 $N$번의 지진이 일어난 이후로 남은 기둥의 번호가 $A_1$, $A_2$, $\cdots$, $A_N$번이라는 것을 알아냈다.
	
	JOI 교수는 기둥이 건설되었을 때 $2N$개 기둥의 높이로 가능한 경우가 몇 가지인지를 알고 싶다. 당신은 JOI 교수의 제자로 경우의 수를 계산하는 프로그램을 작성하는 요구를 받았다.
	
	$N$번의 지진 이후에 남은 기둥의 번호가 주어졌을 때, $2N$개의 기둥의 높이로 가능한 경우의 수를 1 000 000 007로 나눈 나머지를 구하여라.
	
	\InputFile
	
	표준 입력에서 다음과 같은 형식으로 주어진다. 모든 수는 정수이다.
	
	$N$
	
	$A_1$ $\cdots$ $A_N$
	
	\OutputFile
	
	표준 출력에 하나의 줄을 출력하여라. 이 줄은 답을 1 000 000 007로 나눈 나머지이다.
	
	

	\Constraints


	\begin{itemize}
		
		\item $1 \le N \le 600$.
		\item $1 \le A_i \le 2N$ ($1 \le i \le N$).
		\item $A_i < A_{i+1}$ ($1 \le i \le N-1$).
	\end{itemize}


	\SubtaskWithCost{1}{6}
	\begin{itemize}
		\item $N \le 13$.
	\end{itemize}

	
	\SubtaskWithCost{2}{52}
	\begin{itemize}
		\item $N \le 60$.
	\end{itemize}
	
	\SubtaskWithCost{3}{42}
	
	추가 제한조건이 없다.
		
	\Examples
		
	\begin{example}
	\exmp{
3
3 4 6
	}{%
5
	}%
\end{example}

	이 예제에서, 예를 들어 기둥의 높이가 (2, 2, 3, 3, 1, 1)이었다고 하자. 각 $k$에 ($1 \le k \le 3$) 대해 높이가 $k$인 기둥이 정확히 두 개 존재하므로 고문서에 쓰인 것과 일치한다.

	\begin{itemize}
		\item 첫 번째 지진 이후 2, 4, 6번 기둥이 고대인들에 의해 보호되었다. 높이는 (1, 2, 2, 3, 0, 1)이 되었다.
		\item 두 번째 지진 이후 3, 4, 6번 기둥이 고대인들에 의해 보호되었다. 높이는 (0, 1, 2, 3, 0, 1)이 되었다.
		\item 세 번째 지진 이후 3, 4, 6번 기둥이 고대인들에 의해 보호되었다. 높이는 (0, 0, 2, 3, 0, 1)이 되었다.

	\end{itemize}

	세 지진 이후 3, 4, 6번 기둥이 남았고 입력에 주어진 정보와 같다. 또한, 기둥의 높이가 (2, 3, 2, 3, 1, 1), (2, 3, 3, 2, 1, 1), (3, 2, 2, 3, 1, 1), (3, 2, 3, 2, 1, 1)인 경우도 가능하다.
	
	그러므로, 총 5가지의 기둥 높이가 고문서에 쓰인 것과 입력으로 주어진 정보와 일치한다.


	\begin{example}
	\exmp{
		1
		1
	}{%
		0
	}%
	\end{example}

	이 예제에서, (1, 1)이 고문서와 일치하는 유일한 높이이다. 첫 번째 지진 이후 기둥의 높이는 (0, 1)이 된다.
	
	그러므로 고문서에 쓰인 것과 입력을 모두 만족 시키는 높이는 존재하지 않는다.
	

	\begin{example}
	\exmp{
		10
		5 8 9 13 15 16 17 18 19 20
	}{%
		147003663
	}%
	\end{example}

	건설되었을 때 가능한 $2N$개의 기둥 높이로 111 147 004 440가지가 가능하다. 111 147 004 440을 1 000 000 007로 나눈 나머지는 147 003 663이다. 그러므로 147 003 663을 출력한다.

	
\end{problem}

