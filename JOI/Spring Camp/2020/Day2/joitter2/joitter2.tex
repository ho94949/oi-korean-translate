\begin{problem}{조이터에서 친구를 만드는건 재밌어}
	{standard input}{standard output}
	{3초}{1024MB}{}
	
	조이터는 당신의 추억을 친구와 남길 수 있는 소셜 미디어 서비스이다.
	
	조이터에서 당신은 다른 사용자를 \textbf{팔로우}할 수 있다. 사용자 $a$가 사용자 $b$를 팔로우하면, 사용자 $a$는 사용자 $b$의 게시글을 자신의 타임라인에서 읽을 수 있다. 이 경우에 사용자 $b$가 사용자 $a$를 팔로우할 수도, 아닐 수도 있다. 하지만, 사용자 $a$가 특정 사용자 $b$를 한 번 넘게 팔로우할 수는 없다.
	
	1번, 2번, $\cdots$, $N$번 사용자 총 $N$명의 사용자가 조이터를 시작했다. 처음에는 아무도 서로를 팔로우하고 있지 않다.
	
	이제부터 총 $M$일 동안 $i$ 번째 ($1 \le i \le M$) 날에 $A_i$번 사용자가 $B_i$번 사용자를 팔로우한다. 
	
	조이터는 \textbf{교류 이벤트}를 $M$일 동안 열려고 한다. 교류 이벤트는 다음과 같은 방법으로 이루어진다.
	
	\begin{enumerate}
		\item 사용자 한 명을 고른다. 이 사용자를 $x$라고 하자.
		\item 사용자 $x$가 현재 팔로우하고 있는 사용자 중 한 명을 고른다. 이 사용자를 $y$라고 하자.
		\item 다음 조건을 만족하도록 사용자 $z$를 고른다: $z$는 $x$와 다르며, $x$가 $z$를 팔로우하고 있지 않고, $y$가 $z$를 팔로우하고, $z$가 $y$를 팔로우한다.
		\item $x$가 $z$를 팔로우한다.
		\item 조건을 만족하는 $(x, y, z)$가 없을 때까지 위 과정을 반복한다.
	\end{enumerate}
	
	교류 이벤트가 언제 진행될지는 결정되지 않았다. 그래서 각 $M$ 개의 날에 대해 어떤 사용자가 다른 사용자를 팔로우한 이후 교류 이벤트가 개최되었을 때, 교류 이벤트가 종료된 시점에 $N$명이 팔로우하고 있는 사람 수의 총합의 최댓값을 구하고 싶다. 단, 교류 이벤트는 시작한 날에 종료되는 것으로 한다.
	
	사용자의 수와 $M$일간의 팔로우 정보가 주어졌을 때, 각 일에 교류 이벤트가 열린 직후에 $N$명이 팔로우하는 사람 수의 총합의 최댓값을 구하는 프로그램을 작성하여라.
	
	
\InputFile

표준 입력에서 다음과 같은 형식으로 주어진다. 모든 값은 정수이다.

$N$ $M$

$A_1$ $B_1$

$\vdots$

$A_M$ $B_M$

\OutputFile

$M$ 개의 줄을 표준 출력으로 출력하여라. $i$ 번째 ($1 \le i \le M$) 줄에는, $i$ 번째에 어떤 사용자가 다른 사용자를 팔로우한 이후 교류 이벤트가 개최되었을 때, 교류 이벤트가 종료된 시점에 $N$명이 팔로우하는 사람 수의 총합의 최댓값을 출력하여라.

\Constraints

\begin{itemize}
	\item $2 \le N \le 100\ 000$.
	\item $1 \le M \le 300\ 000$.
	\item $1 \le A_i \le N$ ($1 \le i \le M$).
	\item $1 \le B_i \le N$ ($1 \le i \le M$).
	\item $A_i \ne B_i$ ($1 \le i \le M$).
	\item $(A_i, B_i) \ne (A_j, B_j)$ ($1 \le i < j \le M$).
\end{itemize}


\SubtaskWithCost{1}{1}
\begin{itemize}
	\item $N \le 50$.
\end{itemize}

\SubtaskWithCost{2}{16}
\begin{itemize}
	\item $N \le 2\ 000$.
\end{itemize}

\SubtaskWithCost{3}{83}

추가 제한조건이 없다.

\Examples

\begin{example}
	\exmp{
		4 6
		1 2
		2 3
		3 2
		1 3
		3 4
		4 3
	}{%
		1
		2
		4
		4
		5
		9
	}%
\end{example}

\begin{itemize}
\item 첫째 날에 1번 사용자가 2번 사용자를 팔로우한다. 이후 교류 이벤트를 개최해도, 새롭게 누군가가 다른 누군가를 팔로우하는 일이 없이, 팔로우 수 합계는 1이다.
\item 둘째 날에 2번 사용자가 3번 사용자를 팔로우한다. 이후 교류 이벤트를 개최해도, 새롭게 누군가가 다른 누군가를 팔로우하는 일이 없이, 팔로우 수 합계는 2이다.
\item 셋째 날에 3번 사용자가 2번 사용자를 팔로우한다. 이후 교류 이벤트를 개최 하면, 예를 들면 1번 사용자가 3번 사용자를 팔로우 할 수 있다. 이때 팔로우 수 합계는 4가 되고, 이 값이 최댓값이다.
\item 넷째 날에 1번 사용자가 3번 사용자를 팔로우한다. 이후 교류 이벤트를 개최해도, 새롭게 누군가가 다른 누군가를 팔로우하는 일이 없이, 팔로우 수 합계는 4이다.
\item 다섯째 날에 3번 사용자가 4번 사용자를 팔로우한다. 이후 교류 이벤트를 개최해도, 새롭게 누군가가 다른 누군가를 팔로우하는 일이 없이, 팔로우 수 합계는 5이다.
\item 여섯째 날에 4번 사용자가 3번 사용자를 팔로우한다. 이후 교류 이벤트를 개최 하면, 예를 들면 1번 사용자가 3번 사용자를 팔로우하고, 2번 사용자가 4번 사용자를 팔로우하고, 4번 사용자가 2번 사용자를 팔로우한다. 이때 팔로우 수 합계는 9가 되고, 이 값이 최댓값이다.
\end{itemize}
\begin{example}
	\exmp{
		6 10
		1 2
		2 3
		3 4
		4 5
		5 6
		6 5
		5 4
		4 3
		3 2
		2 1
	}{%
		1
		2
		3
		4
		5
		7
		11
		17
		25
		30
	}%
\end{example}

\end{problem}

