\begin{problem}{치료 계획}
	{standard input}{standard output}
	{2초}{512MB}{}
	
	JOI 왕국에는 $1$번부터 $N$번까지 번호가 붙은 $N$개의 집이 있다. 이 집은 번호순으로 일렬로 놓여있고, 각 집에는 한 명의 국민이 살고 있다. $x$번 ($1 \le x \le N$) 집에 사는 국민을 $x$번 국민이라고 한다.
	
	지금 신종 바이러스가 발생해서 모든 국민이 바이러스에 감염되었다. 이 문제를 해결하기 위해, $M$개의 치료 계획이 제안되었다. $i$ 번째 ($1 \le i \le M$) 치료 계획의 비용은 $C_i$이다. $i$ 번째 치료 계획이 실행되면, 다음 일이 일어난다.
	
	\begin{itemize}
		\item[] 지금으로부터 $T_i$일 후 저녁에, $L_i \le x \le R_i$를 만족하는 $x$번 국민이 바이러스에 감염된 경우, 그 국민은 치료된다.
	\end{itemize}

	바이러스는 다음과 같은 방식으로 인접한 국민에게 퍼진다.
	
	\begin{itemize}
		
		\item[] 만약 $x$번 ($1 \le x \le N$) 국민이 어느 날 아침 바이러스에 감염 되었다면, ($x \ge 2$인 경우) $x-1$번 국민과 ($x\le N-1$인 경우) $x+1$번 국민은 같은 날 정오에 감염된다.
	\end{itemize}

	이미 치료된 국민도 다시 바이러스에 감염될 수 있다.
	
	당신은 JOI 왕국의 총리이기 때문에, 다음 조건을 만족하도록 치료 계획 중 몇 개를 골라야 한다.
	
	\begin{itemize}
		\item[] \textbf{조건} 계획이 모두 실행 된 이후, 어떤 국민도 바이러스에 감염되어 있지 않다.
	\end{itemize}

	단, 같은 날에 두 개 이상의 계획을 실행하는 것도 가능하다.
	
	집의 수와 치료 계획의 정보가 주어졌을 때, 위 조건을 만족하면서 치료하는 것이 가능한지, 가능하다면 필요한 비용 합의 최솟값을 구하여라.
	
	
	
	\InputFile
	
	표준 입력에서 다음과 같은 형식으로 주어진다. 모든 수는 정수이다.
	
	$N$ $M$ 
	
	$T_1$ $L_1$ $R_1$ $C_1$
	
	$\vdots$
	
	$T_M$ $L_M$ $R_M$ $C_M$
	
	\OutputFile
	
	
	표준 출력에 하나의 줄을 출력하여라. 조건을 만족하지 못하는 경우 \texttt{-1}을 출력하여라. 조건을 만족할 수 있는 경우, 필요한 비용 합의 최솟값을 출력하여라.
	
	
	

	\Constraints


	\begin{itemize}
		
		\item $1 \le N \le 1\ 000\ 000\ 000$.
		\item $1 \le M \le 100\ 000$.
		\item $1 \le T_i \le 1\ 000\ 000\ 000$ ($1 \le i \le M$).
		\item $1 \le L_i \le R_i \le N$ ($1 \le i \le M$).
		\item $1 \le C_i \le 1\ 000\ 000\ 000$ ($1 \le i \le M$).
	\end{itemize}


	\SubtaskWithCost{1}{4}
	\begin{itemize}
		\item $T_i = 1$ ($1 \le i \le M$).
	\end{itemize}

	
	\SubtaskWithCost{2}{5}
	\begin{itemize}
		\item $M \le 16$.
	\end{itemize}


	\SubtaskWithCost{3}{30}
	\begin{itemize}
		\item $M \le 5000$.
	\end{itemize}
	
	\SubtaskWithCost{4}{61}
	
	추가 제한조건이 없다.
		
	\Examples
		
	\begin{example}
	\exmp{
10 5
2 5 10 3
1 1 6 5
5 2 8 3
7 6 10 4
4 1 3 1
	}{%
7
	}%
\end{example}

	이 예제에서, 다음과 같이 치료 계획을 진행 할 수 있다.

	\begin{itemize}
		\item 둘째 날 저녁, 1번 치료 계획 실행되어 5, 6, 7, 8, 9, 10번 국민이 치료되었다. 이제 1, 2, 3, 4번 국민은 바이러스에 감염되었다.
		\item 셋째 날 정오, 5번 국민이 바이러스에 감염되었다. 이제 1, 2, 3, 4, 5번 국민은 바이러스에 감염되었다.
		\item 넷째 날 정오, 6번 국민이 바이러스에 감염되었다. 이제 1, 2, 3, 4, 5, 6번 국민은 바이러스에 감염되었다.
		\item 넷째 날 저녁, 5번 치료 계획이 실행되어 1, 2, 3번 국민이 치료되었다. 이제 4, 5, 6번 국민은 바이러스에 감염되었다.
		\item 다섯째 날 정오, 7번 국민이 바이러스에 감염되었다. 이제 3, 4, 5, 6, 7번 국민은 바이러스에 감염되었다.
		\item 다섯째 날 저녁, 3번 치료 계획이 실행되어 3, 4, 5, 6, 7번 국민이 치료되었다. 이제 바이러스에 감염된 국민은 없다.
		
	\end{itemize}

	1, 3, 5번 치료 계획을 실행하는 데에는 총 7의 비용이 든다. 조건을 만족하면서 7보다 더 작은 비용으로 치료 계획을 실행하는 것은 불가능하기 때문에, 7을 출력한다.

	\begin{example}
	\exmp{
		10 5
		2 6 10 3
		1 1 5 5
		5 2 7 3
		8 6 10 4
		4 1 3 1
	}{%
		-1
	}%
	\end{example}

	이 예제에서, 조건을 만족시킬 수 없기 때문에 $-1$을 출력한다.

	\begin{example}
	\exmp{
		10 5
		1 5 10 4
		1 1 6 5
		1 4 8 3
		1 6 10 3
		1 1 3 1
	}{%
		7
	}%
	\end{example}

	이 예제는 서브태스크 1의 조건을 만족한다.
	
\end{problem}

