\begin{problem}{수도}
	{standard input}{standard output}
	{5 seconds}{512 megabytes}{}
	
	JOI 왕국에는 $1$번부터 $N$번까지 번호가 붙어 있는 $N$ 개의 마을이 있다. 이 마을 사이를 연결하는 $N-1$ 개의 도로가 있다. $i$ 번째 ($1 \le i \le N-1$) 도로는 $A_i$번 마을과 $B_i$번 마을 사이를 연결한다. 모든 도로는 양방향으로 지날 수 있다. 어떤 마을에서도 다른 모든 마을 까지 몇 개의 도로를 거치면 갈 수 있다.
	
	현재, JOI 왕국은 $1$번부터 $K$번까지 번호가 붙어있는 $K$ 개의 도시로 나뉘어 있다. $j$번 마을은 $C_j$번 도시에 속한다. 모든 도시에는 적어도 하나의 마을이 속해 있다.
	
	K 이사장은 JOI 왕국의 왕이다. 그는 정확히 하나의 도시를 \textbf{수도}로 고를 것이다. 안보 문제로, 수도는 다음과 같은 조건을 만족해야 한다.
	
	\begin{itemize}
		\item[] 해당 도시에 있는 어떤 두 마을 사이에도, 그 도시에 포함된 마을만 거쳐서 오가는 것이 가능하다.
	\end{itemize}

	하지만, K 이사장은 현재 도시가 수도를 고를 수 없는 구조 일 수 있다는 생각을 했다.
	
	이 문제를 해결하기 위해서 K 이사장은 도시를 합칠 것이다. 정확히는 다음 조작을 할 수 있다.
	
	\begin{itemize}
		\item[] $1 \le x \le K$, $1 \le y \le K$와 $x \ne y$를 만족하는 $x$, $y$를 골라서 $y$번 도시에 속하는 모든 마을을 $x$번 도시에 속하도록 변경한다.
	\end{itemize}

	도시를 합치는 데에는 매우 큰 비용이 들기 때문에, K 이사장은 도시를 합치는 횟수를 최소화 하여 수도를 고르고 싶다.
	
	JOI왕국의 마을과 도로의 구조, 도시와 어떤 마을이 어떤 도시에 속해 있는지가 주어질 때, 도시를 합치는 횟수의 최솟값을 구하여라.
	
	\InputFile
	
	표준 입력에서 다음과 같은 형식으로 주어진다. 모든 값은 정수이다.

	$N$ $K$
	
	$A_1$ $B_1$
	
	$\vdots$
	
	$A_{N-1}$ $B_{N-1}$
	
	$C_1$
	
	$\vdots$
	
	$C_N$
	
	\OutputFile
	
	표준 출력에 하나의 줄을 출력하여라. 수도를 고르기 위한 도시를 합치는 횟수의 최솟값을 출력하여라.
	
	\Constraints
	
	\begin{itemize}
	\item $1 \le N \le 200\ 000$.
	\item $1 \le K \le N$.
	\item $1 \le A_i \le N$ ($1 \le i \le N-1$).
	\item $1 \le B_i \le N$ ($1 \le i \le N-1$).
	\item 어떤 마을에서도 다른 모든 마을 까지 몇 개의 도로를 거치면 갈 수 있다.
	\item $1 \le C_j \le K$ ($1 \le j \le N$).
	\item 모든 $k$에 ($1\le k \le K$) 대해, $C_j=k$를 만족하는 정수 $j$가 ($1\le j \le N$) 존재한다.
	\end{itemize}
	
	
	\SubtaskWithCost{1}{1}
	\begin{itemize}
		\item $N \le 20$
	\end{itemize}

	\SubtaskWithCost{2}{10}
	\begin{itemize}
		\item $N \le 2\ 000$
	\end{itemize}

	
	\SubtaskWithCost{3}{30}
	
	\begin{itemize}
		\item 모든 마을은 도로로 직접 연결된 마을이 최대 2개이다.
	\end{itemize}

	\SubtaskWithCost{4}{59}
	
	추가 제한조건이 없다.
	
	\Examples
		
	\begin{example}
	\exmp{
	6 3
	2 1
	3 5
	6 2
	3 4
	2 3
	1
	3
	1
	2
	3
	2
		}{%
	AABABB
		}%
	\end{example}

	이 예제에서, $(x, y) = (1, 3)$을 골라 3번 도시를 1번 도시에 합친다. 이 때, 1번 도시를 수도로 고를 수 있다. 처음에 어떤 도시도 수도로 고를 수 없다. 그렇기 때문에 도시를 합치는 횟수의 최솟값은 1이다.
	
	이 예제는 서브태스크 1, 2와 4의 조건을 만족한다.


	\begin{example}
	\exmp{
		8 4
		4 1
		1 3
		3 6
		6 7
		7 2
		2 5
		5 8
		2
		4
		3
		1
		1
		2
		3
		4
	}{%
		1
	}%
	\end{example}

	이 예제는 서브태스크 1, 2, 3과 4의 조건을 만족한다.
	

	
\end{problem}

