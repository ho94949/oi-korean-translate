\begin{problem}{건물 4}
	{standard input}{standard output}
	{2 seconds}{512 megabytes}{}
	
	JOI왕국에서 올림픽게임이 곧 열릴 것이다. 전 세계에 있는 참가자들을 환영하기 위해서 공항부터 숙박시설까지 길에 있는 건물들을 장식할 것이다. 총 $2N$ 개의 건물이 있고, $1$번 부터 $2N$번 까지 번호가 붙어있다.
	
	$K$ 이사장은 건물 장식 계획을 맡았다. 그는 장식 계획의 공모를 받았고, 계획들을 관찰한 결과 계획 A와 계획 B의 두 계획을 골랐다. 계획 A에서는 $i$번 ($1 \le i \le 2N$) 건물의 아름다움이 $A_i$ 이다. 계획 B에서는 $i$번 ($1 \le i \le 2N$) 건물의 아름다움이 $B_i$이다.
	
	두 계획은 모두 좋기 때문에, 둘 중 하나만 고르는 것은 어렵다. 이사장은 각 건물마다 계획 A와 계획 B중 하나를 골라서 장식하기로 했다. 공평하게 장식하기 위해서 $N$ 개의 건물에는 계획 A가, 나머지 $N$ 개의 건물에는 계획 B를 고를 것이다. 게다가, 건물의 아름다움이 공항부터 숙박시설까지 가는 동안 올라가는 것이 참가자들에게 아름답게 보이기 때문에, $i$번 ($1 \le i \le 2N$) 건물의 아름다움을 $C_i$ 라고 하면 모든 $1 \le i \le 2N-1$을 만족하는 $i$에 대해 $C_i \le C_{i+1}$을 만족해야 한다.
	
	건물의 수와 각 계획에 대해 건물의 아름다움이 정해졌을 때, 조건을 만족하도록 건물을 장식하는 것이 가능한지, 가능하다면 방법 하나를 출력하여라.
	
	\InputFile
	
	표준 입력에서 다음과 같은 형식으로 주어진다. 모든 값은 정수이다.

	$N$
	
	$A_1$ $\cdots$ $A_{2N}$
	
	$B_1$ $\cdots$ $B_{2N}$
	
	\OutputFile
	
	조건을 만족하도록 건물을 장식하는 것이 불가능하다면, \texttt{-1}을 표준 출력으로 출력하여라.
	
	가능하다면, 건물을 작싱하는 길이 $2N$의 문자열 $S$를 출력하여라. $i$ 번째 ($1 \le i \le 2N$) 문자는 $i$번 빌딩을 계획 A로 장식한다면 \texttt{A}, 계획 B로 장식한다면 \texttt{B}이다. 답이 여럿 있을 경우, 아무거나 출력하여라.
	
	\Constraints
	
	\begin{itemize}
	\item $1 \le N \le 500\ 000$.
	\item $0 \le A_i \le 1\ 000\ 000\ 000$ ($1 \le i \le 2N$).
	\item $0 \le B_i \le 1\ 000\ 000\ 000$ ($1 \le i \le 2N$).
	\end{itemize}
	
	
	\SubtaskWithCost{1}{11}
	\begin{itemize}
		\item $N \le 2\ 000$
	\end{itemize}
	
	\SubtaskWithCost{2}{89}
	
	추가 제한조건이 없다.
	
	\Examples
		
	\begin{example}
	\exmp{
	3
	2 5 4 9 15 11
	6 7 6 8 12 14
		}{%
	AABABB
		}%
	\end{example}

각 건물에 대해 계획 A, A, B, A, B, B를 고른다. 이 경우 모든 계획은 세 번씩 골라졌다. 각 건물의 아름다움은 2, 5, 6, 9, 12, 14이므로, 조건이 만족되었다.

	\begin{example}
	\exmp{
		2
		1 4 10 20
		3 5 8 13
	}{%
		BBAA
	}%
	\end{example}

건물을 장식 하는 방법이 여럿 있을 경우, 아무거나 출력해도 좋다.	
	

\begin{example}
	\exmp{
		2
		3 4 5 6
		10 9 8 7
	}{%
		-1
	}%
\end{example}

조건을 만족하도록 장식하는 방법이 없기 때문에 \texttt{-1}을 출력하여라.

\begin{example}
	\exmp{
		6
		25 18 40 37 29 95 41 53 39 69 61 90
		14 18 22 28 18 30 32 32 63 58 71 78
	}{%
		BABBABAABABA
	}%
\end{example}

	
\end{problem}

