\begin{problem}{수확}
	{standard input}{standard output}
	{3 seconds}{512 megabytes}{}
	
	IOI 농장은 사과를 기르는 거대한 농장이다. 이 농장은 큰 원 모양의 호수 둘레를 따라 있는 것으로 유명하다.
	
	IOI 농장에는 1번부터 $N$번까지 번호가 붙어있는 $N$명의 직원이 있다. 또한, IOI 농장에는 $M$개의 사과나무가 심겨 있고, 1번부터 $M$번까지 번호가 붙어 있다. 호수의 둘레는 $L$ 미터이다.
	
	처음, 시각 0초에 $i$번 ($1 \le i \le N$) 직원은 호수의 가장 북쪽에 있는 점으로부터 시계방향으로 $A_i$ 미터 떨어진 곳에서 기다리고 있다. $A_i$는 ($1 \le i \le N$) 서로 다르다. $j$번 ($1 \le j \le M$) 사과나무는 호수의 가장 북쪽에 있는 점으로부터 시계방향으로 $B_j$ 미터 떨어진 곳에 심겨 있다. $B_j$는 ($1 \le j \le M$) 모두 다르다. 게다가 어떤 사과나무도 직원이 있는 곳에 심겨 있지 않다.
	
	IOI 농장에 심어진 사과를 품종개량을 하기 위한 결과 하나의 나무에는 한 번에 하나의 사과만 열린다. 그리고 사과가 수확되었을 때 새로운 사과는 정확히 $C$초 이후에 생긴다. 시각 0초에 모든 사과나무에는 사과가 열려 있고 모든 직원이 시계방향으로 걷기 시작한다. 모든 직원의 속도는 초당 1미터이다. 만약 직원이 사과나무에 도착했을 때 사과나무에 사과가 열려 있으면 직원은 그 사과를 항상 수확할 것이다(만약 사과가 열리는 시간과 동시에 직원이 사과나무에 도착한다면 직원은 그 사과도 항상 수확한다). 우리는 사과를 수확하는 데 걸리는 시간을 무시할 것이다.
	
	K 이사장은 IOI 농장의 주식 지분을 가지고 있다. 당신이 IOI 농장의 관리인이기 때문에 K 이사장은 직원의 효율성에 대해서 보고하기를 바란다. 정확히는 K 이사장은 다음 $Q$개의 값을 알고 싶다.
	
	\begin{itemize}
		\item[] 각 $k$에 ($1 \le k \le Q$) 대해, $V_k$번 직원이 시각 $T_k$ 초까지 수확한 사과의 개수 (시각 $T_k$에 수확한 사과가 존재한다면, 이도 포함한다.)
	\end{itemize}
	
	직원의 수, 사과나무의 수, 호수의 둘레, 새로운 사과가 열리는 데 필요한 시간, 직원의 위치, 사과나무의 위치와 $Q$개의 질문에 대한 정보가 주어졌을 때 각 질문에 대해 수확한 사과의 개수를 출력하여라.
	
	
	\InputFile
	
	표준 입력에서 다음과 같은 형식으로 주어진다. 모든 수는 정수이다.
	
	$N$ $M$ $L$ $C$
	
	$A_1$ $\cdots$ $A_N$
	
	$B_1$ $\cdots$ $B_M$
	
	$Q$
	
	$V_1$ $T_1$
	
	$\cdots$

	$V_Q$ $T_Q$
	
	
	\OutputFile
	
	표준 출력에 $Q$개의 줄을 출력하여라. $k$ 번째 ($1 \le k \le Q$) 줄은, $k$ 번째 질문에 대한 답이다.
	

	\Constraints


	\begin{itemize}
		\item $1 \le N \le 200\ 000$.
		\item $1 \le M \le 200\ 000$.
		\item $N+M \le L \le 1\ 000\ 000\ 000$.
		\item $1 \le C \le 1\ 000\ 000\ 000$.
		\item $0 \le A_i < L$ ($1 \le i \le N$).
		\item $A_i < A_{i+1}$ ($1 \le i \le N-1$).
		\item $0 \le B_j < L$ ($1 \le j \le M$).
		\item $B_j < B_{j+1}$ ($1 \le j \le M-1$).
		\item $A_i \ne B_j$ ($1 \le i \le N$, $1 \le j \le M$).
		\item $1 \le Q \le 200\ 000$.
		\item $1 \le V_k \le N$ ($1 \le k \le Q$).
		\item $1 \le T_k \le 1\ 000\ 000\ 000\ 000\ 000\ 000 = 10^{18}$ ($1 \le k \le Q$).
	\end{itemize}


	\SubtaskWithCost{1}{5}
	\begin{itemize}
		\item $N \le 3\ 000$.
		\item $M \le 3\ 000$.
		\item $Q \le 3\ 000$.
	\end{itemize}

	
	\SubtaskWithCost{2}{20}
	\begin{itemize}
		\item $T_k \ge 1\ 000\ 000\ 000\ 000\ 000 = 10^{15}$ ($1 \le k \le Q$).
	\end{itemize}
	
	\SubtaskWithCost{3}{75}
	
	추가 제한조건이 없다.
		
	\Examples
		
	\begin{example}
	\exmp{
3 2 7 3
1 4 6
0 5
3
1 7
2 3
3 8
	}{%
2
1
1
	}%
\end{example}

	\begin{itemize}
		\item 시각 1초에, 2번 직원은 2번 사과나무에서, 3번 직원은 1번 사과나무에서 사과를 수확한다.
		\item 시각 3초에, 2번 직원은 1번 사과나무에 도착한다. 그때 사과가 열리지 않았으므로, 직원은 사과를 수확하지 않는다.
		\item 시각 4초에, 1번 직원은 2번 사과나무에서 사과를 수확한다.
		\item 시각 6초에, 1번 직원은 1번 사과나무에서 사과를 수확한다. 3번 직원은 2번 사과나무에 도착하지만, 그때 사과가 열리지 않았으므로, 직원은 사과를 수확하지 않는다.
		\item 시각 8초에, 2번 직원은 2번 사과나무에서 사과를 수확한다. 3번 직원은 1번 사과나무에 도착하지만, 그때 사과가 열리지 않았으므로, 직원은 사과를 수확하지 않는다.
	\end{itemize}
	
	시각 7초까지 1번 직원이 수확한 사과의 개수가 2개이므로, 첫째 줄에는 2를 출력한다.

	\begin{example}
	\exmp{
		5 3 20 6
		0 4 8 12 16
		2 11 14
		9
		4 1932
		2 93787
		1 89
		5 98124798
		1 2684
		1 137598
		3 2
		3 8375
		4 237
	}{%
		146
		7035
		7
		7359360
		202
		10320
		0
		628
		18
	}%
	\exmp{
		8 15 217 33608
		0 12 71 96 111 128 152 206
		4 34 42 67 76 81 85 104 110 117 122 148 166 170 212
		14
		2 223544052420046341
		3 86357593875941375
		4 892813012303440034
		1 517156961659770735
		7 415536186438473633
		6 322175014520330760
		7 557706040951533058
		6 640041274241532527
		5 286263974600593111
		8 349405886653104871
		1 987277313830536091
		5 989137777159975413
		2 50689028127994215
		7 445686748471896881
	}{%
		33230868503053
		3
		5
		1
		123542793648997
		8
		165811220737767
		8
		7
		1
		1
		7
		7535161012043
		132506837660717
	}%
	\end{example}

	
\end{problem}

