\begin{problem}{초대}
	{standard input}{standard output}
	{3초}{128MB}{}
	
	20XX년, 드디어 IOI가 JOI 나라의 JOI 마을에서 열리게 되어, 이를 기념하기 위한 파티가 열린다. JOI 마을에는 $A$ 마리의 개($1, 2, \cdots, A$의 번호가 붙어있다.)와,  $B$ 마리의 고양이($1, 2, \cdots, B$의 번호가 붙어있다.)가 있다. 당신은 이 $A+B$ 마리 모두를 파티에 초대하고 싶다.
	
	개와 고양이 사이에는 $N$ 개의 \textbf{사이좋음 그룹}이 있다. $i$ 번째의 사이좋음 그룹은 번호가 $P_i$ 이상 $Q_i$ 이하인 개 $Q_i - P_i+1$ 마리와 번호가 $R_i$ 이상 $S_i$ 이하인 고양이 $S_i- R_i+1$ 마리로 구성되어있다. 또한, 각 사이좋음 그룹에는 \textbf{사이좋음 정도}라는 정수가 정해져 있다. $i$ 번째 사이좋음 그룹의 사이좋음 정도는 $T_i$이다. 한 마리의 개나 한 마리의 고양이가 여러 사이좋음 그룹에 들어 있을 수도 있고, 어떤 사이좋음 그룹에도 들어 있지 않은 개나 고양이가 있을 수도 있다.
	
	당신은 번호가 $C$ 인 개와 매우 사이가 좋아, 그 개를 초대하는 데에 성공했다. 당신은 다음과 같은 행동을 반복해서 남은 개나 고양이를 초대하려 한다.
	
	\begin{itemize}
		\item $A+B$ 마리 모두를 이미 초대한 경우에는 종료한다.
		\item 아직 초대하지 않은 개나 고양이 각각에 대해, 초대할 때 \textbf{행복도}를 구한다. 행복도는 그 개나 고양이가 들어 있는 사이좋음 그룹 중에서, 이미 초대에 성공한 개나 고양이가 한 마리 이상 들어 있는 사이좋음 그룹의 사이좋음 정도의 최댓값이다. 그런 사이좋음 그룹이 존재하지 않은 경우에 행복도는 0이다.
		\item 행복도가 최대인 개나 고양이를 고른다. 그런 개나 고양이가 여럿 있으면 개를 우선시하고, 그래도 여럿 있으면 번호가 작은 동물을 고른다.
		\item 선택한 개나 고양이의 행복도가 0인 경우 초대는 실패한다. 그렇지 않은 경우, 선택된 개나 고양이를 초대하는 데에 성공한다.
	\end{itemize}

	당신은, 이 초대 방법이 어떤 결과가 될 지 미리 계산하고 싶어졌다.
	
	개의 수 $A$, 고양이의 수 $B$, 당신과 매우 사이가 좋은 개의 번호 $C$와 $N$ 개의 사이좋음 그룹의 정보가 주어졌을 때, $A+B$ 마리 모두를 초대하는데 성공하는지 판단하고, 성공한 경우에는 각 단계에서 초대한 개나 고양이의 행복도 총합이 얼마인지 구하는 프로그램을 작성하여라.
	
	\Constraints
	
	
	\begin{tabular}{ll}
		$1 \le A \le 1\ 000\ 000\ 000$ & 개의 수 \\
		$1 \le B \le 1\ 000\ 000\ 000$ & 고양이의 수 \\
		$1 \le N \le 100\ 000$ & 사이좋음 그룹의 수 \\
		$1 \le T_i \le 1\ 000\ 000\ 000$ 사이좋음 정도
	\end{tabular}
	
	
	\InputFile
	
	다음 정보가 표준 입력으로 주어진다.
	
	\begin{itemize}
		\item 첫째 줄에는 정수 $A$, $B$, $C$ ($1 \le C \le A$) 가 공백으로 구분되어 주어지고, 각각 개의 수, 고양이의 수, 당신과 매우 사이가 좋은 개의 번호를 의미한다.
		\item 둘째 줄에는 정수 $N$이 주어지고 사이좋음 그룹의 수를 의미한다.
		\item $2+i$ 번째 ($1 \le i \le N$) 줄에는 정수 $P_i$, $Q_i$, $R_i$, $S_i$, $T_i$ ($1 \le P_i \le Q_i \le A$, $1 \le R_i \le S_i \le B$) 가 공백으로 구분되어 주어지고, $i$ 번째 사이좋음 그룹은 번호가 $P_i$ 이상 $Q_i$ 이하인 개와 번호가 $R_i$ 이상 $S_i$ 이하인 고양이로 구성되어있다는 것을 의미한다.
	\end{itemize}
	
	
	
	\OutputFile
	
	표준 출력으로, 다음 정수 중 하나를 첫째 줄에 출력하여라.
	
	\begin{itemize}
		\item $A+B$ 마리 모두를 초대하는 데에 성공한 경우, 각 단계에서 선택된 개나 고양이의 행복도 총합을 의미하는 정수.
		\item 초대가 도중에 실패했을 경우, 정수 $-1$.
	\end{itemize}
	
	\Scoring
	
	채점 데이터 중, 배점의 30\%에 대해 $A \le 1\ 000$, $B \le 1\ 000$, $N \le 2\ 000$을 만족한다.
	
	채점 데이터 중, 배점의 50\%에 대해 $N \le 2\ 000$을 만족한다.
	
	\Examples
	
	\begin{example}
		\exmp{
			5 6 3
			4
			2 4 1 3 20
			1 2 2 4 40
			4 5 2 3 30
			4 4 4 6 10
		}{%
			280
		}%
	\end{example}
	
	이 예에서, 개나 고양이는 다음과 같이 초대된다.
	
	\begin{itemize}
		\item 당신은 3번 개를 초대했다.
		\item 행복도를 계산해 보면, 2번 개: 20, 4번 개: 20, 1번 고양이: 20, 2번 고양이: 20, 3번 고양이: 20, 다른 초대를 받지 않은 개나 고양이: 0이다. 당신은 2번 개를 골라서, 초대에 성공한다.
		\item 행복도를 계산해 보면, 1번 개: 40, 4번 개: 20, 1번 고양이: 20, 2번 고양이: 40, 3번 고양이: 40, 4번 고양이: 40, 다른 초대를 받지 않은 개나 고양이: 0이다. 당신은 1번 개를 골라서, 초대에 성공한다.
		\item 위와 같이 초대가 계속되어, 다음의 표에 나타난 차례대로 모든 개나 고양이를 초대한다.
	\end{itemize}

	\begin{center}
	\begin{tabular}{|c|c|c|}
	\hline
	동물 & 번호 & 행복도 \\ \hline
	개 & 3 & --- \\ \hline
	개 & 2 & 20 \\ \hline
	개 & 1 & 40 \\ \hline
	고양이 & 2 & 40 \\ \hline
	고양이 & 3 & 40 \\ \hline
	고양이 & 4 & 40 \\ \hline
	개 & 4 & 30 \\ \hline
	개 & 5 & 30 \\ \hline
	고양이 & 1 & 20 \\ \hline
	고양이 & 5 & 10 \\ \hline
	고양이 & 6 & 10 \\ \hline
	\end{tabular}
	\end{center}
	
	표의 ``행복도" 열의 값은, 해당 개나 고양이를 초대 할 때 행복도를 의미한다. 총합인 280을 출력한다.
	
	\begin{example}
		\exmp{
			10 10 1
			2
			1 5 1 5 3
			6 10 6 10 4
		}{%
			-1
		}%
	\end{example}
	
	이 예에서는 1번 개, 2번 개, 3번 개, 4번 4개, 5번 개, 1번 고양이, 2번 고양이, 3번 고양이, 4번 고양이, 5번 고양이 10마리를 초대한 후 선택된 6번 개의 행복도가 0이므로, 초대가 중간에 실패한다.
	
	
\end{problem}

