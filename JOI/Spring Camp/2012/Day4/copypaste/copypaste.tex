\begin{problem}{복사 붙여넣기}
	{standard input}{standard output}
	{17초}{512MB}{}
	
	텍스트 에디터의 제일 중요한 기능 중 하나로 복사 붙여넣기가 있다. JOI 회사는 복사 붙여넣기를 매우 고속으로 처리할 수 있는 텍스트 에디터의 개발을 시작했다. JOI 회사에 소속된 우수한 프로그래머인 당신은, 중요한 복사 붙여넣기 기능의 실제 구현을 담당하게 되었다. JOI 회사의 운명이 걸린 일이므로 정확하고 빠른 프로그램을 작성하고 싶다.
	
	구체적인 사양은 다음과 같다. 우선, 파일의 내용을 문자열 $S$ 라 하자. 이제 복사 붙여넣기가 $N$ 번 일어난다. $i$ 번째의 조작은 위치 $A_i$부터 위치 $B_i$까지의 문자열을 복사해서, 복사한 문자열을 원래 문자열의 위치 $C_i$에 삽입한다. 여기서, 위치 $x$는 앞에 부터 세어 $x$ 개의 문자 직후를 의미한다(위치 0은 문자열의 가장 앞부분이다). 단, 조작 후의 문자열 길이가 $M$을 넘을 경우, 길이가 $M$이 되도록 오른쪽부터 차례로 문자를 삭제한다. 
	
	문자열 길이 상한 $M$, 처음 문자열 $S$, 조작 횟수 $N$과 $N$ 번의 복사 붙여넣기 조작이 주어졌을 때, 조작 후의 문자열을 출력하는 프로그램을 작성하여라.
	
	\Constraints
	
	
	\begin{tabular}{ll}
		$1 \le M \le 1\ 000\ 000$ & 문자열의 길이 상한 \\
		$1 \le N \le 1\ 000\ 000$ & 조작 횟수 \\
	\end{tabular}
	
	
	\InputFile
	
	다음 정보가 표준 입력으로 주어진다.
	
	\begin{itemize}
		\item 첫째 줄에는 정수 $M$이 주어지며, 문자열 길이 상한을 의미한다.
		\item 둘째 줄에는 문자열 $S$가 주어지며, 처음 문자열을 의미한다. $S$는 알파벳 소문자로 되어있고, 길이는 1 이상 $M$ 이하이다.
		\item 셋째 줄에는 정수 $N$이 주어지며, 조작 횟수를 의미한다.
		\item $3+i$ 번째 ($1 \le i \le N$) 줄에는 정수 $A_i$, $B_i$, $C_i$가 공백으로 구분되어 주어지며, $i$ 번째 조작은  위치 $A_i$부터 위치 $B_i$까지의 문자열을 복사해서, 복사한 문자열을 원래 문자열의 위치 $C_i$에 삽입한다는 것을 의미한다. $i$ 번째 문자열의 조작 직전의 문자 길이를 $L_i$ 라고 하면, $0 \le A_i < B_i \le L_i$ 와 $0 \le C_i \le L_i$ 를 만족한다.
	\end{itemize}
	
	
	\OutputFile
	
	표준 출력으로, $N$ 번의 조작 이후 문자열을 첫째 줄에 출력한다.
	
	\Scoring
	
	채점 데이터 중, 배점의 10\%에 대해 $M \le 100\ 000, N \le 100\ 000$을 만족한다.
	
	
	\Examples
	
	\begin{example}
		\exmp{
			18
			copypaste
			4
			3 6 8
			1 5 2
			4 12 1
			17 18 0
		}{%
			acyppypastoopyppyp
		}%
	\end{example}
	
	\begin{itemize}
		\item 처음 문자열은 \texttt{copypaste}이다.
		\item 첫 번째 조작에서, 위치 3부터 위치 6까지의 문자열 \texttt{ypa}를 복사해서, 위치 8에 삽입해서, 문자열은 \texttt{copypastypae}가 된다.
		\item 두 번째 조작에서, 위치 1부터 위치 5까지의 문자열 \texttt{opyp}를 복사해서, 위치 2에 삽입해서, 문자열은 \texttt{coopyppypastypae}가 된다.
		\item 세 번째 조작에서, 위치 4부터 위치 12까지의 문자열 \texttt{yppypast}를 복사해서, 위치 1에 삽입해서, 문자열은 \texttt{cyppypastoopyppypastypae}가 되지만, 길이가 $M=18$을 넘으므로, 오른쪽에서 부터 문자열을 삭제해서, 문자열 \texttt{cyppypastoopyppypa}가 된다.
		\item 네 번째 조작에서, 위치 17부터 위치 18까지의 문자열 \texttt{a}를 복사해서, 위치 1에 삽입해서, 문자열은 \texttt{acyppypastoopyppypa}가 되지만, 길이가 $M=18$을 넘으므로, 오른쪽에서 부터 문자열을 삭제해서, 문자열 \texttt{acyppypastoopyppyp}가 된다.
	\end{itemize}

	
	\begin{example}
		\exmp{
			100
			joi
			3
			0 1 0
			3 4 3
			2 3 3
		}{%
			jjooii
		}%
	\end{example}


\end{problem}

