\begin{problem}{캥거루}
	{standard input}{standard output}
	{2초}{64MB}{}
	
	K 이사장은 캥거루에 흥미를 느껴서, 캥거루의 행동을 관찰하기로 했다. K 이사장은 N 마리의 캥거루를 관찰하고 있다. 캥거루에는 주머니가 하나씩 달려있다. 캥거루에는 번호 $1, 2, \cdots, N$이 붙어있다. $i$ 번 캥거루의 몸 크기는 $A_i$ 이며, $i$ 번 캥거루의 주머니 크기는 $B_i$이다. 주머니 크기는 몸 크기보다 작다. ($A_i > B_i$)
	
	처음에는 어떤 캥거루의 주머니 안에도 캥거루가 들어있지 않다. 캥거루는 다음의 조작을 \textbf{조작을 더 할 수 없을 때까지} 반복한다.
	
	\begin{itemize}
	\item[]  $A_i < B_j$를 만족하는 $i$ 번 캥거루와 $j$ 번 캥거루가 있어서, $i$ 번 캥거루가 다른 캥거루의 주머니에 들어있지 않고, $j$ 번 캥거루의 주머니 안에 어떤 캥거루도 들어있지 않는 두 캥거루가 존재할 때, $i$ 번 캥거루는 $j$ 번 캥거루 안에 들어간다. 이때, $i$ 번 캥거루에 주머니 안에 있는 다른 캥거루가 있어도, $j$ 번 캥거루가 다른 캥거루의 주머니 안에 있어도 상관없다. 이런 $(i, j)$ 쌍이 여러 개 있으면, 어떤 쌍이 선택될지는 알 수 없다. $i$ 번 캥거루 안에 다른 캥거루가 들어있을 경우, 안에 들어 있는 캥거루는 $i$ 번 캥거루와 같이 이동한다.
	\end{itemize}
	
	캥거루의 몸 크기와 주머니 크기가 주어졌을 때, 최종 상태로 가능한 경우가 몇 가지인지를 $1\ 000\ 000\ 007 (=10^9+7)$로 나눈 나머지를 구하여라.
	
	\Constraints
	
	
	\begin{tabular}{ll}
		$1 \le N \le 300$ & 캥거루의 수 \\
		$1 \le B_i < A_i \le 1\ 000\ 000\ 000$ & 	$i$ 번째 캥거루의 주머니 크기와 몸 크기 \\
	\end{tabular}
	
	
	\InputFile
	
	다음 정보가 표준 입력으로 주어진다.
	
	\begin{itemize}
		\item 첫째 줄에는 정수 $N$이 주어진다. $N$은 캥거루의 수를 의미한다.
		\item 다음 $N$ 개의 줄에는 캥거루의 정보가 주어진다. $i+1$ 번째 ($1 \le i \le N$) 줄에는 두 개의 정수 $A_i$, $B_i$가 공백으로 구분되어 주어진다. $A_i$는 $i$ 번 캥거루의 몸 크기를, $B_i$는 $i$ 번 캥거루의 주머니 크기를 각각 의미한다.
	\end{itemize}
	
	
	\OutputFile
	
	표준 출력으로, 최종 상태로 가능한 경우가 몇 가지인지를 $1\ 000\ 000\ 007 (=10^9+7)$로 나눈 나머지를 첫째 줄에 출력하여라.
	
	\Scoring
	
	채점 데이터 중, 배점의 50\%에 대해 $N \le 30$을 만족한다.
	
	채점 데이터 중, 배점의 70\%에 대해 $N \le 70$을 만족한다.
	
	
	\Examples
	
	\begin{example}
		\exmp{
			5
			4 3
			3 1
			6 5
			2 1
			4 2
		}{%
			4
		}%
	\end{example}
	
	1, 2, 5번 캥거루는 3번 캥거루의 주머니 안에 들어갈 수 있다. 또한, 4번 캥거루는 1번 캥거루 혹은 3번 캥거루의 들어갈 수가 있고, 3번 캥거루는 다른 어떤 캥거루의 주머니에도 들어갈 수 없다. 그러므로, 최종 상태로 가능한 것은 다음의 네 종류이다.
	
	\begin{itemize}
		\item 4번 캥거루가 3번 캥거루의 주머니 안에 들어있다.
		\item 4번 캥거루는 1번 캥거루의 주머니 안에, 1번 캥거루는 3번 캥거루의 주머니 안에 들어있다.
		\item 4번 캥거루는 1번 캥거루의 주머니 안에, 2번 캥거루는 3번 캥거루의 주머니 안에 들어있다.
		\item 4번 캥거루는 1번 캥거루의 주머니 안에, 5번 캥거루는 3번 캥거루의 주머니 안에 들어있다.
	\end{itemize}
	
	
	
	
	\begin{example}
		\exmp{
			20
			7 6
			7 3
			10 1
			7 2
			10 7
			10 7
			8 6
			3 2
			5 4
			7 2
			3 2
			10 9
			9 4
			7 2
			8 6
			5 4
			8 6
			7 4
			10 5
			9 3
		}{%
			21060
		}%
	\end{example}


\end{problem}

