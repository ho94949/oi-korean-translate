\begin{problem}{점치기}
	{standard input}{standard output}
	{2초}{64MB}{}
	
	K 이사장은 점을 매우 좋아해서, 언제나 이런저런 점을 치고 있다. 오늘은 카드를 사용해서 올해 IOI에서 일본선수단의 미래를 점치기로 했다.
	
	점을 치는 방법은 다음과 같다.
	
	\begin{itemize}
		\item 처음 카드를 세로로 $M$ 행, 가로 $N$ 열의 직사각형 모양으로 모두 앞면이 보이도록 배치한다.
		\item $i = 1, \cdots, K$ 에 대해, ``위에서 $A_i$ 번째 행부터 $B_i$ 번째 행까지, 왼쪽에서 $C_i$ 번째 행부터 $D_i$ 번째 열까지에 있는 모든 카드의 앞뒷면을 뒤집는다."라는 조작을 한다. 즉, 위에서부터 $a$ 번째 행, 왼쪽에서부터 $b$ 번째 열에 있는 카드를 $(a, b)$번 카드라고 하면, 각 $i$에 대해 $A_i \le a \le B_i$ 이면서 $C_i \le b \le D_i$를 만족하는 $(a, b)$번 카드를 모두 뒤집는다.
		\item 조작이 끝난 경우, 뒷면과 앞면이 되어있는 카드가 몇 장인지에 따라 점의 결과가 결정된다.
	\end{itemize}

	K 이사장은 카드를 뒤집는 횟수가 생각보다 많다는 것을 깨달았기 때문에, 카드를 실제로 사용해서 점을 두는 대신 조작이 끝났을 때 앞면인 카드가 몇 장인지만 구하기로 했다.

	행의 수 $M$, 열의 수 $N$, 조작의 횟수 $K$와 $K$ 번의 조작이 주어졌을 때, 조작 후에 앞면인 카드의 수를 구하는 프로그램을 작성하여라.
	
	
	\Constraints
	

\begin{tabular}{ll}
	$1 \le M \le 1\ 000\ 000\ 000 (=10^9)$ & 행의 수 \\
	$1 \le N \le 1\ 000\ 000\ 000 (=10^9)$ & 열의 수 \\
	$1 \le K \le 100\ 000$ & 조작의 회수
\end{tabular}


	\InputFile
	
	다음 정보가 표준 입력으로 주어진다.
	
	\begin{itemize}
		\item 첫째 줄에는 정수 $M$, $N$, $K$가 공백으로 구분되어 주어지며, 카드가 $M$ 행 $N$ 열로 나열되어 있고, 조작을 $K$ 번 한다는 것을 의미한다.
		\item $1+i$ 번째 행 ($1 \le i \le K$) 에는 네 개의 정수 $A_i$, $B_i$, $C_i$, $D_i$ ($1 \le A_i \le B_i \le M$, $1 \le C_i \le D_i \le N$)가 주어지며, $i$ 번째 조작이 위에서 $A_i$ 번째 행부터 $B_i$ 번째 행까지, 왼쪽에서 $C_i$ 번째 행부터 $D_i$ 번째 열까지에 있는 모든 카드의 앞뒷면을 뒤집는다는 것을 의미한다.
	\end{itemize}


	
	\OutputFile
	
	표준 출력으로, $K$ 번의 조작 이후에 앞면이 되어 있는 카드의 장수를 첫째 줄에 출력하여라.
	
	\Scoring
	
	채점 데이터 중, 배점의 30\%에 대해 $K \le 3\ 000$을 만족한다.
	
	\Examples
		
	\begin{example}
	\exmp{
6 5 3
2 4 1 4
4 6 3 5
1 2 3 5
	}{%
11
	}%
	\end{example}
	 
	이 예에서, $K=3$번의 조작은 다음과 같이 진행된다.
	
	앞면인 카드를 □, 뒷면인 카드를 ■라고 나타내면
	
	\begin{center}
	초기상태
	
	□□□□□
	
	□□□□□
	
	□□□□□
	
	□□□□□
	
	□□□□□
	
	□□□□□
	
	↓
	
	□□□□□
	
	■■■■□
	
	■■■■□
	
	■■■■□
	
	□□□□□
	
	□□□□□
	
	↓
	
	□□□□□
	
	■■■■□
	
	■■■■□
	
	■■□□■
	
	□□■■■
	
	□□■■■
	
	↓
	
	□□■■■
	
	■■□□■
	
	■■■■□
	
	■■□□■
	
	□□■■■
	
	□□■■■
	
	최종상태

\end{center}

최종상태에서 앞면인 카드의 수인 11을 출력한다.
	
\end{problem}

