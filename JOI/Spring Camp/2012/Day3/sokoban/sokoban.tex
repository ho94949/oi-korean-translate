\begin{problem}{소코반}
	{standard input}{standard output}
	{2초}{512MB}{}
	
	``소코반"은 오랫동안 사람들에게 사랑받아온 퍼즐이다.
	``소코반"은 세로 $M$ 행, 가로 $N$ 열의 격자에서 진행되는 게임이다. 격자 위에 있는 상자를 플레이어를 조작해서 목적지까지 옮기는 것이 목적이다. 이 문제에서는 상자가 한 개 있는 경우만 생각한다. 격자의 일부 격자 칸에는 벽이 있어서 플레이어, 상자, 목적지는 각각 벽이 아닌 한 격자 칸에 있다(플레이어나 상자를 격자 칸 밖으로 움직일 수는 없다). 플레이어는 다음 중 하나의 조작을 할 수 있다.
	
	\begin{itemize}
		\item 플레이어가 있는 격자 칸에 인접한 격자 칸 중 벽이 아닌 상자가 없는 격자를 한 칸 골라서, 플레이어가 그 칸으로 이동한다.
		
		\item 플레이어가 있는 격자 칸과 상자가 있는 격자 칸이 인접하면서, 플레이어의 반대쪽으로 상자와 인접한 벽이 아닌 격자 칸이 있으면, 상자는 해당 격자 칸으로 이동하며, 플레이어는 상자가 있는 격자 칸으로 이동한다.
	\end{itemize}

	여기서 격자와 격자가 인접한다는 것은 두 격자가 한 변을 공유한다는 것을 의미한다.
	
	아래는 ``소코반" 문제의 예이다. \texttt{\#}은 벽, \texttt{@}은 플레이어, \texttt{O}는 상자, \texttt{X}는 목표지점, \texttt{.}은 그 외의 격자 칸을 의미한다.
	
	\begin{center}
		
	\texttt{..\#@.}
	
	\texttt{.X.O.}
	
	\texttt{\#\#..\#}

	\end{center}
	
	이 상태에서, 다음의 조작을 통해 상자를 목적지에 놓을 수 있다.
	
	\begin{enumerate}
		\item 플레이어를 오른쪽으로 움직인다.
		\item 플레이어를 아래쪽으로 움직인다.
		\item 상자와 플레이어를 왼쪽으로 움직인다.
		\item 상자와 플레이어를 왼쪽으로 움직인다.
	\end{enumerate}
	
	
	한편, 다음 상태라면 상자를 목적지점까지 움직일 수 없다.
	
	\begin{center}
	\texttt{..\#..}
	
	\texttt{.X.O.}
	
	\texttt{\#\#.@\#}
	\end{center}

	
	당신은 격자의 벽의 위치와 목적지 지점이 정해져 있을 때, 플레이어와 상자를 배치해서 풀 수 있는 ``소코반" 문제가 몇 종류 있는지 알고 싶다. 여기서 풀 수 있는 ``소코반" 문제는, 조작을 몇 번 반복해서 목적지점으로 상자를 이동할 수 있는 초기상태를 의미한다. 단, \textbf{플레이어와 상자는 각각 벽이 아니면서 목적지점이 아닌 격자에 배치되어야 하며, 플레이어와 상자를 서로 다른 격자에 배치해야 한다.}
	
	격자의 크기와 벽의 위치와 목적지의 위치가 주어졌을 때, 풀 수 있는 ``소코반" 문제가 몇 종류 있는지 구하는 프로그램을 작성하여라.
	
	\Constraints
	
	
	\begin{tabular}{ll}
		$1 \le M \le 1\ 000$ & 격자의 세로 길이 \\
		$1 \le N \le 1\ 000$ & 격자의 가로 길이 \\
		
	\end{tabular}
	
	
	\InputFile
	
	다음 정보가 표준 입력으로 주어진다.
	
	\begin{itemize}
		\item 첫째 줄에는 정수 $M$, $N$이 공백으로 구분되어 주어지며, 각각은 격자의 세로와 가로 길이이다.
		\item 다음 $M$ 개의 줄에는 격자의 정보가 주어진다. 각 줄은 $N$ 개의 문자로 되어있다. 각 문자는 \texttt{\#} \texttt{X} \texttt{.} 중 하나이며, \texttt{\#}은 벽, \texttt{X}는 목적지, \texttt{.} 은 그 외의 격자 칸(플레이어나 상자의 초기위치로 가능하다.)을 의미한다. 문자 \texttt{X}는 정확히 한 번 주어진다.
		
	\end{itemize}
	
	
	
	\OutputFile
	
	표준 출력으로, 풀 수 있는 ``소코반" 문제가 몇 종류 있는지 의미하는 정수를 첫째 줄에 출력하여라.
	
	\Scoring
	
	채점 데이터 중, 배점의 20\%에 대해 $M \le 50, N \le 50$을 만족한다.
	
	\Examples
	
	\begin{example}
		\exmp{
			3 5
			..\#..
			.X...
			\#\#..\#
		}{%
			9
		}%
	\end{example}
	
	풀 수 있는 소코반 문제는 다음과 같이 9종류이다.
	
	\begin{verbatim}
	..#@.    ..#.@    ..#..    ..#..    ..#..    ..#..    ..#@.    ..#.@    ..#..
	.XO..    .XO..    .XO@.    .XO.@    .XO..    .XO..    .X.O.    .X.O.    .X.O@
	##..#    ##..#    ##..#    ##..#    ##@.#    ##.@#    ##..#    ##..#    ##..#
	\end{verbatim}
	
	
	\begin{example}
		\exmp{
			2 3
			.X.
			...
		}{%
			0
		}%
	\end{example}
	
	이 예에서는, 풀 수 있는 ``소코반" 문제를 만들 수 없다.
	
	\begin{example}
	\exmp{
		4 7
		.\#.\#.\#\#
		\#\#.\#..\#
		....X..
		\#\#.\#...
	}{%
		24
	}%
\end{example}
	
	
\end{problem}

