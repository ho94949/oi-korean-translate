\begin{problem}{빌딩 장식 2}
	{standard input}{standard output}
	{1초}{64MB}{}
	
	
	일본에는 $N$ 개의 거리가 있고, 그 사이를 $N-1$ 개의 양방향으로 통행할 수 있는 도로가 연결하고 있다. 어떤 거리에서 다른 어떤 거리까지도 몇 개의 도로를 거쳐서 갈 수 있다. 각 도로에는 빌딩이 하나 있고, $i$ 번째 도로 빌딩의 높이를 $H_i$라고 하자.
	
	국제정보올림피아드에 참가하기 위해 일본에 온 세계 선수단을 환영하기 위해 관광여행을 개최하기로 했다. 관광여행은 어느 거리에서 시작하고, 도로를 통해 다른 거리로 이동하는 것을 반복해 어떤 거리를 방문한 후 끝낸다. 이때, 같은 거리를 두 번 이상 방문하지는 않는다.
	
	세계 선수들을 더욱 환영하기 위해 관광에서 방문하는 몇몇 거리에 있는 빌딩을 장식하기로 했다. 어떤 저명한 디자이너에게 디자인을 의뢰했는데, 장식하는 빌딩의 높이는 출발하는 도시에서 도착하는 도시로 갈수록 높이가 높아질 필요가 있다고 했다. 즉, 장식하는 빌딩이, 방문하는 순서대로 $i_1$, $i_2$, $\cdots$, $i_k$ 번째 거리의 빌딩이라고 할 때, $H_{i_1} < H_{i_2} < \cdots < H_{i_k}$여야 한다(방문하는 모든 빌딩을 장식할 필요는 없음에 주의하여라).
	
	가능한 한 화려하게 꾸미기 위해서 최대한 많은 빌딩을 장식하고 싶다. 관광여행을 시작하는 거리, 끝내는 거리, 그리고 빌딩을 장식하는 거리를 마음대로 선택할 수 있다면, 장식할 수 있는 빌딩 수의 최댓값을 구하는 프로그램을 작성하여라.

	
	\Constraints
	

\begin{tabular}{ll}
	$2 \le N \le 100\ 000$ & 거리의 수 \\
	$1 \le H_i \le 1\ 000\ 000\ 000$ & 	$i$ 번째 거리에 있는 빌딩의 높이\\
\end{tabular}


	\InputFile
	
	다음 정보가 표준 입력으로 주어진다.
	
	\begin{itemize}
		\item 첫째 줄에는 정수 $N$이 주어지고, 거리의 수를 의미한다.
		\item 다음 $N$ 개의 줄에는 각 거리에 있는 빌딩의 높이에 관련된 정보가 주어진다. $i$ 번째 줄에는 정수 $H_i$가 주어진다. 이는 $i$ 번째 거리에 있는 빌딩의 높이가 $H_i$라는 것을 의미한다.
		\item 다음 $N-1$개의 줄에는 도로에 관련된 정보가 주어진다. $i$ 번째 줄에는 정수 $A_i$, $B_i$ ($1 \le A_i < B_i \le N$) 이 공백으로 구분되어 주어진다. 이는 $i$ 번째 도로가 $A_i$ 번 거리와 $B_i$ 번 거리를 잇는다는 것을 의미한다.
	\end{itemize}

	
	\OutputFile
	
	표준 출력으로, 장식 할 수 있는 빌딩 수의 최댓값을 나타내는 정수를 출력하여라.
	
	\Scoring
	
	채점 데이터 중, 배점의 10\%에 대해 $N \le 100$을 만족한다.
	
	채점 데이터 중, 배점의 30\%에 대해 $N \le 2000$을 만족한다.
	
	\Examples
		
	\begin{example}
	\exmp{
7
4
2
5
3
1
8
7
1 2
2 3
3 4
4 5
3 6
6 7
	}{%
4
	}%
	\end{example}

	
\end{problem}

