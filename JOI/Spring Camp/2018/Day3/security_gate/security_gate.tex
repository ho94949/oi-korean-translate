\begin{problem}{보안 게이트}
	{standard input}{standard output}
	{5 seconds}{1536 megabytes}{}
	
	당신은 Just Odd Invention 이라는 회사를 아는가? 이 회사의 일은 ``그저 기묘한 발명 (just odd invetion)"을 하는 것이다. 줄여서 JOI 회사라고 부른다.
	
	JOI 회사에서는 기밀정보를 지키기 위해 입구에 보안 게이트를 설치했다. 회사를 출입하기 위해서는 반드시 보안 게이트를 거칠 필요가 있다. 또한, 한 번에 두 명 이상의 사람이 보안 게이트를 지날 수는 없다.
	
	이 보안 게이트는, 사람이 게이트를 통과 할 때 사람이 회사로 들어왔는지 혹은 나왔는지에 대한 정보를 기록한다. 이제, JOI 회사의 사원 IOI군이 어떤 날의 게이트 기록을 보았다. 기록은 문자열 $S$로 표현된다. $S$의 $i$번째 문자가 `\texttt{(}'인 경우, 게이트를 $i$ 번째로 통과한 사람이 회사로 들어 갔다는 의미이고, S의 $i$번째 문자가 `\texttt{)}'인 경우, 게이트를 $i$ 번째로 통과한 사람이 회사에서 나왔다는 의미이다. IOI군은 이 날의 시작 혹은 마지막 날의 JOI 회사 안에 아무도 없다는 것을 알 고 있다. `\texttt{(}' 과 `\texttt{)}'로만 이루어 져있는데, 기록으로 등장할 수 없는 문자열이 있을 수 있음에 유의하여라. 예를 들어, \texttt{())(} 혹은 \texttt{(()}같은 문자열은 기록으로 등장할 수 없는데, JOI 회사 안에 있는 사람의 수가 음수가 되어야 하거나, 이 날의 시작 혹은 마지막 날의 JOI회사 안에 사람이 있었다는 의미이기 때문이다.
	
	IOI군이 기록을 확인 한 순간, $S$는 JOI회사에 퍼져나간 바이러스로 인해 바뀌었다! 조사 이후에, 그는 수정이 다음과 같은 절차로 이루어져 있다고 가정했다:
	
	\begin{itemize}
		\item 처음에, $S$의 연속된 구간에 있는 문자열이 모두 다음과 같이 바뀌었다. 그 구간에 포함되어 있는 문자 전부에 대해, 각 문자가 `\texttt{(}'이라면 `\texttt{)}'로 바뀌고, `\texttt{)}'이라면 `\texttt{(}'로 바뀌었다. 변화 후의 문자열을 $S'$이라고 하자. 여기서 변화한 구간의 길이가 0일수도 있다, 즉, $S=S'$일 수도 있다.
		\item 다음에 $S'$의 0개 이상의 문자가 `\texttt{x}'로 바뀌었다. 변화 후의 문자열은 $S''$이다.
	\end{itemize}
	
	IOI군은 $S$의 정보를 기억하고 있지 않기 때문에, $S''$으로 부터 $S$를 복원하려고 생각하고 있다. 이를 위해 IOI군은 우선 $S'$으로 가능한 문자열의 경우의 수를 세고 싶다. ($S$가 아님에 주의하여라.)
	
	\InputFile
	
	표준 입력에서 다음 입력이 주어진다.
	
	\begin{itemize}
		\item 첫째 줄에는 정수 $N$이 주어진다. 이는 $S''$의 길이가 $N$이라는 의미이다.
		\item 다음 줄에는 각 문자가 `\texttt{(}', `\texttt{)}' 혹은 `\texttt{x}'인 문자열 $S''$이 주어진다.
	\end{itemize}
	
	\OutputFile
	
	표준 출력으로 한 개의 줄을 출력하여라. 출력은 $S'$으로 가능한 문자열의 경우의 수를 $1\ 000\ 000\ 007$ 로 나눈 나머지여야 한다. 만약 해당하는 $S'$이 없으면 0을 출력하여라.
	
	\Constraints
	
	\begin{itemize}
		\item $1 \le N \le 300$.
	\end{itemize}
	
	
	\SubtaskWithCost{1}{4}
	\begin{itemize}
		\item $N \le 100$.
		\item $S''$의 \texttt{x}의 갯수는 최대 4개이다.
	\end{itemize}

	\SubtaskWithCost{2}{8}
\begin{itemize}
	\item $N \le 100$.
	\item $S''$의 \texttt{x}의 갯수는 최대 12개이다.
\end{itemize}

	\SubtaskWithCost{3}{18}
\begin{itemize}
	\item $N \le 100$.
	\item $S''$의 \texttt{x}의 갯수는 최대 20개이다.
\end{itemize}

	\SubtaskWithCost{4}{43}
\begin{itemize}
	\item $N \le 100$.
\end{itemize}


	
	\SubtaskWithCost{5}{27}
	추가 제한조건이 없다.
	
	\Examples
	
	\begin{example}
		\exmp{
			4
			x))x
		}{%
			3
		}%
	\end{example}
	
	이 입력 예제에서, $S' = \texttt{)))(}$은 $S$가 될 수 있는 문자열이 존재하지 않기 때문에, 불가능하다.
	다음 세 문자열이 $S'$이 될 수 있다.
	
	\begin{itemize}
		\item $S' = \texttt{())(}$. 예를 들어, $S = \texttt{()()}$
		\item $S' = \texttt{()))}$. 예를 들어, $S = \texttt{()()}$
		\item $S' = \texttt{))))}$. 예를 들어, $S = \texttt{(())}$
	\end{itemize}
	
	이 세 문자열만 $S'$이 될 수 있으므로, 3을 출력한다.
	
	
	\begin{example}
		\exmp{
			10
			xx(xx()x(x
		}{%
			45
		}%
	\exmp{
		5
		x))x(
	}{%
		0
	}%
	\exmp{
		10
		xxxxxxxxxx
	}{%
		684
	}%
\end{example}


\end{problem}
