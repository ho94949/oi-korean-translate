\begin{problem}{항공노선도}
	{standard input}{standard output}
	{2 seconds}{1024 megabytes}{}
	
	Alice는 JOI왕국에 살고 있다. 그는 IOI공화국에 사는 Bob을 초대하게 되어, 앞서 JOI왕국 내의 항공노선도를 Bob에게 보내게 되었다. JOI왕국은 0번 섬에서 부터 $N-1$번 섬까지 총 $N$ 개의 섬으로 이루어진 섬나라이다. JOI왕국에는 $M$개의 항공노선이 있어서, $i+1$ 번째 ($0 \le i \le M-1$) 항공노선은, $A_i$번 섬과 $B_i$번 섬을 양방향으로 잇는다. 같은 쌍의 섬을 잇는 서로 다른 두 개의 항공노선은 존재하지 않는다. JOI왕국에서 IOI공화국으로 정보를 보내기 위해서는, JOI왕국이 운영하는 전보를 써야한다. 하지만, 전보를 쓸 때, 정점 번호와 간선 번호가 무작위로 섞일 것이다.
	
	정확히는, 정보는 다음과 같은 순서로 보내지게 된다. $G$를 Alice가 보내는 그래프라고 하자. ($V$를 $G$의 정점 개수, $U$를 $G$의 간선 개수라고 하자.)
	
	\begin{itemize}
		\item Alice는 $G$의 정점 개수 $V$와 간선 개수 $U$를 지정한다. 또한, $G$의 각 정점에 $0$, $1$, $\cdots$, $V-1$의 번호를 $G$의 각 간선에 $0$, $1$, $\cdots$, $U-1$의 번호를 각각 붙인다.
		\item Alice는 인자 $C_0$, $C_1$, $\cdots$, $C_{U-1}$ 및 $D_0$, $D_1$, $\cdots$, $D_{U-1}$을 지정한다. 이 인자는 $G$의 간선을 의미한다. 즉, 각 $j$ ($0 \le j \le U-1$) 에 대해, $G$의 간선 $j$는 정점 $C_j$와 $D_j$를 연결한다.
		\item JOI왕국에 의해 $G$의 정점 번호가 무작위로 섞인다. 우선, JOI왕국은 $0$, $1$, $\cdots$, $V-1$로 이루어진 순열 $p[0]$, $p[1]$, $\cdots$, $p[V-1]$을 생성한다. 다음, $C_0$, $C_1$, $\cdots$, $C_{U-1}$을 $p[C_0]$, $p[C_1]$, $\cdots$, $p[C_{U-1}]$로, $D_0$, $D_1$, $\cdots$, $D_{U-1}$을 $p[D_0]$, $p[D_1]$, $\cdots$, $p[D_{U-1}]$로 바꾼다.
		\item 이어서, JOI왕국에 의해 $G$의 간선 번호가 무작위로 섞인다. 우선, JOI왕국은 $0$, $1$, $\cdots$, $U-1$로 이루어진 순열 $q[0]$, $q[1]$, $\cdots$, $q[U-1]$을 생성한다. 다음, $C_0$, $C_1$, $\cdots$, $C_{U-1}$을 $C_{q[0]}$, $C_{q[1]}$, $\cdots$, $C_{q[U-1]}$로, $D_0$, $D_1$, $\cdots$, $D_{U-1}$을 $D_{q[0]}$, $D_{q[1]}$, $\cdots$, $D_{q[U-1]}$로 바꾼다.
		\item $V$, $U$의 값과, 바뀌어진 후의 인자 $C_0$, $C_1$, $\cdots$, $C_{U-1}$ 및 $D_0$, $D_1$, $\cdots$, $D_{U-1}$가 Bob에게 전달된다.
	\end{itemize}

	단, 이 전보로 보낼 수 있는 그래프는 단순그래프 뿐이다. 단순 그래프는 다중간선이나 루프가 없는 그래프이다. 즉, 각 $i$, $j$ ($0 \le i < j \le U-1$)에 대해 $(C_i, D_i) \ne (C_j, D_j)$ 와 $(C_i, D_i) \ne (D_j, C_j)$를 만족하며, 또한 모든 $i$ ($0 \le i \le U-1$) 에 대해, $C_i \ne D_i$ 를 만족하는 그래프를 보내는 것이 가능하다.
	
	Alice는 가능한 한 적은 정점수의 그래프로 Bob에게 JOI왕국 내의 항공노선도를 보내고 싶다.
	
	Alice와 Bob의 통신을 구현하기 위해서, 다음 2개의 프로그램을 작성하여라.
	
	\begin{enumerate}
		\item 첫 번째 프로그램은, JOI왕국의 섬의 수 $N$, JOI왕국의 항공노선의 수 $M$, JOI왕국의 항공노선을 의미하는 수열 $A$, $B$가 주어질 때, Alice가 보내는 그래프 $G$의 정보를 만든다.
		\item 두 번째 프로그램은, Bob이 받은 그래프 $G$가 주어졌을 때, JOI왕국의 항공노선도를 복원한다. 
	\end{enumerate}

\Specification

당신은 파일 두 개를 제출해야 한다.

첫째 파일의 이름은 \texttt{Alice.cpp}이다. 이 파일은 Alice의 일을 나타내고, 다음 함수를 구현해야 한다. 또한, \texttt{Alicelib.h}를 include해야 한다.

\begin{itemize}
	\item \texttt{void Alice( int N, int M, int A[], int B[] )}
	
	이 함수는 각 테스트케이스 마다 정확히 한 번 불린다.
	
	\begin{itemize}
		\item 인자 \texttt{N}은 JOI왕국의 섬의 수를 의미한다.
		\item 인자 \texttt{M}은 JOI왕국의 항공노선의 수 $M$을 의미한다.
		\item 인자 \texttt{A[]}, \texttt{B[]}는 길이 $M$의 배열이며, JOI왕국의 항공노선을 의미한다.
	\end{itemize}
	
	함수 \texttt{Alice} 안에서는, 다음의 함수를 호출해서 보낼 그래프 $G$의 정보를 설정한다.
	\begin{itemize}
	\item \texttt{void InitG( int V, int U )}
	
	이 함수를 호출하는 것으로, G의 정점 개수와 간선 개수를 설정한다.
	
	\begin{itemize}
		\item 인자 \texttt{V}는 $G$의 정점 개수를 의미한다. \texttt{V}는 1 이상 1500 이하의 정수여야 한다. 이 범위 밖의 수로 함수를 호출 한 경우 \textbf{오답 [1]}이 된다.
		\item 인자 \texttt{U}는 $G$의 간선 개수를 의미한다. \texttt{U}는 0 이상 $V(V-1)/2$ 이하의 정수여야 한다. 이 범위 밖의 수로 함수를 호출 한 경우 \textbf{오답 [2]}이 된다.
	\end{itemize}

	\item \texttt{void MakeG( int pos, int C, int D )}

	이 함수는, $G$의 간선을 설정한다.
	
	\begin{itemize}
		\item 인자 \texttt{pos}는 $G$의 정점 번호를 의미한다. \texttt{pos}는 0 이상 $U-1$ 이하의 정수여야 한다. 이 범위 밖의 수로 함수를 호출 한 경우 \textbf{오답 [3]}이 된다. 또한, 같은 인자 \texttt{pos}로 함수를 두 번 이상 호출 할 수 없다. 같은 인자로 두 번 이상 호출 한 경우, \textbf{오답 [4]}이 된다.
		
		\item 인자 \texttt{C}와 \texttt{D}는 \texttt{pos}번 간선의 양쪽의 정점을 의미한다. \texttt{C}와 \texttt{D}는 0 이상 $V-1$ 이하의 정수여야 한다. 또한, \texttt{C} $\ne$ \texttt{D}를 만족해야 한다. \texttt{C} 혹은 \texttt{D}가 이 조건을 만족하지 않은 경우, \textbf{오답 [5]}가 된다.
		
	\end{itemize}

	여기서, $U$와 $V$는 \texttt{initG}에서 설정된 값이다.
	 
	\end{itemize}
	
	함수 \texttt{Alice} 안에서는, 함수 \texttt{InitG}를 한 번 호출 한 후, 함수 \texttt{MakeG}를 정확히 $U$번 호출해야 한다. 함수 \texttt{InitG}가 두 번 이상 호출 된 경우, \textbf{오답 [6]}이 된다. 함수 InitG가 호출 되기 전에, 함수 \texttt{MakeG}가 호출 된 경우, \textbf{오답 [7]}이 된다. 함수 \texttt{Alice}의 실행이 완료 될 때, 함수 \texttt{InitG}가 호출 되지 않은 경우 혹은 \texttt{MakeG}의 호출 횟수가 $U$가 아니었던 경우, \textbf{오답 [8]}이 된다. 함수 \texttt{Alice}의 실행이 종료될 때, 지정한 그래프 $G$가 단순 그래프가 아닌 경우, \textbf{오답 [9]}이 된다.
	
	\texttt{Alice}의 호출이 오답으로 판정된 경우, 그 시점에서 프로그램은 종료된다.
	
\end{itemize}

첫째 파일의 이름은 \texttt{Bob.cpp}이다. 이 파일은 Bob의 일을 나타내고, 다음 함수를 구현해야 한다. 또한, \texttt{Bob.h}를 include해야 한다.

\begin{itemize}
	\item \texttt{void Bob( int V, int U, int C[], int D[] )}
	
	이 함수는 각 테스트케이스 마다 정확히 한 번 불린다.
	
	\begin{itemize}
		\item 인자 \texttt{V}은 그래프 $G$의 정점 개수를 의미한다.
		\item 인자 \texttt{U}은 그래프 $G$의 간선 개수를 의미한다.
		\item 인자 \texttt{C[]}, \texttt{D[]}는 길이 $U$의 배열이며, 그래프 $G$를 의미한다.
	\end{itemize}
	
	함수 \texttt{Bob} 안에서는, 다음의 함수를 호출해서 복원한 JOI왕국의 항공노선도의 정보를 복원한다.
	\begin{itemize}
		\item \texttt{void InitMap( int N, int M )}
		
		이 함수를 호출하는 것으로, JOI왕국의 섬의 수와 JOI왕국의 항공노선의 수를 복원한다.
		
		\begin{itemize}
			\item 인자 \texttt{N}은 복원한 JOI왕국의 섬의 수를 의미한다. \texttt{N}은 실제 JOI왕국의 섬의 수와 일치해야 한다. 일치하지 않을 경우, \textbf{오답 [10]}이 된다.
			\item 인자 \texttt{M}은 복원한 JOI왕국의 항공노선의 수를 의미한다. \texttt{N}은 실제 JOI왕국의 항공노선의 수와 일치해야 한다. 일치하지 않을 경우, \textbf{오답 [11]}이 된다. 
		\end{itemize}
		
		\item \texttt{void MakeMap( int A, int B )}
		
		이 함수를 호출하는 것으로, JOI왕국의 항공노선을 복원한다.
		
		\begin{itemize}
			
			\item 인자 \texttt{A}와 \texttt{B}는 JOI왕국의 \textbf{A}번 섬과 \textbf{B}번 섬을 잇는 항공노선이 존재한다는 것을 의미한다. \texttt{A}와 \texttt{B}는 0 이상 $N-1$ 이하의 정수여야 한다. 또한, \texttt{A} $\ne$ \texttt{B}를 만족해야 한다. \texttt{A} 혹은 \texttt{B}가 이 조건을 만족하지 않은 경우, \textbf{오답 [12]}가 된다. JOI왕국의 섬 \texttt{A}와 섬 \texttt{B}를 연결하는 항공노선이 없을 경우, \textbf{오답 [13]}이 된다. 또한, 과거에 호출한 것과 같은 항공노선을 호출해서는 안된다. \texttt{MakeMap( A, B )}를 호출 할 때, 이미 \texttt{MakeMap( A, B )} 또는 \texttt{MakeMap( B, A )}를 호출 한 경우, \textbf{오답 [14]}이 된다.
			
		\end{itemize}
		
		여기서, $N$은 \texttt{initMap}에서 설정된 값이다.
		
	\end{itemize}
		
	함수 \texttt{Bob} 안에서는, 함수 \texttt{InitMap}을 한 번 호출 한 후, 함수 \texttt{MakeMap}을 정확히 $M$번 호출해야 한다. 함수 \texttt{InitMap}이 두 번 이상 호출 된 경우, \textbf{오답 [15]}이 된다. 함수 InitMap이 호출 되기 전에, 함수 \texttt{MakeMap}이 호출 된 경우, \textbf{오답 [16]}이 된다. 함수 \texttt{Bob}의 실행이 완료 될 때, 함수 \texttt{InitMap}이 호출 되지 않은 경우 혹은 \texttt{MakeMap}의 호출 횟수가 $M$이 아니었던 경우, \textbf{오답 [17]}이 된다. 여기서, $M$은 \texttt{initMap}에서 설정된 값이다.
	
	\texttt{Bob}의 호출이 오답으로 판정된 경우, 그 시점에서 프로그램은 종료된다.
	
\end{itemize}


채점은 다음과 같은 방식으로 진행된다. 오답으로 판정된 경우, 그 시점에서 프로그램은 종료된다.

\begin{enumerate}
	\item[(1)] JOI왕국의 항공노선도의 정보를 인자로 하여, 함수 \texttt{Alice}를 한 번 호출한다.
	\item[(2)] 함수 \texttt{Alice}안에서 설정된 그래프를 $G$라고 하자. $G$의 정점 번호와 간선 번호를 뒤섞은 그래프를 의미하는 정보를 인자로 하여, 함수 \texttt{Bob}을 한 번 호출한다.
	\item[(3)] 채점이 종료된다.
\end{enumerate}


\Notes

\begin{itemize}
	\item 당신의 프로그램은 내부에서 사용할 목적으로 함수나 전역변수를 사용할 수 있다. 제출된 파일들은 같이 컴파일 되어 하나의 실행 파일이 된다. 모든 글로벌 변수나 함수는 충돌을 피하기 위하여 \texttt{static}으로 선언되어야 한다. 채점 될 때는, Alice과 Bob에 해당하는 두 프로세스로 나누어서 실행 될 것이다. Alice의 프로세스와 Bob의 프로세스는 전역변수를 공유할 수 없다.
	\item 당신의 프로그램은 표준 입출력을 사용해서는 안된다. 당신의 프로그램은 어떠한 방법으로도 다른 파일에 접근해서는 안된다. 단, 당신의 프로그램은 디버그 목적으로 표준 에러출력에 출력할 수 있다.
\end{itemize}

당신은 대회 홈페이지의 아카이브에서 프로그램을 테스트 하기 위한 목적의 샘플 그레이더를 받을 수 있다. 아카이브는 당신의 프로그램의 예제 소스 또한 첨부되어 있다.
샘플 그레이더는 파일 \texttt{grader.cpp}이다. 당신의 프로그램을 테스트 하기 위해서, \texttt{grader.cpp}, \texttt{Alice.cpp}, \texttt{Bob.cpp}, \texttt{Alicelib.h}, \texttt{Boblib.h}를 같은 디렉토리 안에 놓고, 컴파일 하기 위해 다음 커맨드를 실행하여라.

\begin{itemize}
	\item \texttt{g++ -std=c++14 -O2 -o grader grader.cpp Alice.cpp Bob.cpp}
\end{itemize}

컴파일이 성공적이면, 파일 \texttt{grader}가 생성된다.

실제 그레이더와 샘플 그레이더는 다름에 주의하여라. 샘플 그레이더는 하나의 프로세스에서 실행 되며, 입력을 표준 입력으로 부터 받고, 출력을 표준 출력에 출력한다.

\InputFile



샘플 그레이더는 표준 입력에서 다음과 같은 형식으로 입력받는다.

\begin{itemize}
	\item 첫째 줄에는 정수 $N$, $M$이 주어진다. 이는 JOI왕국의 섬의 수가 $N$이며, JOI왕국의 항공노선의 수가 $M$이라는 것을 의미한다.
	\item 다음 $M$개의 줄에는 $M$개의 항공노선의 정보가 주어진다. $M$개의 줄 중 $i+1$ 번째 ($0 \le i \le M-1$) 줄에는 두 개의 정수 $A_i$, $B_i$가 주어진다. 이는 JOI왕국의 항공노선의 정보를 의미한다.
\end{itemize}

\OutputFile

프로그램이 정상적으로 종료되었다면, 샘플 그레이더는 다음과 같은 정보를 표준 출력 및 표준 에러에 출력한다. (따옴표는 출력하지 않는다.)

\begin{itemize}
	\item 프로그램의 실행 중에 오답으로 판단 된 경우, 오답의 종류를 ``\texttt{Wrong Answer [1]}"과 같은 형식으로 출력하고, 실행이 종료된다.
	
	\item \texttt{Alice}와 \texttt{Bob} 모두 오답이 아닌 경우, ``\texttt{Accepted}"가 출력되고 또한, $V$의 값이 출력된다.

\end{itemize}

프로그램이 다양한 오답의 종류에 속해 있을 경우, 샘플 그레이더는 그 중 하나만 출력 할 것이다.

\Constraints

\begin{itemize}
	\item $1 \le N \le 1\ 000$.
	\item $0 \le M \le N(N-1)/2$.
	\item $0 \le A_i \le N-1$ ($0 \le i \le M-1$).
	\item $0 \le B_i \le N-1$ ($0 \le i \le M-1$).
	\item $A_i \ne B_i$ ($0 \le i \le M-1$).
	\item $(A_i,B_i) \ne (A_j, B_j)$ 이며, $(A_i, B_i) \ne (B_j, A_j)$ ($0 \le  i < j \le M-1$).
\end{itemize}




\SubtaskWithCost{1}{22}
\begin{itemize}
	\item $N \le 22$
\end{itemize}

\SubtaskWithCost{2}{15}
\begin{itemize}
	\item $N \le 40$
\end{itemize}

\SubtaskWithCost{3}{63}

추가 제한조건이 없다.


\Scoring

\begin{itemize}
	\item 서브태스크 1 혹은 2에서는, 각 서브태스크의 모든 테스트 케이스를 맞은 경우, 만점이다.
	\item 서브태스크 3에서는, 모든 테스트 케이스를 맞은 경우, $V-N$의 최댓값을 $\textrm{MaxDiff}$로 해서, 다음 기준으로 채점한다.
	\begin{itemize}
		\item $101 \le \textrm{MaxDiff}$ 이면, 0점.
		\item $21 \le \textrm{MaxDiff} \le 100$ 이면, $13+\left\lfloor\dfrac{100-\textrm{MaxDiff}}{4}\right\rfloor$점. 여기서, $x$를 넘지 않는 최대의 정수를 $\left\lfloor x \right\rfloor$ 로 표현한다.
		\item $13 \le \textrm{MaxDiff} \le 20$이면, $33+(20-\textrm{MaxDiff}) \times 3$점.
		\item $\textrm{MaxDiff} \le 12$ 이면, 63점.
	\end{itemize}
\end{itemize}

\Examples

이 함수는 그레이더의 예제 입력과 해당하는 함수 호출을 보여준다.

\begin{tabular}{|l|l|l|l|l|}
	\hline
	\multirow{2}{*}{예제 입력 1}                                                        & \multicolumn{4}{l|}{함수 호출의 예}           \\ \cline{2-5} 
	& 호출         & 반환값  & 호출           & 반환값  \\ \hline
	\multirow{20}{*}{\begin{tabular}[c]{@{}l@{}}\texttt{4 3}\\ \texttt{0 1}\\ \texttt{0 2}\\ \texttt{0 3}\end{tabular}} & \texttt{Alice(...)} &      &              &      \\ \cline{2-5} 
	&            &      & \texttt{InitG(4,3)}   &      \\ \cline{2-5} 
	&            &      &              & (없음) \\ \cline{2-5} 
	&            &      & \texttt{MakeG(0,0,1)} &      \\ \cline{2-5} 
	&            &      &              & (없음) \\ \cline{2-5} 
	&            &      & \texttt{MakeG(1,0,2)} &      \\ \cline{2-5} 
	&            &      &              & (없음) \\ \cline{2-5} 
	&            &      & \texttt{MakeG(2,0,3)} &      \\ \cline{2-5} 
	&            &      &              & (없음) \\ \cline{2-5} 
	&            & (없음) &              &      \\ \cline{2-5} 
	& \texttt{Bob(...)}   &      &              &      \\ \cline{2-5} 
	&            &      & \texttt{InitMap(4,3)} &      \\ \cline{2-5} 
	&            &      &              & (없음) \\ \cline{2-5} 
	&            &      & \texttt{MakeMap(0,1)} &      \\ \cline{2-5} 
	&            &      &              & (없음) \\ \cline{2-5} 
	&            &      & \texttt{MakeMap(0,2)} &      \\ \cline{2-5} 
	&            &      &              & (없음) \\ \cline{2-5} 
	&            &      & \texttt{MakeMap(0,3)} &      \\ \cline{2-5} 
	&            &      &              & (없음) \\ \cline{2-5} 
	&            & (없음) &              &      \\ \hline
\end{tabular}

이 때, \texttt{Alice(...)}, \texttt{Bob(...)}에 주어진 인자는 다음과 같다.

\begin{tabular}{|l|l|l|}
	\hline
	인자 & Alice(...) & Bob(...)  \\ \hline
	N  & 4          &           \\ \hline
	M  & 3          &           \\ \hline
	V  &            & 4         \\ \hline
	U  &            & 3         \\ \hline
	A  & \{0,0,0\}  &           \\ \hline
	B  & \{1,2,3\}  &           \\ \hline
	C  &            & \{2,2,2\} \\ \hline
	D  &            & \{3,0,1\} \\ \hline
\end{tabular}

\begin{tabular}{|l|}
	\hline
	예제 입력 2                                                                               \\ \hline
	\begin{tabular}[c]{@{}l@{}}\texttt{5 7}\\ \texttt{0 1}\\ \texttt{0 2}\\ \texttt{1 3}\\ \texttt{1 4}\\ \texttt{3 4}\\ \texttt{2 3}\\ \texttt{2 4}\end{tabular} \\ \hline
\end{tabular}

\end{problem}

