\begin{problem}{비타로의 파티}
	{}{}
	{2초}{512MB}{}
	
	비버가 사는 마을 $N$개가 높이가 감소하는 순으로 1번부터 $N$번까지 번호가 붙어있다. 어떤 두 마을도 같은 높이에 있지 않다. 서로 다른 마을을 단방향으로 잇는 수로 $M$ 개가 존재한다. $i$ 번째 ($1 \le i \le M$) 수로는 $S_i$번 마을에서 $E_i$번 마을로 흐른다. 이 수로는 높은 마을에서 낮은 마을로 흐른다. 수로를 거슬로 올라갈 수는 없다.
	
	비버 비타로는 $N$ 명의 친구가 있고 $N$ 개의 마을에 각각 한 마리씩 살고 있다.
	
	비타로는 친구를 초대해서 파티를 $Q$ 번 할 것이다. $j$ 번째 ($1 \le j \le Q$) 파티에는 $Y_j$ 마리의 비버가 너무 바빠서 참석하지 못한다. $j$ 번째 파티는 $T_j$번 마을에서 열리기 때문에, 수로를 통해서 $T_j$번 마을로 올 수 없는 친구들도 참석할 수 없다. 다른 친구들은 파티에 참석한다.
	
	각 친구는 수로를 통해서 파티가 개최되는 마을로 이동한다. 이동하는 경로가 여러 개 존재할 수도 있다. 하지만 비타로의 친구들은 수로를 너무 좋아해서, 여러 개의 경로가 있는 경우에는 거치는 수로의 개수가 최대가 되는 경로로 이동한다.
	
	비타로는 각각의 파티에 대해, 가장 많은 수의 수로를 거친 친구가 몇 개의 수로를 경유했는지가 궁금했다.

	\InputFile
	
	표준 입력에서 다음 입력이 주어진다.
	
	\begin{itemize}
		\item 첫째 줄에는 세 정수 $N$, $M$, $Q$가 공백으로 구분되어 주어진다. 이는 마을이 $N$ 개 있고, 수로가 $M$ 개 있으며, 비타로가 $Q$ 번의 파티를 연다는 것을 의미한다.
		\item 다음 $M$ 개의 줄에는 수로의 정보가 주어진다. $M$ 개의 줄의 $i$ 번째 ($1 \le i \le M$) 줄에는, 정수 $S_i$, $E_i$ 가 공백으로 구분되어 주어진다. 이는 $S_i$번 마을 부터 $E_i$번 마을까지 단방향인 수로가 흐르고 있다는 것을 의미한다.
		\item 다음 $Q$개의 줄에는 비타로가 여는 파티의 정보가 주어진다. $Q$ 개의 줄의 $j$ 번째 ($1 \le j \le Q$) 줄에는, 정수 $T_j$, $Y_j$와 $Y_j$개의 정수 $C_{j, 1}, C_{j,2 }, \cdots, C_{j, Y_j}$가 공백으로 구분되어 주어진다. 이는, $j$ 번째 파티는 $T_j$번 마을에서 열리고, $C_{j, 1}, C_{j, 2}, \cdots, C_{j, Y_j}$번 마을에 사는 친구는 너무 바빠서 참석할 수 없다는 것을 의미한다.
	\end{itemize}
		
	\OutputFile
	
	표준 출력으로 $Q$개의 줄을 출력하여라. $j$ 번째 ($1 \le j \le Q$) 줄은 $j$ 번째 파티에 대해 가장 많은 수의 수로를 경유한 친구가 거친 수로의 개수를 출력하여라. 단, 파티에 친구가 한 명도 참석하지 않을 경우 대신 \texttt{-1}을 출력하여라.
		
	\Constraints
	
	\begin{itemize}
		\item $1 \le N \le 100\ 000$.
		\item $0 \le M \le 100\ 000$.
		\item $1 \le Q \le 100\ 000$.
		\item $1 \le S_i < E_i \le N$ ($1 \le i \le M$).
		\item $(S_i, E_i) \ne (S_j, E_j)$ ($1 \le i < j \le M$).
		\item $1 \le T_j \le N$ ($1 \le j \le Q$).
		\item $0 \le Y_j \le N$ ($1 \le j \le Q$).
		\item $1 \le C_{j, 1} < C_{j, 2}< \cdots < C_{j, Y_j} \le N$ ($1 \le j \le Q$).
		\item $Y_1 + Y_2 + \cdots + Y_Q \le 100\ 000$.
	\end{itemize}
	
	
	\SubtaskWithCost{1}{7}
	\begin{itemize}
		\item $N \le 1\ 000$.
		\item $M \le 2\ 000$.
		\item $Q=1$.
	\end{itemize}

	\SubtaskWithCost{1}{7}
	\begin{itemize}
		\item $Q=1$.
	\end{itemize}


	\SubtaskWithCost{3}{86}
	추가 제한조건이 없다.
	
	\Examples
	
	\begin{example}
		\exmp{
			5 6 3
			1 2
			2 4
			3 4
			1 3
			3 5
			4 5
			4 1 1
			5 2 2 3
			2 3 1 4 5
		}{%
			1
			3
			0
		}%
	\end{example}
	
	첫 번째 파티에 참석한 친구 (2번, 3번, 4번 마을에 사는 친구) 중에서는 2번 마을에 사는 친구와 3번 마을에 살고있는 친구가 파티에 개최되는 4번 마을까지 거친 수로의 수가 제일 많다. 이는 1개이므로, 1을 출력한다.
	
	두 번째 파티에 참석한 친구 (1번, 4번, 5번 마을에 사는 친구) 중에서는 1번 마을에 사는 친구가 파티에 개최되는 5번 마을까지 거친 수가 제일 많다. 이는 3개이므로, 3을 출력한다.
	
	세 번째 파티에 참석하는 친구는, 파티가 개최되는 2번 마을에 사는 친구밖에 없다. 이 친구는 수로를 경유하지 않으므로, 0을 출력한다.
	
	\begin{example}
		\exmp{
12 17 10
1 2
2 3
3 4
1 5
2 6
3 7
4 8
5 6
6 7
7 8
5 9
6 10
7 11
8 12
9 10
10 11
11 12
6 3 1 7 12
3 7 1 2 3 4 5 6 7
11 3 1 3 5
9 2 1 9
8 4 1 2 3 4
1 1 1
12 0
10 3 1 6 10
11 8 2 3 5 6 7 9 10 11
8 7 2 3 4 5 6 7 8
}{%
1
-1
3
1
3
-1
5
2
4
4
}%
	\end{example}
	
	
\end{problem}

