\begin{problem}{수행}
	{standard input}{standard output}
	{0.6초}{256MB}{}
	
	
	타임머신을 손에 넣은 JOI군은 약 1200년 전의 일본으로 여행을 갔다. 거기서 JOI군은 고보대사라는 이름으로 알려진 승려 구카이를 만났다. 구카이는 새로운 수행방법을 생각하고 있었다.
	
	구카이는 다음과 같은 방법으로 수행을 하려는 중이다.
	
	\begin{itemize}
		\item 구카이는 $N$ 개의 문장으로 된 불경을 읽는다. $N$ 개의 문장에는 순서가 있어서, 구카이는 그 순서대로 읽어야 한다.
		\item 각각의 문장은 1 이상 $N$ 이하의 정수 하나가 붙어있다. 단, 서로 다른 문장에 같은 정수가 붙은 경우는 없다.
		\item 정수 $i$ ($1 \le i \le N$) 이 붙은 문장은, 하루를 $N$등분 했을 때 $i$ 번째 시간에 읽어야 한다. 각각의 문장은 매우 짧기 때문에 이 시간 안에 문장을 모두 읽는 것이 가능하다.
	\end{itemize}

	구카이는 어떤 날부터 하루가 시작할 때 수행을 시작해서 가장 빨리 이 수행을 끝내고 싶다. 문장에 붙은 정수에 따라 수행에 며칠이 걸리는지가 달라진다. JOI군은 가장 빨리 수행을 끝낼 때 정확히 $K$ 일이 걸리도록 문장에 정수를 붙이는 방법이 몇 가지 있는지를 구해달라는 구카이의 부탁을 받았다.
	
	불경의 문장 수 $N$과 $K$가 주어질 때, 불경을 모두 읽는 데 가장 빠른 방법으로 읽으면 정확히 $K$ 일이 걸리도록 문장에 정수를 붙이는 방법의 가짓수를 1 000 000 007로 나눈 나머지를 구하는 프로그램을 작성하여라.
	
	
	\InputFile
	
	표준 입력에서 다음 입력이 주어진다.
	
	\begin{itemize}
		\item 첫째 줄에는 두 개의 정수 $N$, $K$가 공백으로 구분되어 주어진다. 이는 불경이 $N$ 개의 문장으로 되어 있고,  불경을 모두 읽는 데 가장 빠른 방법으로 읽으면 정확히 $K$ 일이 걸리도록 문장에 정수를 붙이는 방법의 가지수를 구해야 한다는 것을 의미한다.
	\end{itemize}
		
	\OutputFile
	
	가장 빨리 수행을 끝낼 때 정확히 $K$ 일이 걸리도록 문장에 정수를 붙이는 가짓수를 1 000 000 007로 나눈 나머지를 출력하여라.
	
	\Constraints
	
	\begin{itemize}
		\item $1 \le N \le 100\ 000$.
		\item $1 \le K \le N$.
	\end{itemize}
	
	
	\SubtaskWithCost{1}{4}
	\begin{itemize}
		\item $N \le 10$.
	\end{itemize}

	\SubtaskWithCost{2}{20}
	\begin{itemize}
		\item $N \le 300$.
	\end{itemize}
	
	\SubtaskWithCost{3}{25}
	\begin{itemize}
		\item $N \le 3\ 000$.
	\end{itemize}



	\SubtaskWithCost{4}{51}
	추가 제한조건이 없다.
	
	\Examples
	
	\begin{example}
		\exmp{
			3 2
		}{%
			4
		}%
	\end{example}
	
	 불경을 모두 읽는 데 가장 빠른 방법으로 읽으면 정확히 2일이 걸리는 경우는 아래 4가지이다.
	 
	 \begin{itemize}
	 	\item 첫 번째 문장에 1, 두 번째 문장에 3, 세 번째 문장에 2가 붙어 있다. 이때, 첫째 날에 (1과 3이 붙어 있는) 처음 두 개의 문장을 읽고, 둘째 날에 (2가 붙어있는) 마지막 문장을 읽는다.
	 	\item 첫 번째 문장에 2, 두 번째 문장에 1, 세 번째 문장에 3이 붙어 있다. 
	 	\item 첫 번째 문장에 2, 두 번째 문장에 3, 세 번째 문장에 1이 붙어 있다. 
	 	\item 첫 번째 문장에 3, 두 번째 문장에 1, 세 번째 문장에 2가 붙어 있다. 
	 \end{itemize}

	\begin{example}
		\exmp{
10 5
}{%
1310354
}%
	\end{example}
	
	
\end{problem}

