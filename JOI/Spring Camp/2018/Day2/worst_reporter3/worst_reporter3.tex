\begin{problem}{최악의 기자 3}
	{standard input}{standard output}
	{2초}{256MB}{}
	
	IOI 2018의 개회식에는 $N$ 명의 선수가 일렬로 서 행진한다. 선수가 행진하는 도로는 수직선으로 표현된다. 선수는 전원 수직선 위의 양의 방향을 향해 행진하고 있다. 처음에 앞에서 $i$ 번째 ($1 \le i \le N$) 선수는 좌표 $-i$에 있다. 좌표 0에는 기수 IOI 양이 있다.
	
	모든 선수에는 \textbf{게으름}이라는 값이 정해져 있다. 왼쪽에서 $i$ 번째 선수의 게으름은 $D_i$이다. 선수들은 보통 다음의 규칙에 따라 행동한다.
	
	\begin{itemize}
		\item 앞에서 $i$ 번째 선수는 자신의 바로 앞에 참가자와의 (선수 혹은 IOI 양) 거리가 $D_i+1$ 이상 떨어져 있으면, 바로 앞 참가자와 거리가 1 차이 나는 위치까지 이동한다. 그렇지 않을 경우, 이동하지 않는다.
	\end{itemize}
	
	IOI 양은 단위 시각 마다 1만큼 양의 방향으로 움직인다. 선수들은 모두 위에 쓰인 조건을 만족하면 움직인다.
	
	당신은 개회식을 취재하러 온 기자이다. 당신은 사진을 찍어야 했지만, 개회식 도중에 잠에 빠져버렸다. 어쩔 수 없이 개회장의 사진을 찍어 그 사진에 참가자의 사진을 합성하려고 했다.
	
	사진이 합성되었다는 것을 들키지 않고 합성하는 시간을 예측하기 위해서, 당신은 다음 $Q$개의 값을 알고 싶다.
	
	\begin{itemize}
		\item 시각 $T_j$에, $L_j$ 이상 $R_j$ 이하의 좌표에 위치한 참가자의 수 ($1 \le j \le Q$)
	\end{itemize}

	각 선수의 게으름과 $Q$개의 질문의 정보가 주어졌을 때, 각각에 질문에 대해 조건을 만족하는 참가자의 수를 출력하는 프로그램을 작성하여라.
	
	
	\InputFile
	
	표준 입력에서 다음 입력이 주어진다.
	
	\begin{itemize}
		\item 첫째 줄에는 정수 $N$, $Q$가 공백으로 구분되어 주어진다. 이는 선수의 수와 질문의 수를 의미한다. 선수의 수를 셀 때, IOI 양은 세지 않음에 주의하여라.
		\item 다음 $N$개의 줄의 $i$ 번째 ($1 \le i \le N$) 줄에는 정수 $D_i$가 주어진다. 이는 $i$ 번째 선수의 게으름을 의미한다.
		\item 다음 $Q$개의 줄의 $j$ 번째 ($1 \le j \le Q$) 줄에는 정수 $T_j$, $L_j$, $R_j$가 주어진다. 이는 $j$ 번째 질문의 정보를 의미한다.
	\end{itemize}
		
	\OutputFile
	
	표준 출력에 $Q$ 개의 줄을 출력하여라. $j$ 번째 ($1 \le j \le Q$) 줄에는 $j$ 번째 질문에 대한 답을 정수로 출력하여라.
	
	\Constraints
	
	\begin{itemize}
		\item $1 \le N \le 500\ 000$.
		\item $1 \le Q \le 500\ 000$.
		\item $1 \le D_i \le 1\ 000\ 000\ 000$ ($1 \le i \le N$).
		\item $1 \le T_j \le 1\ 000\ 000\ 000$ ($1 \le j \le Q$).
		\item $1 \le L_j \le R_j \le 1\ 000\ 000\ 000$ ($1 \le j \le Q$).
	\end{itemize}
	
	
	\SubtaskWithCost{1}{7}
	\begin{itemize}
		\item $D_i = 1$ ($1 \le i \le N$).
	\end{itemize}

	\SubtaskWithCost{2}{12}
	\begin{itemize}
		\item $N \le 1\ 000$.
		\item $Q \le 1\ 000$.
		\item $T_j \le 1\ 000$. ($1 \le j \le Q$).
		\item $1 \le L_j \le R_j \le 1\ 000$ ($1 \le j \le Q$).
	\end{itemize}
	

	\SubtaskWithCost{3}{81}
	추가 제한조건이 없다.
	
	\Examples
	
	\begin{example}
		\exmp{
			3 6
			2
			5
			3
			1 2 4
			2 2 4
			3 2 4
			4 2 4
			5 2 4
			6 2 4
		}{%
			0
			1
			1
			2
			1
			2
		}%
	\end{example}
	
	이 입력 예제에서, 선수와 IOI 양은 다음과 같이 행진한다.
	
	수직선의 좌표 중 $L$이상 $R$이하인 점 전체를 $[L, R]$로 표현한다.
	
	\begin{itemize}
	
	\item 처음에, IOI 양은 좌표 0에, 1, 2, 3번째 선수는 좌표 $-1$, $-2$, $-3$에 있다.
	
	\item 시각 1에, IOI 양이 좌표 1로 행진한다. 행진하는 선수는 없고 1, 2, 3번째 선수는 좌표 $-1$, $-2$, $-3$에 있다. 구간 $[2, 4]$에 아무도 없으므로, 첫 번째 질문에는 0을 출력한다.
	\item 시각 2에, IOI 양이 좌표 2로 행진한다. IOI 양과 첫 번째 선수의 거리가 3이 되었기 때문에, 첫 번째 선수는 좌표 1로 행진한다. 1, 2, 3번째 선수는 좌표 $1$, $-2$, $-3$에 있다. 구간 $[2, 4]$에 IOI 양 혼자 있으므로, 두 번째 질문에는 1을 출력한다. 
	\item 시각 3에, IOI 양이 좌표 3으로 행진한다. 행진하는 선수는 없고 1, 2, 3번째 선수는 좌표 $1$, $-2$, $-3$에 있다. 구간 $[2, 4]$에 IOI 양 혼자 있으므로, 세 번째 질문에는 1을 출력한다. 
	\item 시각 4에, IOI 양이 좌표 4로 행진한다. IOI 양과 첫 번째 선수의 거리가 3이 되었기 때문에, 첫 번째 선수는 좌표 3으로 행진한다. 1, 2, 3번째 선수는 좌표 $3$, $-2$, $-3$에 있다. 구간 $[2, 4]$에 IOI 양과 첫 번째 선수가 있으므로, 네 번째 질문에는 2를 출력한다. 
	\item 시각 5에, IOI 양이 좌표 5로 행진한다. 행진하는 선수는 없고 1, 2, 3번째 선수는 좌표 $3$, $-2$, $-3$에 있다. 구간 $[2, 4]$에 첫 번째 선수 혼자 있으므로, 다섯 번째 질문에는 1을 출력한다. 
	\item 시각 6에, IOI 양이 좌표 6으로 행진한다. IOI 양과 첫 번째 선수의 거리가 3이 되었기 때문에, 첫 번째 선수는 좌표 3으로 행진한다. 또한, 첫 번째 선수와 두 번째 선수의 거리가 7이 되었기 때문에, 두 번째 선수는 좌표 4로 행진한다. 또한, 두 번째 선수와 세 번째 선수의 거리가 7이 되었기 때문에, 두 번째 선수는 좌표 3으로 행진한다. 1, 2, 3번째 선수는 좌표 $5$, $4$, $3$에 있다. 구간 $[2, 4]$에 두 번째 선수와 세 번째 선수가 있으므로, 여섯 번째 질문에는 2를 출력한다.
	
	\end{itemize}

	\begin{example}
		\exmp{
4 2
1
1
1
1
2 1 4
1 3 6
}{%
2
0
}%
	\end{example}

이 입력 예제는 서브태스크 1의 조건을 만족한다,	
	
	
\begin{example}
	\exmp{
		6 6
		11
		36
		28
		80
		98
		66
		36 29 33
		190 171 210
		18 20 100
		1000 900 1100
		92 87 99
		200 100 300
	}{%
		1
		6
		0
		5
		2
		7
	}%
\end{example}

	
\end{problem}

