\begin{problem}{고속도로 건설}
	{standard input}{standard output}
	{1초}{256MB}{}
	
	
	JOI 왕국에는 $1$번부터 $N$번까지 번호가 붙은 $N$ 개의 도시가 있다. $1$번 도시는 수도이다. 각 도시에는 \textbf{활기}라고 부르는 값이 있어서, $i$번 ($1 \le i \le N$) 도시의 활기의 초깃값은 $C_i$이다.
	
	JOI 왕국의 도로는 두 개의 서로 다른 도시를 양방향으로 잇는다. JOI 왕국은 처음에는 도로가 없었다. JOI 왕국은 이제부터 $N-1$ 개의 도로를 건설하려고 한다. $j$ 번째 ($1 \le j \le N-1$) 도로 건설은 다음과 같이 진행된다.
	
	\begin{itemize}
		\item $A_j$번 도시와 $B_j$번 도시를 지정한다. 미리 건설된 도로를 사용해서 1번 도시에서 $A_j$번 도시로 갈 수는 있지만, 1번 도시에서 $B_j$번 도시로 갈 수는 없다.
		
		\item $A_j$번 도시와 $B_j$번 도시를 잇는 도로를 짓는다. 건설 비용은 다음 조건을 만족하는 $(s, t)$ 쌍의 개수이다.
		
		\begin{itemize}
			\item[] $s$번 도시와 $t$번 도시는 1번 도시와 $A_j$번 도시를 잇는 최단 경로상에 있고, 1번 도시에서 $A_j$번 도시로 갈 때 $s$번 도시를 $t$번 도시보다 먼저 방문하며, $s$번 도시의 활기는 $t$번 도시의 활기보다 크다.
		\end{itemize}
	
		여기서, 1번 도시와 $A_j$번 도시를 잇는 경로는 1번 도시와 $A_j$번 도시를 포함한다. 1번 도시와 $A_j$번 도시를 잇는 최단 경로는 유일함에 유의하여라.
		
		\item 1번 도시와 $A_j$번 도시를 잇는 최단 경로상에 있는 모든 도시의 활기를 $B_j$번 도시의 활기로 바꾼다.		 
	\end{itemize}
	
	각각 도로 건설에 걸리는 비용을 알고 싶다.
	
	도시와 도로의 건설 계획에 대한 정보가 주어졌을 때, 각각의 도로 건설에 드는 비용을 구하는 프로그램을 작성하여라.
	
	
	\InputFile
	
	표준 입력에서 다음 입력이 주어진다.
	
	\begin{itemize}
		\item 첫째 줄에는 정수 $N$이 주어진다. 이는 JOI 왕국에 $N$ 개의 도시가 있다는 것을 의미한다.
		\item 둘째 줄에는 $N$ 개의 정수 $C_1$, $C_2$, $\cdots$, $C_N$이 공백으로 구분되어 주어진다. 이는 $i$번 ($1 \le i \le N$) 도시의 활기의 초깃값이 $C_i$라는 것을 의미한다.
		\item 다음 $N-1$ 개의 줄의 $j$ 번째 ($1 \le j \le N-1$) 줄에는 두 개의 정수 $A_j$, $B_j$ 가 공백으로 구분되어 주어진다. 이는 $j$ 번째 도로 건설에 $A_j$번 도시와 $B_j$번 도시가 지정됨을 의미한다.
	\end{itemize}
		
	\OutputFile
	
	표준 출력으로 $N-1$ 개의 줄을 출력한다. $N-1$ 개의 줄의 $j$ 번째 ($1 \le j \le N-1$) 줄에는 $j$ 번째 도로 건설에 드는 비용을 출력하여라.
	
	\Constraints
	
	\begin{itemize}
		\item $1 \le N \le 100\ 000$.
		\item $1 \le C_i \le 1\ 000\ 000\ 000$ ($1 \le i \le N$).
		\item $1 \le A_j \le N$ ($1 \le j \le N-1$).
		\item $1 \le B_j \le N$ ($1 \le j \le N-1$).
		\item $j$ 번째 도로 건설보다 전에 지어진 도로를 사용해서 1번 도시에서 $A_j$번 도시로 갈 수는 있지만, 1번 도시에서 $B_j$번 도시로 갈 수는 없다. ($1 \le j \le N-1$)
	\end{itemize}
	
	
	\SubtaskWithCost{1}{7}
	\begin{itemize}
		\item $N \le 500$.
	\end{itemize}

	\SubtaskWithCost{2}{9}
	\begin{itemize}
		\item $N \le 4\ 000$.
	\end{itemize}
	

	\SubtaskWithCost{3}{84}
	추가 제한조건이 없다.
	
	\Examples
	
	\begin{example}
		\exmp{
			5
			1 2 3 4 5
			1 2
			2 3
			2 4
			3 5
		}{%
			0
			0
			0
			2
		}%
	\end{example}
	
	이 입력 예제에서는, 다음과 같이 도로 건설이 진행된다.
	
	\begin{itemize}
		\item 첫 번째 도로 건설을 만족하는 $s$번 도시와 $t$번 도시는 없기 때문에 건설 비용은 0이다. 1번 도시와 2번 도시를 잇는 도로를 지은 후에 1번 도시의 활기를 2로 바꾼다.
		\item 두 번째 도로 건설도 건설 비용은 0이다. 2번 도시와 3번 도시를 잇는 도로를 지은 후에 1번 도시와 2번 도시의 활기를 3으로 바꾼다.
		\item 세 번째 도로 건설도 건설 비용은 0이다. 2번 도시와 4번 도시를 잇는 도로를 지은 후에 1번 도시와 2번 도시의 활기를 4로 바꾼다.
		\item 네 번째 도로 건설은 $(s, t) = (1, 3), (2, 3)$이 조건을 만족하므로 건설 비용은 2이다. 3번 도시와 5번 도시를 잇는 도로를 지은 후에 1번 도시와 2번 도시와 3번 도시의 활기를 5로 바꾼다.
	\end{itemize}
	
	\begin{example}
		\exmp{
10
1 7 3 4 8 6 2 9 10 5
1 2
1 3
2 4
3 5
2 6
3 7
4 8
5 9
6 10
}{%
0
0
0
1
1
0
1
2
3
}%
	\end{example}
	
	
\end{problem}

