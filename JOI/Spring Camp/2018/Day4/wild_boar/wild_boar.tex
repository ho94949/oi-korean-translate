\begin{problem}{멧돼지}
	{standard input}{standard output}
	{10초}{1024MB}{}
	
	멧돼지 JOI군이 사는 IOI 숲에는 $N$개의 야생동물 급식소가 있고, $M$개의 도로가 있다. 급식소에는 1번 부터 $N$번 까지의 번호가 붙어있다. $i$ 번째 ($1 \le i \le M$) 도로는 $A_i$번 급식소와 $B_i$번 급식소를 양방향으로 이으며, JOI군이 이 도로를 지나는 데 양방향 모두 $C_i$ 시간이 걸린다. 어떤 급식소에서도 다른 모든 급식소까지 한 개 이상의 도로를 통해 서로 오가는 것이 가능하다.
	
	JOI군은 유턴을 잘 못 하기 때문에, 도로를 가던 도중 유턴해서 출발했던 급식소로 돌아올 수 없다. 또한, 급식소에 도착한 이후에는 가장 최근에 사용한 도로를 써서 되돌아가는 것도 할 수 없다.
	
	매일 JOI군은 보급계획에 따라서 음식을 먹는다. 하루의 보급 계획은 음식을 보급받을 $L$개의 급식소 번호를 차례로 나열한 $X_1$, $X_2$, $\cdots$, $X_L$이다. JOI군은 $X_1$번 급식소에서 시작해서 해당 순서로 급식소를 방문 하여 $X_L$번을 마지막으로 보급계획을 끝낸다. 중간에 다른 급식소를 거쳐도 된다. 보급 계획에 같은 급식소가 여러 번 등장할 수는 있지만, 모든 $j$ ($1 \le j \le L-1$) 에 대해, $X_j \ne X_{j+1}$을 만족한다. 보급계획 중에는 불가능 한 것이 있을 수도 있다.
	
	처음에 JOI군은 최초 보급계획 $X_1$, $X_2$, $\cdots$, $X_L$을 정한다. $k$ 번째 ($1 \le k \le T$) 날 아침, JOI군은 $P_k$번째 값을 $Q_k$로 바꿀 것이다. (즉, $X_{P_k}$ 가 $Q_k$가 된다. 그리고 새로운 보급계획을 따라 음식을 먹는다. 모든 $j$ ($1 \le j \le L-1$) 에 대해, 값이 바뀐 이후에도 $X_j \ne X_{j+1}$을 만족함이 보장된다.
	
	$T$ 일 동안 각 날의 보급계획에 대해 해당하는 보급계획이 가능한지 불가능 한지 판단하고, 가능하다면 해당 보급계획을 진행할 수 있는 가장 짧은 시간을 출력하여라.
	
	\InputFile
	
	표준 입력에서 다음 입력이 주어진다.
	
	\begin{itemize}
		\item 첫째 줄에는 네 정수 $N$, $M$, $T$, $L$이 공백으로 구분되어 주어진다. 이는 IOI 숲에 $N$개의 급식소와 $M$개의 도로가 있으며, JOI군은 $T$ 일 동안의 보급계획을 생각 중이고, 각 보급계획이 $L$개의 급식소 번호를 차례로 나열했다는 의미이다.
		\item 다음 $M$ 개의 줄의 $i$ 번째 ($1 \le i \le M$) 줄에는 공백으로 구분된 세 정수 $A_i$, $B_i$, $C_i$가 주어진다. 이는 왼쪽에서 $i$ 번째 도로가 $A_i$번 급식소와 $B_i$번 급식소를 양방향으로 연결하며, 이 도로를 지나는 데 양방향 모두 $C_i$ 시간이 걸린다는 것이다.
		\item 다음 $L$ 개의 줄의 $j$ 번째 ($1 \le j \le L$) 줄에는 정수 $X_j$가 주어진다. 이는 최초 보급계획이 $X_1$, $X_2$, $\cdots$, $X_L$임을 의미한다.
		\item 다음 $T$ 개의 줄의 $k$ 번째 ($1 \le k \le T$) 줄에는 공백으로 구분된 두 정수 $P_k$, $Q_k$가 주어진다. 이는 $k$ 번째 날 아침에 JOI군이 보급계획의 $P_k$ 번째 값을 $Q_k$로 바꾼다는 의미이다.
	\end{itemize}
	
	\OutputFile
	
	표준 출력으로 $T$ 개의 줄을 출력하여라. $k$ 번째 ($1 \le k \le T$) 줄은 $k$번째 보급계획이 불가능 하면 \texttt{-1}, 가능하면 보급계획을 실행하는 데 드는 최소시간이다.
	
	\Constraints
	
	\begin{itemize}
		\item $2 \le N \le 2\ 000$.
		\item $N-1 \le M \le 2\ 000$.
		\item $1 \le T \le 100\ 000$.
		\item $2 \le L \le 100\ 000$.
		\item $1 \le A_i < B_i \le N$ ($1 < i \le M$).
		\item $(A_i, B_i) \ne (A_j, B_j)$ ($1 \le i< j \le M$).
		\item 어떤 급식소에서도 다른 모든 급식소까지 한 개 이상의 도로를 통해 서로 오가는 것이 가능하다.
		\item $1 \le C_i \le 1\ 000\ 000\ 000$ ($1 \le i \le M$).
		\item $1 \le X_j \le N$ ($1 \le j \le L$).
		\item $1 \le P_k \le L$ ($1 \le k \le T$).
		\item $1 \le Q_k \le N$ ($1 \le k \le T$).
		\item 모든 $j$ ($1 \le j \le L-1$) 에 대해, $X_j \ne X_{j+1}$을 만족한다. 또한, 보급 계획이 변경된 이후에도 모든 $j$ ($1 \le j \le L-1$) 에 대해, $X_j \ne X_{j+1}$을 만족한다.
	\end{itemize}
	
	
	\SubtaskWithCost{1}{12}
	\begin{itemize}
		\item $N \le 10$.
		\item $M \le 10$.
		\item $T = 1$.
		\item $L \le 10$.
		\item $C_i \le 10$ ($1 \le i \le M$).
	\end{itemize}


	\SubtaskWithCost{2}{35}
	\begin{itemize}
		\item $N \le 500$.
		\item $M \le 500$.
		\item $T = 1$.
	\end{itemize}


	\SubtaskWithCost{3}{15}
	\begin{itemize}
		\item $T = 1$.
	\end{itemize}
	
	\SubtaskWithCost{4}{38}
	추가 제한조건이 없다.
	
	\Examples
	
	\begin{example}
		\exmp{
			3 3 1 3
			1 2 1
			2 3 1
			1 3 1
			1
			2
			3
			3 1
		}{%
			3
		}%
	\end{example}
	
	이 입력 예제에서는 최초 보급계획은 1, 2, 3이다. JOI군은 첫 번째 날 아침에 세 번째 값을 1로 바꾼다. 즉, 첫 번째 날의 보급계획은 1, 2, 1이다.
	
	처음에 JOI군은 1번 급식소에서 음식을 보급받을 것이다. 그리고, 1번째 도로를 사용하여, 2번 급식소로 갈 것이다. 그 후 2번 급식소에서 음식을 보급받을 것이다. 그리고, 2번째 도로를 사용하여, 3번 급식소로 갈 것이다. 마지막으로, 3번째 도로를 사용하여, 1번 급식소로 갈 것이다. 그 후 1번 급식소에서 음식을 보급받을 것이다. 이 방법으로 음식을 보급받는 데에는 3시간이 걸린다. 이것이 최소시간이므로 3을 출력한다.
	
	JOI군은 유턴을 할 수 없으므로 $1 \rightarrow 2 \rightarrow 1$과 같은 순서로 급식소를 방문하는 것은 불가능하다.
	
	\begin{example}
		\exmp{
			4 4 4 3
			1 2 1
			2 3 1
			1 3 1
			1 4 1
			4
			1
			3
			3 4
			1 2
			3 2
			2 4
		}{%
			5
			2
			3
			-1
		}%
	\end{example}

이 입력 예제에서는 첫 번째 날의 보급계획은 4, 1, 4이다. 처음에 JOI군은 4번 급식소에서 음식을 보급받을 것이다. 그리고, 4번째 도로를 사용하여, 1번 급식소로 갈 것이다. 그 후 1번 급식소에서 음식을 보급받을 것이다. 그 후, 1, 2, 3, 4번째 도로 순서로 사용하여 급식소를 $1 \rightarrow 2 \rightarrow 3 \rightarrow 1 \rightarrow 4$ 순서로 움직여, 4번 급식소에서 음식을 보급받을 것이다. 이것이 최소 시간이다.

네 번째 날의 보급계획은 2, 4, 2이다. 이 보급계획은 실행할 수 없기 때문에, -1을 출력한다.

		\begin{example}
		\exmp{
			5 6 1 5
			1 2 8
			1 3 8
			1 4 8
			2 5 2
			3 4 6
			4 5 6
			2
			5
			1
			5
			3
			5 2
		}{%
			38
		}%
	\end{example}
	
\end{problem}
