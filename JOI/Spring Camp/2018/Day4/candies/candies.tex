\begin{problem}{사탕}
	{standard input}{standard output}
	{5초}{512MB}{}
	
	책상 위에 $N$ 개의 사탕이 있다. 각 사탕은 \textbf{맛}이란 값이 붙어 있다. 왼쪽에서 $i$ 번째 ($1 \le i \le N$) 사탕의 맛은 $A_i$이다.
	
	JOI양은 $N$개의 사탕 중 몇 개를 먹기로 결심했다. JOI양은 자신이 먹는 사탕의 맛의 합을 최대화하고 싶다. 하지만 사탕을 평범하게 선택하는 것이 재미 없다고 생각한 JOI양은 연속된 두 개의 사탕을 고르면 안 된다는 규칙을 만들었다.
	
	JOI양은 얼마나 몇 개의 사탕을 먹을지 아직 결정하지 않았기 때문에, 각 $j$ ($1 \le j \le \left\lceil \frac{N}{2} \right\rceil$)에 대해, $j$개의 사탕을 먹을 때 가능한 맛의 합의 최댓값을 구하고 싶어한다. 여기서 $\left\lceil x \right\rceil$은, $x$보다 크거나 같은 최소의 정수를 의미한다.
	
	\InputFile
	
	표준 입력에서 다음 입력이 주어진다.
	
	\begin{itemize}
		\item 첫째 줄에는 정수 $N$이 주어진다. 이는 책상 위에 $N$ 개의 사탕이 있다는 의미이다.
		\item 다음 $N$ 개의 줄의 $i$ 번째 ($1 \le i \le N$) 줄에는 정수 $A_i$가 주어진다. 이는 왼쪽에서 $i$ 번째 사탕의 맛이 $A_i$임을 의미한다.
	\end{itemize}
		
	\OutputFile
	
	표준 출력으로 $\left\lceil \frac{N}{2} \right\rceil$개의 줄을 출력하여라. $j$ 번째 ($1 \le j \le \left\lceil \frac{N}{2} \right\rceil$) 줄은 $j$ 개의 사탕을 먹을 때 가능한 맛의 합의 최댓값이다.
		
	\Constraints
	
	\begin{itemize}
		\item $1 \le N \le 200\ 000$.
		\item $1 \le A_i \le 1\ 000\ 000\ 000$ ($1 \le i \le N$).
	\end{itemize}
	
	
	\SubtaskWithCost{1}{8}
	\begin{itemize}
		\item $N \le 2\ 000$.
	\end{itemize}

	\SubtaskWithCost{2}{92}
	추가 제한조건이 없다.
	
	\Examples
	
	\begin{example}
		\exmp{
			5
			4
			5
			1
			7
			6
		}{%
			7
			12
			10
		}%
	\end{example}
	
	이 입력 예제에서는 5개의 사탕이 있고, 맛은 왼쪽에서부터 각각 3, 5, 1, 7, 6이다.
	
	JOI양은 다음과 같이 사탕을 먹는다.
	
	\begin{itemize}
		\item 한 개의 사탕을 먹을 때는 왼쪽에서부터 4번째 사탕을 먹는다. (맛은 7이다.)
		\item 두 개의 사탕을 먹을 때는 왼쪽에서부터 2번째, 4번째 사탕을 먹는다. (맛은 5, 7이다.)
		\item 세 개의 사탕을 먹을 때는 왼쪽에서부터 1번째, 3번째, 5번째 사탕을 먹는다. (맛은 3, 1, 6이다.)
	\end{itemize}

	연속된 두 개의 사탕을 모두 고르는 것은 불가능 하다. 예를 들면, 두 개의 사탕을 먹을 때, 왼쪽에서 부터 4번째, 5번째 사탕을 (맛은 7, 6이다) 먹는 것은 불가능하다는 것에 주의하여라.
	
	\begin{example}
		\exmp{
20
623239331
125587558
908010226
866053126
389255266
859393857
596640443
60521559
11284043
930138174
936349374
810093502
521142682
918991183
743833745
739411636
276010057
577098544
551216812
816623724
}{%
936349374
1855340557
2763350783
3622744640
4439368364
5243250666
5982662302
6605901633
7183000177
7309502029
}%
	\end{example}
	
	
\end{problem}

