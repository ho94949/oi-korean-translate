\begin{problem}{유괴2}
	{standard input}{standard output}
	{5 seconds}{512 megabytes}{}
	
	어느 화창한 날, 도시에서 유괴사건이 발생했다. 범인은 Anna와 Bruno고, 차를 통해 유괴현장에서 도망쳤다고 추정하고 있다. 차는 아직 발견되지 않았다. 경찰은 아직도 차의 행방을 쫓고 있다.
	
	유괴범은 $H$개의 동서방향 도로가 있고, $W$개의 남북방향 도로가 있는 격자 모양의 도시에서 차를 운전하고 있다. 두 교차점의 사이는 1km 이다.
	
	각 도로는 \textbf{혼잡도}라고 불리는 정수가 붙어있다. 북쪽에서 $i$ 번째 ($1 \le i \le H$) 동서방향 도로의 혼잡도는 $A_i$이고, 서쪽에서 $j$ 번째 ($1 \le j \le W$) 남북방향 도로의 혼잡도는 $B_j$ 이다. 이 $H+W$개의 값은 서로 다르다. 각 도로에 대해, 혼잡도는 어느 지점에 있든 동일하다.
	
	조사는 유괴범이 다음과 같은 방법으로 이동했다는 것을 알아냈다.
	
	\begin{itemize}
		\item 도시 밖이나, 도로 밖으로 나가지는 않았다.
		\item 처음 유괴범은 유괴한 교차점으로 부터 이동 가능한 방향중 어떤 한 방향을 택해 움직였다.
		\item 어떤 교차점에 도착했을 때, 현재 달리는 방향의 도로보다 교차하는 다른 도로의 혼잡도가 더 클 경우에, 그 교차점에서 회전한다. 회전 할 수 있는 방향이 둘인 경우에는, 어느쪽도 고를 가능성이 있다.
		\item 어떤 교차점에 도착했을 때, 현재 달리는 방향의 도로가 교차하는 다른 도로보다 혼잡도가 더 클 경우에, 계속 직진한다. 만약 도시의 경계에 도달하여 직진할 수 없을 경우에는 그 자리에서 멈춘다.
	\end{itemize}

	유괴사건이 발생했을 거라고 추정되는 $Q$개의 후보지가 있다. 이 $Q$개의 후보지는 서로 다르다. 조사팀의 사람을 정하기 위해서 경찰은 각 후보지에 대해서 유괴사건이 그 후보지에서 발생 했을 경우에 범죄자가 운전할 수 있는 최대 거리를 알고 싶다.
	
	각 $Q$개의 질의에 대해, 후보지에 대해서 범죄자가 운전할 수 있는 최대 거리를 구하여라. 
		
	\InputFile
	
	다음 정보가 표준 입력으로 주어진다.
	
	\begin{itemize}
		\item 첫째 줄에는 공백으로 구분된 두 정수 $H$, $W$, $Q$가 주어진다. 이는 도시가 동서방향 도로가 $H$개, 남북방향 도로가 $W$개 있으며, 범죄 후보지가 $Q$개 라는 의미이다.
		\item 둘째 줄에는 공백으로 구분된 $H$개의 정수 $A_1$, $A_2$, $\cdots$, $A_H$가 주어진다. 이는 북쪽에서 $i$ 번째 ($1 \le i \le H$) 동서방향 도로의 혼잡도가 $A_i$라는 의미이다.
		\item 셋째 줄에는 공백으로 구분된 $W$개의 정수 $B_1$, $B_2$, $\cdots$, $B_W$가 주어진다. 이는 북쪽에서 $j$ 번째 ($1 \le j \le W$) 동서방향 도로의 혼잡도가 $B_j$라는 의미이다.
		\item 다음 $Q$개의 줄의 $k$ 번째 ($1 \le k \le Q$) 줄에는 공백으로 구분된 두 정수 $S_k$, $T_k$가 주어진다. $k$ 번째 유괴사건 후보지가 북쪽에서 $S_k$ 번째 동서방향 도로와 서쪽에서 $T_k$ 번째 남북방향 도로의 교차로라는 의미이다.
	\end{itemize}

	
	\OutputFile
	
	표준 출력으로 $Q$ 개의 줄을 출력하여라. $k$ 번째 줄은 $k$ 번째 후보지에 대해서 범죄자가 운전할 수 있는 최대 거리를 (km단위로) 출력해야 한다.
	
	\Constraints
	
	\begin{itemize}
	
	\item $2 \le H \le 50\ 000$.	
	\item $2 \le W \le 50\ 000$.	
	\item $2 \le Q \le 100$.	
	\item $1 \le A_i \le 1\ 000\ 000\ 000$ ($1 \le i \le H$).	
	\item $1 \le B_j \le 1\ 000\ 000\ 000$ ($1 \le j \le W$).
	\item $H+W$개의 정수 $A_1$, $A_2$, $\cdots$, $A_H$, $B_1$, $B_2$, $\cdots$, $B_W$는 서로 다르다.
	\item $1 \le S_k \le H$ ($1 \le k \le Q$).
	\item $1 \le T_k \le W$ ($1 \le k \le Q$).
	\item $(S_k, T_k) \ne (S_l, T_l)$ ($1 \le k < l \le Q$).
	\end{itemize}
	
	
	\SubtaskWithCost{1}{13}
	\begin{itemize}
		\item $H \le 8$
		\item $W \le 8$
		\item $Q = 1$
	\end{itemize}


	\SubtaskWithCost{2}{10}
	\begin{itemize}
		\item $H \le 2\ 000$
		\item $W \le 2\ 000$
		\item $Q = 1$
	\end{itemize}

	\SubtaskWithCost{3}{17}
	\begin{itemize}
		\item $Q = 1$
	\end{itemize}

	\SubtaskWithCost{4}{4}
	\begin{itemize}
		\item $H \le 2\ 000$
		\item $W \le 2\ 000$
	\end{itemize}


	\SubtaskWithCost{5}{56}
	
	추가 제한조건이 없다.
	
	\Examples
		
	\begin{example}
	\exmp{
3 3 5
3 2 6
1 4 5
1 1
1 2
2 2
3 1
3 3
	}{%
4
5
4
4
2
	}%
	\end{example}

	예를 들어, 세번째 질의에 대해서 운전자가 이동한 거리는 다음 방법으로 최대가 된다.
	
	\begin{itemize}
		\item 북쪽에서 두 번째 동서방향 도로와 서쪽에서 두 번째 남북방향 도로의 교차로에서 동쪽으로 1km 움직였다.
		\item 북쪽에서 두 번째 동서방향 도로와 서쪽에서 세 번째 남북방향 도로의 교차로에서 남쪽 혹은 북쪽으로 움직일 수 있다. 남쪽을 골라서 1km 움직였다.
		\item 북쪽에서 세 번째 동서방향 도로와 서쪽에서 세 번째 남북방향 도로의 교차로에서 서쪽으로만 움직일 수 있다. 서쪽으로 1km 움직였다.
		\item 북쪽에서 세 번째 동서방향 도로와 서쪽에서 두 번째 남북방향 도로의 교차로에서 서쪽으로만 움직일 수 있다. 서쪽으로 1km 움직였다.
		\item 북쪽에서 세 번째 동서방향 도로와 서쪽에서 첫 번째 남북방향 도로의 교차로에서 더 이상 움직일 수 없다. 그 장소에서 멈췄다.
	\end{itemize}

	위와 같이 움직인 경우에, 이동한 거리는 4km 이다.

	\begin{example}
	\exmp{
		4 5 6
		30 10 40 20
		15 55 25 35 45
		1 3
		4 3
		2 2
		4 1
		2 5
		3 3
	}{%
		7
		6
		9
		4
		6
		9
	}%
	\end{example}
	
	
\end{problem}

