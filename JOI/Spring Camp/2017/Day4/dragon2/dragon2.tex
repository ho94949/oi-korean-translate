\begin{problem}{용2}
	{standard input}{standard output}
	{3 seconds}{256 megabytes}{}
	
	JOI평원에 사람들은 용과 함께 살고 있다.
	
	JOI평원은 광활한 좌표평면이고, 교차하는 X축과 Y축이 설정되어 있다. X좌표가 $x$, Y좌표가 $y$인 점을 $(x, y)$로 표시한다.
	
	JOI평원은 $N$마리의 용이 생활하고 있고, 1번부터 $N$번까지의 번호가 붙어있다. 또한, 용은 $M$종류의 종족이 있어서 1번부터 $M$번까지의 번호가 붙어있다. $i$번 ($1 \le i \le N$) 용은 평소에는 JOI평원의 $(A_i, B_i)$ 에 살고 있고, $C_i$번 종족이다. JOI평원에는 모든 종족의 용이 생활하고 있지 않을 수도 있다.
	
	JOI평원에, 사람이 사는 두 마을은 $(D_1, E_1)$과 $(D_2, E_2)$에 있다. 두 도시는 도로로 연결되어 있으며, 이는 두 점을 잇는 선분이다.
	
	점 $(A_1, B_1)$, $\cdots$, $(A_N, B_N)$과 $(D_1, E_1)$, $(D_2, E_2)$는 서로 다르며, 어떠한 세 점도 일직선 위에 있지 않다.
	
	가끔, 용의 종족 사이에서 대립이 벌어진다. $a$번 ($1 \le a \le M$) 종족이 $b$번 ($1 \le b \le M$, $a \ne b$) 종족에 대해 적의를 가지면 $a$번 종족의 모든 용이 $b$번 종족의 모든 용을 향해 화염구를 뿜는다. 화염구는 일직선으로 날아가고, 맞은 이후에도 계속 날아간다. 즉, 화염구의 궤적은 반직선이다.
	
	종족 사이의 대립이 일어났을 때, 도로와 화염구의 궤적이 교차하면 도로는 손상을 입을 것이다. 일어날 수 있는 $Q$개의 대립이 주어졌을 때, 각각의 대립에 대해서 도로와 교차하는 화염구의 갯수를 알고 싶다.
	
	\InputFile
	
	다음 정보가 표준 입력으로 주어진다.
	
	\begin{itemize}
		\item 첫째 줄에는 공백으로 구분된 두 정수 $N$, $M$이 주어진다. 이는 $N$마리의 용이 JOI 평원에 살고 있고, $M$종류의 종족이 존재한다는 의미이다.
		\item 다음 $N$개의 줄의 $i$ 번째 ($1 \le i \le N$) 줄에는 공백으로 구분된 세 정수 $A_i$, $B_i$, $C_i$가 주어진다. 이는 $i$번 ($1 \le i \le N$)용이 $(A_i, B_i)$에 살고 있고, $C_i$번 종족이라는 의미이다.
		\item 다음 줄에는 공백으로 구분된 네 정수 $D_1$, $E_1$, $D_2$, $E_2$가 존재한다. 이는 사람이 사는 두 마을이 $(D_1, E_1)$, $(D_2, E_2)$라는 의미이다.
		\item 다음 줄에는 정수 $Q$가 주어진다. 이는 일어날 수 있는 대립의 갯수가 $Q$개라는 의미이다.
		\item 다음 $Q$개의 줄의 $j$ 번째 ($1 \le j \le Q$) 줄에는 공백으로 구분된 두 정수 $F_j$, $G_j$가 주어진다. 이는 가능한 $j$ 번째 대립이 $F_j$번 종족이 $G_j$번 종족에게 적의를 품는다는 의미이다.
	\end{itemize}
	
	
	\OutputFile
	
	표준 출력으로 $Q$ 개의 줄을 출력하여라. $j$ 번째 ($1 \le j \le Q$) 줄은, $j$번째 대립이 일어났을 때, 도로와 교차하는 화염구의 갯수여야 한다.
	
	\Constraints
	
	\begin{itemize}
		\item $2 \le N \le 30\ 000$.
		\item $2 \le M \le N$.
		\item $-1\ 000\ 000\ 000 \le A_i \le 1\ 000\ 000\ 000$ ($1 \le i \le N$).
		\item $-1\ 000\ 000\ 000 \le B_i \le 1\ 000\ 000\ 000$ ($1 \le i \le N$).
		\item $1 \le C_i \le M$ ($1 \le i \le N$).
		\item $-1\ 000\ 000\ 000 \le D_1 \le 1\ 000\ 000\ 000$.
		\item $-1\ 000\ 000\ 000 \le E_1 \le 1\ 000\ 000\ 000$.
		\item $-1\ 000\ 000\ 000 \le D_2 \le 1\ 000\ 000\ 000$.
		\item $-1\ 000\ 000\ 000 \le E_2 \le 1\ 000\ 000\ 000$.
		\item N+2개의 점 $(A_1, B_1)$, $\cdots$, $(A_N, B_N)$, $(D_1, E_1)$, $(D_2, E_2)$는 서로 다르며, 어떠한 세 점도 일직선 위에 있지 않다.
		\item $1 \le Q \le 100\ 000$.
		\item $1 \le F_j \le M$ ($1 \le j \le Q$).
		\item $1 \le G_j \le M$ ($1 \le j \le Q$).
		\item $F_j \ne G_j$ ($1 \le j \le Q$).
		\item $(F_j, G_j) \ne (F_k, G_k)$ ($1 \le j < k \le Q$).
	\end{itemize}
	
	
	\SubtaskWithCost{1}{15}
	\begin{itemize}
		\item $N \le 3\ 000$
	\end{itemize}
	
	\SubtaskWithCost{2}{45}
	\begin{itemize}
		\item $Q \le 100$
	\end{itemize}
	
	\SubtaskWithCost{3}{40}
	
	추가 제한조건이 없다.
	
	\Examples
	
	\begin{example}
		\exmp{
			4 2
			0 1 1
			0 -1 1
			1 2 2
			-6 1 2
			-2 0 2 0
			2
			1 2
			2 1
		}{%
			1
			2
		}%
	\end{example}
	
	첫 번째 종족간의 대립에서, 다음을 만족한다.
	
	\begin{itemize}
		\item 1번 용이 3번 용에게 발사한 화염구는 도로와 교차하지 않는다.
		\item 1번 용이 4번 용에게 발사한 화염구는 도로와 교차하지 않는다.
		\item 2번 용이 3번 용에게 발사한 화염구는 도로와 교차한다.
		\item 1번 용이 4번 용에게 발사한 화염구는 도로와 교차하지 않는다.
	\end{itemize}

	그러므로, 하나의 화염구가 도로를 교차한다.
	
	두 번째 종족간의 대립에서, 다음을 만족한다.
	
	\begin{itemize}
		\item 3번 용이 1번 용에게 발사한 화염구는 도로와 교차한다.
		\item 3번 용이 2번 용에게 발사한 화염구는 도로와 교차한다.
		\item 4번 용이 1번 용에게 발사한 화염구는 도로와 교차하지 않는다.
		\item 4번 용이 2번 용에게 발사한 화염구는 도로와 교차하지 않는다.
	\end{itemize}

	그러므로, 두 개의 화염구가 도로를 교차한다.

	\begin{example}
	\exmp{
		3 2
		-1000000000 -1 1
		-999999998 -1 1
		0 0 2
		999999997 1 999999999 1
		1
		1 2
	}{%
		1
	}%
	\exmp{
		6 3
		2 -1 1
		1 0 1
		0 3 2
		2 4 2
		5 4 3
		3 9 3
		0 0 3 3
		6
		1 2
		1 3
		2 1
		2 3
		3 1
		3 2
	}{%
		4
		2
		4
		0
		2
		1
	}%
	\end{example}
	
	
\end{problem}

