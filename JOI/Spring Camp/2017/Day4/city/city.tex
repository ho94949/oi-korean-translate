\begin{problem}{도시}
	{}{}
	{3초}{256MB}{}
	
	JOI 왕국에는 다양한 도시가 있다. 도로 체계는 다음 조건을 만족한다:
	
	\begin{itemize}
		\item (조건 1) 도시는 0번부터 $N-1$번 까지의 번호가 붙어있다. 여기서, $N$은 JOI 왕국의 도시의 개수다.
		\item (조건 2) 도시는 $N-1$개의 도로로 연결되어 있다. 도로는 양방향으로 통행할 수 있다. 어떤 도시에서 다른 도시까지 몇개의 도로를 거치면 통행할 수 있다.
		\item (조건 3) 0번 도시로부터 다른 도시까지 최대 18개의 도로를 거치면 갈 수 있다.
	\end{itemize}

	어느 날 JOI 왕국에서 많은 사람이 0번 도시에서 다른 도시로 옮겨갔다. 많은 JOI 왕국의 국민이 서로 다른 두 개의 도시를 목적지로 하고 있었기 때문에 다음과 같은 질문을 했다. 2개의 서로 다른 두 도시 $X$, $Y$에 대해 (0), (1), (2) 중에 어느 것이 성립하는가?
	
	\begin{enumerate}
		\item [(0)] 0번 도시부터 도시 $X$ 까지 가는 도중에 도시 $Y$를 반드시 거친다.
		\item [(1)] 0번 도시부터 도시 $Y$ 까지 가는 도중에 도시 $X$를 반드시 거친다.
		\item [(2)] (0)과 (1) 모두 아니다.
	\end{enumerate}

	위의 조건에서 (0), (1), (2) 중에 정확히 하나가 성립한다. 단, $X$가 0번 도시인 경우에 $Y$에 상관없이 (1)이 성립하는 것으로 한다. 또한, $Y$가 0번 도시인 경우에 $X$에 상관없이 (0)이 성립하는 것으로 한다.
	
	JOI 왕국 뿐만 아니라 다른 나라에 대해서도 도로 체계의 (조건 1)\textasciitilde (조건 3)이 성립하는 것이 잘 알려져 있다. 그렇기 때문에 다른 나라에서도 쓸 수 있게 JOI 왕국에는 다음과 같은 2개의 기계를 만들려고 한다.
	
	\begin{itemize}
		\item (기계 1) 도시의 개수 $N$과 도로의 정보를 보고, 각각의 도시에 0 이상 $2^{60}-1$ 이하의 정수를 붙인다.
		\item (기계 2) 도시 $X$, $Y$에 대해 (기계 1)이 도시 $X$, $Y$에 부여한 번호를 보고 질문에 대해 대답한다.
	\end{itemize}
	
	큰 번호가 붙을 경우 이를 관리하는 것은 힘들다. 그렇기 때문에 가능한 한 적은 번호가 붙게 기계를 개발하려 한다.
	
	기계 2가 사용될 때 도시의 수 $N$이나 도로의 정보를 주지 않음에 유의하여라.
	
	\Specification
	
	당신은 \textit{같은 프로그래밍 언어로 작성된} 파일 두 개를 작성해야 한다.
	
	첫 번째 파일의 이름은 \texttt{Encoder.c} 혹은 \texttt{Encoder.cpp}이다. 이 파일은 기계 1을 구현한 파일이며, 다음 함수를 구현해야 한다. 이 파일은 \texttt{Encoder.h}를 include해야 한다.
	
	\begin{itemize}
		\item \texttt{void Encode(int N, int A[], int B[])}
		
		이 함수는 각 테스트 케이스마다 정확히 한 번 불린다.
		\begin{itemize}
			\item 인자 \texttt{N}은 도시의 개수 $N$을 의미한다.
			\item 인자 \texttt{A[]}, \texttt{B[]}는 길이 $N-1$의 배열이고, 도로의 정보를 의미한다. 원소 \texttt{A[i]}, \texttt{B[i]} ($0 \le $\texttt{i}$\le N-2$)는 \texttt{A[i]}번 도시와 \texttt{B[i]}번 도시를 직접 연결하는 도로가 있다는 것을 의미한다.
		\end{itemize}
		
		당신의 프로그램은 다음의 함수를 호출해야 한다.
		\begin{itemize}
			\item \texttt{void Code(int city, long long code)}
			
			이 함수는 도시에 정수를 붙이는 것을 의미한다.
			
			\begin{itemize}
				\item 인자 \texttt{city}는 정수를 붙일 도시의 번호를 의미한다. \texttt{city}는 0 이상 $N-1$ 이하이다. 만약에 이 범위를 벗어나서 함수를 호출 한 경우 \textbf{오답 [1]}이 된다. 같은 \texttt{city}를 인자로 하여 함수를 두 번 이상 호출한 경우 \textbf{오답 [2]}이 된다.
				\item 인자 \texttt{code}는 \texttt{city}번 도시에 붙일 정수이다. \texttt{code}는 값은 0 이상 $2^{60}-1$이어야 한다. 만약에 이 범위를 벗어나서 함수를 호출한 경우 \textbf{오답 [3]}이 된다.
			\end{itemize}
			
		\end{itemize}
		
		함수 \texttt{Code}는 프로그램에서 정확히 $N$번 호출되어야 한다. \texttt{Encode} 함수가 종료되었을 때, \texttt{Code}이 호출된 횟수가 $N$과 다르면 \textbf{오답 [4]}이 된다.
		
		만약 \texttt{Encode}가 함수를 올바르지 않게 호출될 경우 프로그램이 종료된다.
		
	\end{itemize}
	
	두 번째 파일의 이름은 \texttt{Device.c} 혹은 \texttt{Device.cpp}이다. 이 파일은 기계 2를 구현한 파일이며, 다음 함수를 구현해야 한다. 이 파일은 \texttt{Device.h}를 include해야 한다.
	
	
	\begin{itemize}
		\item \texttt{void InitDevice()}
		
		이 함수는 기계 2의 초기화에 대응한다. 다음 \texttt{Answer} 함수가 불리기 전에 \texttt{InitDevice}는 정확히 한 번만 불린다.
		
		\item \texttt{int Answer(long long S, long long T)}
		
		이 함수는 각 질문에 대응한다. 각 질문에 대응하여 \texttt{Answer}가 한 번 불린다.

		\begin{itemize}
			\item 인자 \texttt{S}, \texttt{T}는 두 개의 서로 다른 도시 $X$, $Y$에 붙은 정수이다.
			\item 함수 \texttt{Answer}는 질문에 답하기 위해, 다음의 조건을 만족하는 값을 반환해야 한다.
			
			\begin{itemize}
				\item 0번 도시부터 도시 $X$ 까지 가는 도중에 도시 $Y$를 반드시 거치는 경우, \texttt{0}을 반환한다.
				\item 0번 도시부터 도시 $Y$ 까지 가는 도중에 도시 $X$를 반드시 거치는 경우, \texttt{1}을 반환한다.
				\item 위 두 경우 모두 아닌 경우 \texttt{2}를 반환한다.
			\end{itemize}
		
			즉, \texttt{Answer}의 반환값은 0 이상 2 이하의 정수여야 한다. 이 범위 밖의 수를 반환한 경우 \textbf{오답 [5]}이 된다. 또한, 범위 내의 정수여도, 위 조건을 만족하지 않는 값을 반환한 경우 \textbf{오답 [6]}이 된다.
		
		\end{itemize}
	
	\end{itemize}
	
	채점은 다음과 같은 방식으로 진행된다. 만약 프로그램이 오답으로 판단된 경우, 즉시 채점은 종료된다.
	
	\begin{enumerate}
		\item 함수 \texttt{Encode}를 1회 호출한다.
		\item 함수 \texttt{InitDevice}를 1회 호출한다.
		\item 각 테스트케이스에 대해, 기계 2에게 $Q$개의 질문을 준다. $j$ 번째 ($1 \le j \le Q$) 질문에 대해, 함수 \texttt{Answer}는 인자 \texttt{S}에 $S_j$를, 인자 \texttt{T}에 $T_j$를 넣어서 호출되고, $S_j$와 $T_j$는 함수 \texttt{Encode}가 $X_j$번과 $Y_j$번 도시에 설정한 값이다.
		\item 당신의 프로그램은 정답이 된다.
	\end{enumerate}
	
	\Notes
	
	\begin{itemize}
		\item 실행 시간과 메모리 사용량은 채점 방식의 1, 2, 3에서 계산된다.
		\item 당신의 프로그램은 채점 방식 1.의 \texttt{Encode}, 채점 방식 2.의 \texttt{InitDevice} 혹은 에서 채점 방식 3.의 \texttt{Answer}에서 오답으로 판단되면 안된다. 당신의 프로그램은 런타임 에러 없이 실행되어야 한다.
		\item 당신의 프로그램은 내부에서 사용할 목적으로 함수나 전역변수를 사용할 수 있다. 제출한 프로그램은 그레이더와 함께 컴파일 되어 하나의 실행파일이 된다. 모든 전역변수나 내부 함수는 다른 파일과의 충돌을 피하기 위해 \texttt{static}으로 선언되어야 한다. 기계 1과 기계 2는 2개의 별개의 프로세스로 실행되기 때문에 채점될 때 전역변수를 공유하지 않는다.
		\item 당신의 프로그램은 표준 입출력을 사용해서는 안 된다. 당신의 프로그램은 어떠한 방법으로도 다른 파일에 접근해서는 안 된다. 
	\end{itemize}
	
	당신은 대회 홈페이지의 아카이브에서 프로그램을 테스트하기 위한 목적의 샘플 그레이더를 받을 수 있다. 아카이브는 당신의 프로그램의 예제 소스 또한 첨부되어 있다.
	샘플 그레이더는 파일 \texttt{grader.c} 혹은 \texttt{grader.cpp}이다. 당신의 프로그램이 \texttt{Encoder.c}와 \texttt{Device.c} 혹은, \texttt{Encoder.cpp}와 \texttt{Device.cpp} 인 경우 다음 커맨드로 컴파일할 수 있다.
	
	\begin{itemize}
		\item C
		\texttt{g++ -std=c11 -O2 -o grader grader.c Encoder.c Device.c -lm}
		\item C++
		\texttt{g++ -std=c++14 -O2 -o grader grader.cpp Encoder.cpp Device.cpp }
	\end{itemize}
	
	컴파일이 성공적이면, 파일 \texttt{grader}가 생성된다.
	
	실제 그레이더와 샘플 그레이더는 다름에 주의하여라. 샘플 그레이더는 하나의 프로세스에서 실행되며, 입력을 표준 입력으로 부터받고, 출력을 표준 출력에 출력한다.
	
	\InputFile
	
	샘플 그레이더는 다음 형식으로 표준 입력으로 부터 데이터를 입력받는다.
	
	\begin{itemize}
		\item 첫째 줄에 두 개의 정수 $N$, $Q$가 공백으로 구분되어 주어진다. 이는 $N$ 개의 도시가 있고, 질문이 $Q$ 개라는 의미이다.
		\item 다음 $N-1$ 개의 줄의 $i+1$ 번째 ($0 \le i \le N-2$) 줄에는, 두 개의 정수 $A_i$와 $B_i$가 공백으로 구분되어 주어진다. 이는 $A_i$번 도시와 $B_i$번 도시가 도로로 직접 연결되어 있다는 의미이다.
		\item 다음 $Q$개의 줄의 $j$ 번째 ($1 \le j \le Q$) 줄에는, 3개의 정수 $X_j$, $Y_j$, $E_j$가 공백으로 구분되어 주어진다. 이는 $j$번째 질문에 대해, $X=X_j$이고 $Y=Y_j$라는 것을 의미하고, 당신의 프로그램이 질문에 대해 $E_j$ 이외의 답을 반환한 경우에는 샘플 그레이더가 오답으로 판단한다는 의미이다.
	\end{itemize}
	
	
	\OutputFile
	
	프로그램이 정상적으로 종료되었다면, 샘플 그레이더는 다음과 같은 정보를 표준 출력에 출력한다. (따옴표는 출력하지 않는다.)
	
	\begin{itemize}
		\item 정답인 경우, 도시에 붙은 정수의 최댓값을 ``\texttt{Accepted max\_code=123456}"과 같은 형식으로 출력한다. 
		\item 오답으로 판단 된 경우, 오답의 종류를 ``\texttt{Wrong Answer [1]}"과 같은 형식으로 출력한다.
	\end{itemize}
	
	프로그램이 다양한 오답의 종류에 속해 있으면 샘플 그레이더는 그중 하나만 출력할 것이다.
	
	\Constraints
	
	$N$, $Q$, $A_i$, $B_i$, $X_j$, $Y_j$의 의미에 대해서는 입력 형식을 참고하여라.
	
	\begin{itemize}
		\item $2 \le N \le 250\ 000$.
		\item $1 \le Q \le 250\ 000$.
		\item $0 \le A_i \le N-1$ ($0 \le i \le N-2$).
		\item $0 \le B_i \le N-1$ ($0 \le i \le N-2$).
		\item $A_i \ne B_i$. ($0 \le i \le N-2$)
		\item 도시 0부터 어떤 다른 도시에 대해서도, 18개 이하의 도로를 사용하여 오가는 것이 가능하다.
		\item $0 \le X_j \le N-1$ ($1 \le j \le Q$).
		\item $0 \le Y_j \le N-1$ ($1 \le j \le Q$).
		\item $X_j \ne Y_j$ ($1 \le j \le Q$).
	\end{itemize}
	
	
	\SubtaskWithCost{1}{8}
	\begin{itemize}
		\item $N \le 10$.
	\end{itemize}
	
	
	\SubtaskWithCost{2}{92}
	
	추가 제한조건이 없다. 이 서브태스크에 대해서는 다음과 같이 점수가 정해진다.
	
	\begin{itemize}
		\item 이 서브태스크의 모든 테스트케이스에 대해 도시에 할당된 정수의 최댓값을 $L$이라고 하자.
		\item 이 때, 이 서브태스크의 점수는
		\begin{itemize}
			\item $2^{38} \le L$인 경우, 0점.
			\item $2^{36} \le L \le 2^{38}-1$인 경우, 10점.
			\item $2^{35} \le L \le 2^{36}-1$인 경우, 14점.
			\item $2^{34} \le L \le 2^{35}-1$인 경우, 22점.
			\item $2^{28} \le L \le 2^{34}-1$인 경우, $\lfloor372-10\log_2(L+1)\rfloor$점 ($\lfloor x \rfloor$는 $x$를 넘지 않는 최대의 정수)
			\item $L \le 2^{28}-1$ 인 경우, 92점
		\end{itemize}
	\end{itemize}
	
	채점 시스템에 대해서 $2^{38} \le L$인 경우에, 이 서브태스크의 정보가 ``정답: 0점"과 같이 표시되는 것이 아니라, ``오답" 으로 표시됨에 주의하여라.
	
	
	
	\Examples
	
	예제 입력과 이에 해당하는 함수 호출을 보여준다.
	
	
	
\begin{tabular}{|l|l|l|}
	\hline
	\multirow{2}{*}{예제 입력}                                                                                                              & \multicolumn{2}{l|}{예제 함수 호출} \\ \cline{2-3} 
	& 기계 1         & 기계 2           \\ \hline
	\multirow{13}{*}{\begin{tabular}[c]{@{}l@{}}\texttt{6 5}\\ \texttt{4 1}\\ \texttt{0 3}\\ \texttt{4 5}\\ \texttt{3 2}\\ \texttt{3 4}\\ \texttt{2 4 2}\\ \texttt{1 0 0}\\ \texttt{5 1 2}\\ \texttt{5 3 0}\\ \texttt{4 1 1}\end{tabular}} & \texttt{Encode(...)}  &                \\ \cline{2-3} 
	& \texttt{Code(0,0)}    &                \\ \cline{2-3} 
	& \texttt{Code(2,4)}    &                \\ \cline{2-3} 
	& \texttt{Code(4,16)}   &                \\ \cline{2-3} 
	& \texttt{Code(1,1)}    &                \\ \cline{2-3} 
	& \texttt{Code(3,9)}    &                \\ \cline{2-3} 
	& \texttt{Code(5,25)}   &                \\ \cline{2-3} 
	&              & \texttt{InitDevice()}   \\ \cline{2-3} 
	&              & \texttt{Answer(4,16)}   \\ \cline{2-3} 
	&              & \texttt{Answer(1,0)}    \\ \cline{2-3} 
	&              & \texttt{Answer(25,1)}   \\ \cline{2-3} 
	&              & \texttt{Answer(25,9)}  \\ \cline{2-3} 
	&              & \texttt{Answer(16,1)}   \\ \hline
\end{tabular}
	
	여기서 \texttt{Encode(...)} 호출의 인자들은 다음과 같다.
	
	\begin{tabular}{|l|l|}
		\hline
		인자 & \texttt{Encode(...)}  \\ \hline
		\texttt{N}  & 6          \\ \hline
		\texttt{A}  & \{4, 0, 4, 3, 3\}       \\ \hline
		\texttt{B}  & \{1, 3, 5, 2, 4\}            \\ \hline
	\end{tabular}
	
\end{problem}

