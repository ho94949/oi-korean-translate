\begin{problem}{고장난 기기}
	{}{}
	{2 seconds}{256 megabytes}{}
	
	고고학자 Anna와 Bruno는 이란의 유적을 조사하고 있다.

	두명이 역할을 분할하여 Anna는 유적을 발견하고 유물을 발견하며, Bruno는 베이스캠프에서 결과를 분석한다.
	
	조사는 총 $Q(=\ 1\ 000)$일 동안 계획이 되어있다. 매일, Anna는 Bruno에게 통신 기기를 사용하여 결과를 보낸다. 각 통신기기의 결과는 정수 $X$로 표시된다.
	
	안나는 통신 기기를 하루에 한번만 사용할 수 있다. 통신기기는 0 혹은 1로 되어있는 길이 $N(=\ 150)$의 수열을 보낼 수 있다.
	
	하지만 이 기계가 고장나 버려서, 보내는 길이 $N$의 수열 중 기능이 제대로 작동하지 않는 위치가 생겨버렸다. 기능하지 않는 위치에는, 어떤 값을 설정해도 0이 보내지게 된다. Anna가 수열을 보낼 때, 고장난 위치가 어딘지 알 수 있다. 하지만, Bruno는 그 위치를 모른다. 고장난 위치가 몇 개인지, 어딘지는 매일 바뀐다.
	
	이 조사가 지연되면 문제가 생길 수 있다. Anna와 Bruno는 이란에서 열리는 국제 프로그래밍 대회의 후보인 당신에게 조사 결과를 보내는 프로그램을 작성해달라고 요청했다.

	당신은 Anna와 Bruno 사이에서 통신을 하는 두 개의 프로그램을 작성하여야 한다.
	
	\begin{itemize}
		\item 수열의 길이 $N$과 보내야 할 정수 $X$와, 고장난 위치의 갯수 $K$와, 고장난 위치 $P$가 주어졌을 때, 첫번째 프로그램은 Anna가 보낼 수열 $S$를 정한다.
		\item Bruno가 수열 $A$를 받았을 때, 두 번째 프로그램은 정수 $X$를 복구한다.
	\end{itemize}

	통신 장치가 고장난 위치가 아닌 곳에서는, 수열 $S$와 수열 $A$는 같은 값을 가진다. 통신 장치가 고장난 곳에서는, 수열 $A$는 수열 $S$의 값과 관계 없이 0이다.


	\Specification
	
	당신은 \textit{같은 프로그래밍 언어로 작성된} 파일 두개를 작성해야 한다.
	
	첫 번째 파일의 이름은 \texttt{Anna.c} 혹은 \texttt{Anna.cpp}이다. 이 파일은 Anna가 보낼 수열을 정하는 역할을 하며, 다음 함수를 구현해야 한다. 이 파일은 \texttt{Annalib.h}를 include해야 한다.
	
	\begin{itemize}
		\item \texttt{void Anna(int N, long long X, int K, int P[])}
		
		이 함수는 각 테스트 케이스 마다 정확히 $Q = 1\ 000$ 번 불린다.
		\begin{itemize}
			\item 인자 \texttt{N}은 보낼 수열의 길이를 나타낸다.
			\item 인자 \texttt{X}는 보낼 숫자를 나타낸다.
			\item 인자 \texttt{K}는 부서진 위치의 갯수를 나타낸다.
			\item 인자 \texttt{P[]}는 부서진 위치를 나타내는 길이 $K$의 수열이다.
		\end{itemize}
		
		함수 \texttt{Anna}는 다음 함수를 호출해야 한다.
		\begin{itemize}
			\item \texttt{void Set(int pos, int bit)}
			
			이 함수는, 통신 기기로 보낼 수열 $S$를 설정한다.
			
			\begin{itemize}
				\item 인자 \texttt{pos}는 수열의 값을 설정할 위치이다. \texttt{pos}는 0 이상 $N-1$ 이하이다. \textit{위치가 0부터 시작함에 유의하여라.} 만약에 이 범위를 벗어나서 함수를 호출 한 경우, \textbf{오답 [1]}이 된다. 같은 \texttt{pos}를 인자로 하여 함수를 두 번 이상 호출 한 경우, \textbf{오답 [2]}이 된다.
				
				\item 인자 \texttt{bit}는 \texttt{pos}번째에 설정할 값이다. \texttt{bit}의 값은 0 혹은 1이어야 한다. 다른 인자로 함수를 호출 한 경우 \textbf{오답 [3]}이 된다.
			\end{itemize}
			

		\end{itemize}
	
		함수 \texttt{Set}은 함수 \texttt{Anna} 안에서 정확히 $N$번 호출되어야 한다. \texttt{Anna} 함수가 종료되었을 때, \texttt{Set}이 호출 된 횟수가 $N$과 다르면, \textbf{오답 [4]}이 된다.
		
		만약 \texttt{Anna}가 함수를 올바르지 않게 호출 될 경우 프로그램이 종료된다.
		
	\end{itemize}

	두 번째 파일의 이름은 \texttt{Bruno.c} 혹은 \texttt{Bruno.cpp}이다. 이 파일은 탐사 결과를 복구하는 역할을 하며, 다음 함수를 구현해야 한다. 이 파일은 \texttt{Brunolib.h}를 include해야 한다.
	
	
	\begin{itemize}
		\item \texttt{long long Bruno(int N, int A[])}
		
		이 함수는 각 테스트 케이스 마다 정확히 $Q = 1\ 000$ 번 불린다.
		\begin{itemize}
			\item 인자 \texttt{N}은 Bruno가 받은 수열의 길이를 나타낸다.
			\item 인자 \texttt{A[]}는 Bruno가 받은 길이 $N$의 수열이다.			\item 함수 Bruno는 $X$를 찾아서 반환해야 한다.
		\end{itemize}
	\end{itemize}

	채점은 다음과 같은 방식으로 진행된다. 만약 프로그램이 오답으로 판단된 경우, 즉시 채점은 종료된다.
	
	\begin{enumerate}
		\item \texttt{cnt=0}으로 설정한다.
		\item 함수 \texttt{Anna}를 1회 호출한다.
		\item 함수 \texttt{Anna}에 호출 된 수열을 $S$라고 하자. $S$중 $P$에 포함 된 위치를 0으로 바꾸는 작업을 \texttt{A}에 한 후, 함수 \texttt{Bruno}를 1회 호출한다.
		\item \texttt{cnt=cnt+1}로 설정한다. \texttt{cnt<}$Q$이면 2.로 돌아간다. \texttt{cnt=}$Q$이면, 5.로 간다.
		\item 채점을 한다.
	\end{enumerate}
	
	\Notes
	
	\begin{itemize}
		\item 실행 시간과 메모리 사용량은 채점 방식의 1, 2, 3, 4에서 계산된다.
		\item 당신의 프로그램은 채점 방식 2.의 \texttt{Anna} 혹은 채점 방식 3.의 \texttt{Bruno}에서 오답으로 판단되면 안된다. 당신의 프로그램은 런타임 에러 없이 실행되어야 한다.
		\item 당신의 프로그램은 내부에서 사용할 목적으로 함수나 전역변수를 사용할 수 있다. 제출한 프로그램은 그레이더와 함께 컴파일 되어 하나의 실행파일이 된다. 모든 전역변수나 내부 함수는 다른 파일과의 충돌을 피하기 위해 \texttt{static}으로 선언되어야 한다. Anna와 Bruno는 2개의 별개의 프로세스로 실행되기 때문에 채점 될 때 전역변수를 공유하지 않는다.
		\item 각 프로세스에서, \texttt{Anna}와 \texttt{Bruno}는 각각 $Q = 1\ 000$번 호출된다. \textit{사용할 변수의 초기화는 적절히 진행되어야 한다.}
		\item 당신의 프로그램은 표준 입출력을 사용해서는 안된다. 당신의 프로그램은 어떠한 방법으로도 다른 파일에 접근해서는 안된다. 
	\end{itemize}
	
	당신은 대회 홈페이지의 아카이브에서 프로그램을 테스트 하기 위한 목적의 샘플 그레이더를 받을 수 있다. 아카이브는 당신의 프로그램의 예제 소스 또한 첨부되어 있다.
	샘플 그레이더는 파일 \texttt{grader.c} 혹은 \texttt{grader.cpp}이다. 당신의 프로그램이 \texttt{Anna.c}와 \texttt{Bruno.c} 혹은, \texttt{Anna.cpp}와 \texttt{Bruno.cpp} 인 경우 다음 커맨드로 컴파일할 수 있다.
	
	\begin{itemize}
		\item C
		\texttt{g++ -std=c11 -O2 -o grader grader.c Anna.c Bruno.c -lm}
		\item C++
		\texttt{g++ -std=c++14 -O2 -o grader grader.cpp Anna.cpp Bruno.cpp }
	\end{itemize}
	
	컴파일이 성공적이면, 파일 \texttt{grader}가 생성된다.
	
	실제 그레이더와 샘플 그레이더는 다름에 주의하여라. 샘플 그레이더는 하나의 프로세스에서 실행 되며, 입력을 표준 입력으로 부터 받고, 출력을 표준 출력에 출력한다.
	
	\InputFile
	
	샘플 그레이더는 다음 형식으로 표준 입력으로 부터 데이터를 입력받는다.
	
	\begin{itemize}
		\item 첫째 줄에 정수 $Q$가 주어진다.
		\item 그리고, $Q$개의 쿼리의 정보가 주어진다.
		\item 각 쿼리의 정보는 다음과 같은 두 줄로 되어있다.
		\begin{itemize}
			\item 첫 번째 줄은 공백으로 구분된 세 정수 $N$, $X$, $K$가 주어진다. 이는 보낼 수열의 길이가 $N$이고, Anna가 보낼 정수가 $X$고, 부서진 위치가 $K$개라는 뜻이다.
			\item 두 번째 줄은 공백으로 구분된 $K$개의 정수 $P_0,P_1, \cdots, P_{K-1}$ 이 주어진다. 이는, 각 $i$ ($0 \le i \le K-1$)에 대해, 수열의 $P_i$번째 위치가 고장났다는 것이다.
		\end{itemize}
	\end{itemize}
	
	
	\OutputFile
	
	프로그램이 정상적으로 종료되었다면, 샘플 그레이더는 다음과 같은 정보를 표준 출력에 출력한다. (따옴표는 출력하지 않는다.)
	
	\begin{itemize}
		\item 오답으로 판단 된 경우, 오답의 종류를 ``\texttt{Wrong Answer [1]}"과 같은 형식으로 출력하고, 프로그램이 종료된다.
		\item 만약 모든 Anna의 호출이 오답으로 판단되지 않을 경우,  ``\texttt{Accepted}"와 $L^*$을 표준 출력으로 출력한다. $L^*$의 값은 배점 항목을 참고하여라. 
	\end{itemize}
	
	프로그램이 다양한 오답의 종류에 속해 있을 경우, 샘플 그레이더는 그 중 하나만 출력 할 것이다.
	
	\Constraints
	\begin{itemize}
		\item $Q = 1000$.
		\item $N = 150$.
		\item $0 \le X \le 1\ 000\ 000\ 000\ 000\ 000\ 000$.
		\item $1 \le K \le 40$.
		\item $0 \le P_i \le N-1$ ($0 \le i \le K-1$).
		\item $P_i < P_{i+1}$ ($0 \le i \le K-2$).
	\end{itemize}
	
	\Scoring
	
	\begin{itemize}
		\item $L^*$를 이 문제의 모든 테스트 케이스의 최솟값이라고 하자.
		\begin{itemize}
			\item $K \le L$인 모든 쿼리에 대해서, Bruno가 정답을 말한 최대 정수 $L \le 40$.
		\end{itemize}
		\item 이 문제의 점수는 다음과 같이 계산된다.
		\begin{itemize}
			\item $L^* = 0$인 경우, 점수는 0점이다.
			\item $1 \le L^* \le 14$인 경우, 점수는 8점이다.
			\item $15 \le L^* \le 37$인 경우, 점수는 $(L^* - 15)$ × $2 + 41$점 이다.
			\item $38 \le L^* \le 40$인 경우, 점수는 $(L^* - 38)$ × $5 + 90$점 이다.
		\end{itemize}
	\end{itemize}
	
	

	\Examples
	
	예제 입력과 이에 해당하는 함수 호출을 보여준다. 이 예제는 $Q=2, N=3$이기 때문에 문제의 제한을 만족하지 않음에 유의하여라.
	
	
	
\begin{tabular}{|l|l|l|l|l|}
	\hline
	\multirow{2}{*}{예제 입력}                                                                 & \multicolumn{4}{l|}{예제 함수 호출}       \\ \cline{2-5} 
	& 호출         & 반환값  & 호출       & 반환값  \\ \hline
	\multirow{20}{*}{\begin{tabular}[c]{@{}l@{}}\texttt{2}\\ \texttt{3 14 1}\\ \texttt{2}\\ \texttt{3 9 2}\\ \texttt{0 1}\end{tabular}} & \texttt{Anna(...)}  &      &          &      \\ \cline{2-5} 
	&            &      & \texttt{Set(0,0)} &      \\ \cline{2-5} 
	&            &      &          & (없음) \\ \cline{2-5} 
	&            &      & \texttt{Set(1,0)} &      \\ \cline{2-5} 
	&            &      &          & (없음) \\ \cline{2-5} 
	&            &      & \texttt{Set(2,1)} &      \\ \cline{2-5} 
	&            &      &          & (없음) \\ \cline{2-5} 
	&            & (없음) &          &      \\ \cline{2-5} 
	& \texttt{Bruno(...)} &      &          &      \\ \cline{2-5} 
	&            & \texttt{14}   &          &      \\ \cline{2-5} 
	& \texttt{Anna(...)}  &      &          &      \\ \cline{2-5} 
	&            &      & \texttt{Set(0,0)} &      \\ \cline{2-5} 
	&            &      &          & (없음) \\ \cline{2-5} 
	&            &      & \texttt{Set(1,1)} &      \\ \cline{2-5} 
	&            &      &          & (없음) \\ \cline{2-5} 
	&            &      & \texttt{Set(2,1)} &      \\ \cline{2-5} 
	&            &      &          & (없음) \\ \cline{2-5} 
	&            & (없음) &          &      \\ \cline{2-5} 
	& \texttt{Bruno(...)} &      &          &      \\ \cline{2-5} 
	&            & \texttt{9}    &          &      \\ \hline
\end{tabular}

여기서 \texttt{Anna(...)}, \texttt{Bruno(...)}, \texttt{Anna(...)}, \texttt{Bruno(...)} 호출의 인자들은 다음과 같다.

\begin{tabular}{|l|l|l|l|l|}
	\hline
	인자 & \texttt{Anna(...)} &  \texttt{Bruno(...)} & \texttt{Anna(...)}  & \texttt{Bruno(...)} \\ \hline
	\texttt{N}  & 3     & 3           & 3        & 3           \\ \hline
	\texttt{K}  & 14    &             & 9        &             \\ \hline
	\texttt{X}  & 1     &             & 2        &             \\ \hline
	\texttt{P}  & \{2\} &             & \{0, 1\} &             \\ \hline
	\texttt{A}  &       & \{0, 0, 0\} &          & \{0, 0, 1\} \\ \hline
\end{tabular}

\end{problem}

