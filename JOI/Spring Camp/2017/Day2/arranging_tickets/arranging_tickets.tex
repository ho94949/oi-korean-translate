\begin{problem}{티켓 정리}
	{standard input}{standard output}
	{4 seconds}{256 megabytes}{}
	
	JOI 공화국에는 1번 부터 $N$번 까지 번호가 붙은 $N$개의 기차역이 있다. 기차역은 원형 철도에 시계방향으로 위치해있다.
	
	철도를 이용하기 위한 티켓은 $N$ 종류가 있고, 각각 1번부터 $N$번까지 번호가 붙어 있다. $i$번 ($1 \le i \le N-1$) 티켓을 하나 사용하면, $i$번 기차역에서 $i+1$번 기차역으로, 혹은 $i+1$번 기차역에서 $i$번 기차역으로 사람 한 명이 이동할 수 있다. $N$번 티켓을 하나 사용하면, $1$번 기차역에서 $N$번 기차역으로, 혹은 $N$번 기차역에서 1번 기차역으로 사람 한명이 이동할 수 있다. 이 티켓은, $N$ 종류의 티켓이 정확히 한 장씩 총 $N$장이 들어있는 묶음으로 판매되고 있다.
	
	당신은 JOI 공화국의 여행회사에서 일하고 있다. 당신은 고객들에게 티켓을 나눠주어야 한다.
	
	오늘 티켓을 나눠달라는 $M$개의 요청이 있었다. $i$번째 요청은 $C_i$명의 사람이 $A_i$번 기차역에서 $B_i$번 기차역으로 이동하고 싶다는 요청이었다. $C_i$명의 사람들이 모두 같은 경로로 이동 할 필요는 없다.
	
	모든 요청을 처리하기 위해서 사야 할 묶음의 최소 갯수를 알고 싶다.

	\InputFile
	
	다음 정보가 표준 입력으로 주어진다.
	
	\begin{itemize}
		\item 첫째 줄에는 공백으로 구분된 두 정수 $N$, $M$이 주어진다. JOI 공화국에 $N$개의 기차역이 있으며, $M$개의 요청을 오늘 받았다는 것이다.
		\item 다음 $N$개의 줄의 $i$ 번째 ($1 \le i \le N$) 줄에는 공백으로 구분된 세 정수 $A_i$, $B_i$, $C_i$가 주어진다. 이는 $i$번째 요청이 $C_i$명의 사람이 $A_i$번 기차역에서 $B_i$번 기차역으로 이동하고 싶다는 요청이라는 것을 의미한다.
	\end{itemize}

	
	\OutputFile
	
	표준 출력으로 한 개의 줄을 출력하여라. 출력은 모든 요청을 처리하기 위해서 사야 할 묶음의 최소 갯수이다.
	
	\Constraints
	
	\begin{itemize}
	
	\item $3 \le N \le 200\ 000$.
	\item $1 \le M \le 100\ 000$.
	\item $1 \le A_i \le N$ ($1 \le i \le M$).
	\item $1 \le B_i \le N$ ($1 \le i \le M$).
	\item $1 \le C_i \le 1\ 000\ 000\ 000$ ($1 \le i \le M$).
	\item $A_i \ne B_i$ ($1 \le i \le M$)
	\end{itemize}
	
	
	\SubtaskWithCost{1}{10}
	\begin{itemize}
		\item $N \le 20$
		\item $M \le 20$
		\item $C_i = 1$ ($1 \le i \le M$)
	\end{itemize}


	\SubtaskWithCost{2}{35}
	\begin{itemize}
		\item $N \le 300$
		\item $M \le 300$
		\item $C_i = 1$ ($1 \le i \le M$)
	\end{itemize}

	\SubtaskWithCost{3}{20}
	\begin{itemize}
		\item $N \le 3\ 000$
		\item $M \le 3\ 000$
		\item $C_i = 1$ ($1 \le i \le M$)
	\end{itemize}

	\SubtaskWithCost{4}{20}
	\begin{itemize}
		\item $C_i = 1$ ($1 \le i \le M$)
	\end{itemize}
	
	
	\SubtaskWithCost{5}{15}
	
	추가 제한조건이 없다.
	
	\Examples
		
	\begin{example}
	\exmp{
3 3
1 2 1
2 3 1
3 1 1
	}{%
1
	}%
	\end{example}
	
	모두가 시계방향으로 이동하면, 각 종류의 티켓이 하나씩 필요하다. 즉, 한 묶음만 사도 충분하다.	
	
	\begin{example}
	\exmp{
		3 2
		1 2 4
		1 2 2
	}{%
		3
	}%
	\end{example}

	다음 방법으로 이동하면 각 종류의 티켓이 세 장씩 필요하다:
	
	\begin{itemize}
		\item 첫 번째 요청에서, 세 명이 시계방향으로, 한 명이 반시계방향으로 움직인다.
		\item 두 번째 요청에서, 두 명이 반시계방향으로 움직인다.
	\end{itemize}

	그래서, 세 묶음을 사면 충분하다.
	
	두 묶음을 사서 이동하는 것은 불가능하기 때문에, 3을 출력한다.

	\begin{example}
	\exmp{
		6 3
		1 4 1
		2 5 1
		3 6 1
	}{%
		2
	}%
\end{example}

예를 들면 두 묶음을 사서 다음과 같이 나누어 주면 된다.

\begin{itemize}
	\item 1번 역에서 4번 역으로 이동하고 싶은 사람에게 1, 2, 3번 티켓을 준다.
	\item 2번 역에서 5번 역으로 이동하고 싶은 사람에게 1, 6, 5번 티켓을 준다.
	\item 3번 역에서 6번 역으로 이동하고 싶은 사람에게 3, 4, 5번 티켓을 준다.
\end{itemize}

한 묶음을 사서 이동하는 것은 불가능 하기 때문에, 2를 출력한다.

	
\end{problem}

