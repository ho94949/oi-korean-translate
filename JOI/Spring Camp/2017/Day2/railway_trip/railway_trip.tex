\begin{problem}{철도 여행}
	{standard input}{standard output}
	{2 seconds}{512 megabytes}{}
	
	JOI 전철은 철도 하나를 운영하고 있는 회사이다. JOI 전철의 철도에는 1번부터 $N$번 까지의 번호가 붙어있는 $N$개의 철도역이 일직선상에 놓여있다. 각 $i$ ($1 \le i \le N-1$)에 대해서, $i$번 역과 $i+1$번 역은 선로로 연결되어있다.
	
	JOI전철은 양방향으로 달리는 $K$종류의 열차가 있다. 열차의 종류는 1이상 $K$이하의 정수로 표현된다. 각 역에는 \textbf{중요도}라고 불리는 1이상 $K$이하의 정수가 하나씩 붙어 있다. $i$번 ($1 \le i \le N$) 역의 중요도는 $L_i$이다. 양 끝에 있는, 즉 1번과 $N$번 역은 중요도가 $K$이다.
	
	$j$번 ($1 \le j \le K$) 종류의 열차는 중요도가 $j$이상인 철도역에서만 멈추고, 다른 철도역에서는 멈추지 않는다. 양 끝에 있는 1번과 $N$번 철도역은 중요도가 $K$이기 때문에 모든 열차는 이 철도역에서는 멈춘다.
	
	많은 승객들이 JOI 전철을 이용하고 있다. 여행 중에 승객들은 목적지와 반대방향으로 움직이거나, 목적지를 통과 할 수도 있지만, 결국에는 목적지에 멈추어야 한다. 승객들은 역에 멈추는것을 좋아하지 않는다. 그렇기 때문에 도중에 정차하는 철도역의 수를 최소화 하고 싶다. 어떤 승객이 열차를 갈아타기 위해 역에 멈춘 경우에도 한 번으로 계산하고, 출발역과 도착역은 세지 않는다.
	
	당신의 업무는 승객의 출발역과 도착역이 주어졌을 때, 이 두 역의 사이를 이동하는 도중에 정차하는 철도역의 수를 구하는 프로그램을 작성해야 한다.
	
	\InputFile
	
	다음 정보가 표준 입력으로 주어진다.
	
	\begin{itemize}
		\item 첫째 줄에는 공백으로 구분된 세 정수 $N$, $K$, $Q$가 주어진다. 이는 JOI 전철에는 $N$개의 역이 있고, $K$종류의 열차가 있으며, 질문의 갯수가 $Q$개라는 것이다.
		\item 다음 $N$개의 줄의 $i$ 번째 ($1 \le i \le N$) 줄에는 정수 $L_i$가 주어진다. 이는 $i$번 역의 중요도가 $L_i$라는 것이다.
		\item 다음 $Q$개의 줄의 $k$ 번째 ($1 \le k \le Q$) 줄에는 공백으로 구분된 두 정수 $A_k$, $B_k$가 주어진다. 이는 $k$번째의 승객의 출발역이 $A_k$번 역이고, 도착역이 $B_k$번 역이라는 것을 의미한다.
	\end{itemize}
	
	
	\OutputFile
	
	표준 출력으로 $Q$ 개의 줄을 출력하여라. $k$ 번째 ($1 \le k \le Q$) 줄에는 $A_k$번 역에서 $B_k$번 역으로 이동 할 때, 도중에 정차하는 철도역의 최솟값을 출력하여라.
	
	\Constraints
	
	\begin{itemize}
		
		\item $2 \le N \le 100\ 000$.
		\item $1 \le K \le N$.
		\item $1 \le Q \le 100\ 000$.
		\item $1 \le L_i \le K$ ($1 \le i \le N$).
		\item $1 \le A_k \le N$ ($1 \le k \le Q$).
		\item $1 \le B_k \le N$ ($1 \le k \le Q$).
		\item $A_k \ne B_k$ ($1 \le k \le Q$).
	\end{itemize}
	
	
	\SubtaskWithCost{1}{5}
	\begin{itemize}
		\item $N \le 100$
		\item $K \le 100$
		\item $Q \le 50$
	\end{itemize}

	\SubtaskWithCost{2}{15}
	\begin{itemize}
		\item $Q \le 50$
	\end{itemize}

	\SubtaskWithCost{3}{25}
	\begin{itemize}
		\item $K \le 20$
	\end{itemize}
	
	\SubtaskWithCost{4}{55}
	
	추가 제한조건이 없다.
	
	\Examples
	
	\begin{example}
		\exmp{
			9 3 3 
			3
			1
			1
			1
			2
			2
			2
			3
			3
			2 4
			4 9
			6 7
		}{%
			1
			3
			0
		}%
	\end{example}
	
	이 예제에서, 질문은 세 가지가 있다.
	
	\begin{itemize}
		\item 첫 번째 질문은, 2번 역에서 4번 역까지 이동하는 것이다. 이 때, 2번 역에서 4번 역까지 1번 종류의 열차를 이용하면 도중에 정차하는 역은 3번 역 하나 뿐이 된다.
		\item 두 번째 질문은, 4번 역에서 9번 역까지 이동하는 것이다. 이 때, 우선 4번 역에서 5번 역까지 1번 종류의 열차를 이용하고, 다음에 5번 역에서 1번 역까지 2번 종류의 열차를 이용하고, 마지막으로 1번 역에서 9번 역까지 3번 종류의 열차를 이용하면 도중에 정차하는 역은 5번 역, 1번 역, 8번 역의 셋이 된다.
		\item 세 번째 질문은, 6번 역에서 7번 역까지 이동하는 것이다. 이 때, 6번 역에서 7번 역까지 2번 종류의 열차를 이용하면 도중에 정차하는 역 없이 이동하는 것이 가능하다.
	\end{itemize}
	

	\begin{example}
	\exmp{
		5 2 1
		2
		1
		1
		1
		2
		1 4
	}{%
		1
	}%
	\end{example}
	
	도중에 목적지가 있는 역을 지나쳐도 되는 점에 주의하여라.
	
	\begin{example}
	\exmp{
		15 5 15
		5
		4
		1
		2
		3
		1
		1
		2
		4
		5
		4
		1
		5
		3
		5
		8 1
		11 1
		5 3
		6 11
		9 12
		15 14
		15 2
		3 
		12
		2 1
		4 8
		15 5
		12 6
		1 13
		13 8
		14 9
	}{%
		2
		1
		1
		3
		2
		0
		3
		4
		0
		1
		3
		4
		1
		2
		2
	}%
	\end{example}
	
	
\end{problem}

