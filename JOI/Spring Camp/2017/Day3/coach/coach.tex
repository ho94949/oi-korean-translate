\begin{problem}{장거리 버스}
	{standard input}{standard output}
	{2 seconds}{256 megabytes}{}
	
	IOI나라에는 도시 I와 도시 O를 잇는 장거리버스가 달리고 있다. 버스의 내부에는 정수기가 있다. 탑승객은 정수기에서 물을 받아 마실 수 있다. 버스는 시간 0에 도시 I로 부터 출발하여, 시간 $X$에 도시 O로 도착한다. 버스의 경로 상에, 정수기에 물을 보충할 수 있는 장소가 $N$개 있다. 버스는 $i$ 번째 ($1 \le i \le N$) 장소에 시간 $S_i$에 도착 할 것이다.
	
	처음에 정수기에는 물이 들어있지 않다. 버스가 출발하기 전에 정수기에 물을 채울 수 있다. 또한 버스가 물을 보충할 수 있는 장소에 도착 했을 때에 물을 채울 수 있다. 물은 버스가 어디있든 1리터를 채우는 데에 $W$엔이 든다.
	
	도시 I에서 $M$명의 승객이 버스에 탑승한다. 승객들은 1번부터 $M$번까지의 번호가 붙어있다. 도시 I를 제외하고는 승객이 버스에 타지는 않는다. $j$번 ($1 \le j \le M$) 승객은 시간 $D_j$에 1리터의 물을 마시고 싶어 한다. 또한, 물을 마시면 $T$초 후에 1리터의 물을 마시고 싶어 한다. 다른 말로, $j$번 승객은 $D_j + kT$ ($k = 0, 1, 2, \cdots$) 시간에 물을 마시고 싶어 한다. 여기서, $1 \le D_j < T$를 만족하며, 모든 승객에 대해 $T$는 모두 같은 값이다. 만약 승객이 물을 마시기를 원할 때 정수기에 물이 담겨있지 않다면, 승객은 버스에서 내리게 된다. 만약 $j$번 손님이 도시 O에 도착하기 전에 버스에서 내리게 된다면, 승객에게 $C_j$엔을 환불 해 줘야 한다.
	
	버스기사도 역시 물을 마셔야 한다. 만약 물을 마시면, 그는 $T$초 후에 1리터의 물을 마시고 싶어 한다. 다른 말로, 버스기사는 시간 $kT$ ($k= 0, 1, 2, \cdots$) 시간에 물을 마시고 싶어 한다. 만약 버스기사가 물을 마시기를 원할 때 정수기에 물이 담겨있지 않다면, 버스는 더 이상 운행할 수 없다.
	
	어떠한 두 사람도 물을 마셔야 하는 시간이 같지 않다. 또한, 버스가 도시 O나 물을 보충할 수 있는 장소에 도달 했을 때 물을 마셔야 하는 승객 혹은 버스기사가 있는 경우는 없다.
	
	물을 보충할 수 있는 각 장소에서 물을 담는 양을 조절해서 도시 O까지 가는 도중에 물의 비용과 환불을 해 주는데 드는 비용의 합을 최소로 하고 싶다. 당신은 여행하는 도중 어디서 얼마나 물을 담아야 하는지를 결정해야 한다.
	
	\InputFile
	
	다음 정보가 표준 입력으로 주어진다.
	
	\begin{itemize}
		\item 첫째 줄에는 공백으로 구분된 다섯 정수 $X$, $N$, $M$, $W$, $T$가 주어진다. 이는 도시 O에 버스가 시간 $X$에 도착하며, 물을 보충할 수 있는 장소가 $N$개 있으며, $M$명의 손님이 버스에 타고 있으며, 물의 가격이 1리터당 $W$엔이며, 승객과 버스기사가 물을 마시는 간격이 $T$라는 것을 의미한다.
		\item 다음 $N$개의 줄의 $i$ 번째 ($1 \le i \le N$) 줄에는 정수 $S_i$가 주어진다. 이는 버스가 $i$ 번째 물을 보충할 수 있는 장소에 시간 $S_i$에 도착한다는 것이다.
		\item 다음 $M$개의 줄의 $j$ 번째 ($1 \le j \le M$) 줄에는 공백으로 구분된 두 정수 $D_j$, $C_j$가 주어진다. 이는 $j$번 승객이 시간 $D_j$에 처음 물을 마시기를 원하며, 환불을 할 때 $C_j$엔을 환불 해 주어야 한다는 의미이다.
	\end{itemize}

	
	\OutputFile
	
	표준 출력으로 한 개의 줄을 출력하여라. 출력은 최소 비용이다.
	
	\Constraints
	
	\begin{itemize}
	
	\item $1 \le X \le 1\ 000\ 000\ 000\ 000$.
	\item $1 \le N \le 200\ 000$.
	\item $1 \le M \le 200\ 000$.
	\item $1 \le W \le 1\ 000\ 000$.
	\item $1 \le T \le X$.
	\item $1 \le S_i < X$ ($1 \le i \le N$).
	\item $1 \le D_j < T$ ($1 \le j \le M$).
	\item $1 \le C_j \le 1\ 000\ 000\ 000$ ($1 \le j \le M$).
	\item $D_j$ ($1 \le j \le M$) 는 서로 다르다.
	\item 버스가 도시 O나 물을 보충할 수 있는 장소에 도달 했을 때 물을 마셔야 하는 승객 혹은 버스기사가 있는 경우는 없다.
	\end{itemize}
	
	
	\SubtaskWithCost{1}{16}
	\begin{itemize}
		\item $N \le 8$
		\item $M \le 8$
	\end{itemize}


	\SubtaskWithCost{2}{30}
	\begin{itemize}
		\item $N \le 100$
		\item $M \le 100$
	\end{itemize}

	\SubtaskWithCost{3}{25}
	\begin{itemize}
		\item $N \le 2\ 000$
		\item $M \le 2\ 000$
	\end{itemize}

	\SubtaskWithCost{4}{29}
	
	추가 제한조건이 없다.
	
	\Examples
		
	\begin{example}
	\exmp{
19 1 4 8 7
10
1 20
2 10
4 5
6 5
	}{%
103
	}%
	\end{example}
	
	이 예제에서, 우리가 출발 전에 7리터의 물을 넣고, 물을 보충할 수 있는 장소에서 4리터의 물을 넣은 경우, 버스는 다음과 같이 운행된다:
	
	\begin{enumerate}
		
		\item 버스가 도시 I를 떠난다. 이 때, 정수기는 7리터의 물이 담겨있다.
		\item 버스기사와 1, 2, 3, 4번 승객이 각각 시간 0, 1, 2, 4, 6에 물을 마신다. 정수기에 남은 물의 양은 2리터이다.
		\item 버스기사와 1번 승객이 각각 1리터의 물을 시간 7, 8에 마신다. 정수기에 남은 물의 양은 0리터이다.
		\item 시간 9에, 2번 승객은 물을 마시고 싶어 하지만 정수기에 물이 없기 때문에 버스를 떠난다.
		\item 시간 10에, 물을 보충할 수 있는 장소에서 4리터의 물을 정수기에 보충한다. 이 때, 정수기에는 4리터의 물이 담겨있다.
		\item 3, 4번 승객, 버스기사, 1번 승객이 각각 시간 11, 13, 14, 15에 물을 마신다. 정수기에 남은 물의 양은 0리터이다.
		\item 시간 18에, 3번 승객은 물을 마시고 싶어 하지만 정수기에 물이 없기 때문에 버스를 떠난다.
		\item 시간 19에, 버스는 도시 O에 도착한다.
	\end{enumerate}

	사용한 물의 총량은 11리터이다. 물값은 88엔이다. 2, 3번 승객을 환불 해 줄 때 사용한 돈의 양은 총 15엔이다. 총 사용한 돈의 양은 103엔이다.
	
	103엔 보다 더 작은 양을 사용하여 버스를 운영하는 것은 불가능 하므로, 102엔을 출력한다.
	
	
	\begin{example}
	\exmp{
		105 3 5 9 10
		59
		68
		71
		4 71
		6 32
		7 29
		3 62
		2 35
	}{%
		547
	}%
	\exmp{
		1000000000000 1 1 1000000 6
		999999259244
		1 123456789
	}{%
		333333209997456789
	}%
	\end{example}

	
\end{problem}

