\begin{problem}{긴 저택}
	{standard input}{standard output}
	{3 seconds}{256 megabytes}{}
	
	JOI군의 집 근처에는 긴 저택이 있다. 이 저택은 동에서 서로 일렬로 $N$개의 방이 있고, 가장 동쪽 방부터 $i$ 번째 방을 $i$번 방이라고 부르며, $i$번 ($1 \le i \le N-1$) 방과 $i+1$번 방을 잇는 통로가 있어서 양방향으로 오갈 수 있다. 방에서 통로로 들어가기 위해서는 열쇠가 필요하다. 각 열쇠에는 종류를 나타내기 위한 수가 하나씩 붙어 있다. 여러 열쇠에 같은 수가 붙어 있을 수도 있다.
	
	$i$번 방 혹은 $i+1$번 방으로 부터 두 방을 잇는 통로로 들어가기 위해서는 수 $C_i$가 붙어 있는 열쇠가 필요하다.
	
	방 $i$에는 $B_i$개의 열쇠가 있다. 이 열쇠에는 각각 수 $A_{i, j}$ ($1 \le j \le B_i$) 가 붙어있다. 만약 JOI군이 방에 들어가면, 그는 방에 있는 모든 열쇠를 집을 것이다. 그 이후에는 집은 열쇠들을 통로를 출입하는데 사용할 수 있다.
	
	JOI군은 열쇠를 원하는 횟수 만큼 사용할 수 있다. 가끔 JOI군은 같은 수가 붙어 있는 열쇠를 여러개 가지고 있는 경우도 있다. 하지만, 그 열쇠가 하나 있는 것과 특별히 다른 점은 없다.
	
	JOI군이 길을 잃는 경우를 방지하기 위해서, 다음 종류의 질문에 대답하는 프로그램을 작성하려고 한다.
	
	\begin{itemize}
		\item 만약 JOI군이 어떠한 열쇠도 가지지 않고 방 $x$에서 시작하여, 방 $y$로 이동할 수 있을까?
	\end{itemize}

	JOI군을 대신하여 이 문제를 답할 수 있는 프로그램을 작성하여 주자.
	
	\InputFile
	
	다음 정보가 표준 입력으로 주어진다.
	
	\begin{itemize}
		\item 첫째 줄에는 정수 $N$이 주어진다. 이는 저택에 있는 방의 갯수이다.
		\item 둘째 줄에는 $N-1$개의 공백으로 구분된 정수 $C_1$, $C_2$, $\cdots$, $C_{N-1}$이 주어진다. 이는 우리가 $i$번 방과 $i+1$번 방을 잇는 통로를 오가기 위해서 수 $C_i$가 붙은 열쇠가 필요하다는 의미이다.
		\item 다음 $N$개의 줄의 $i$ 번째 ($1 \le i \le N$) 줄에는 정수 $B_i$와, $B_i$개의 공백으로 구분된 정수 $A_{i, 1}$, $A_{i, 2}$, $\cdots$, $A_{i, B_i}$가 주어진다. 이는 $i$번 방에 $B_i$개의 열쇠가 있으며, 열쇠에 각각 수 $A_{i, j}$ ($1 \le j \le B_i$) 가 붙어있다는 의미이다.
		\item 다음 줄에는, 질문의 갯수 $Q$가 주어진다.
		\item 다음 $Q$개의 줄의 $k$ 번째 ($1 \le k \le Q$) 줄에는 공백으로 구분된 두 정수 $X_k$, $Y_k$가 주어진다. 이는 $k$ 번째 질문이 JOI군이 어떠한 열쇠도 가지지 않고 방 $X_k$에서 시작하여, 방 $Y_k$로 이동할 수 있는지를 묻는다는 의미이다.
	\end{itemize}
	
	
	\OutputFile
	
	표준 출력으로 $Q$개의 줄을 출력하여라. $Q$개의 줄의 $k$ 번째 ($ 1 \le k \le Q$) 줄은 JOI군이 어떠한 열쇠도 가지지 않고 방 $X_k$에서 시작하여, 방 $Y_k$로 이동할 수 있으면 \texttt{YES}, 아니면 \texttt{NO}여야 한다.
	
	\Constraints
	
	\begin{itemize}
		
		\item $2 \le N \le 500\ 000$.
		\item $1 \le Q \le 500\ 000$.
		\item $1 \le B_1 + B_2 + \cdots + B_N \le 500\ 000$.
		\item $1 \le B_i \le N$ ($1 \le i \le N$).
		\item $1 \le C_i \le N$ ($1 \le i \le N-1$).
		\item $1 \le A_{i, j} \le N$ ($1 \le i \le N$, $1 \le j \le B_i$).
		\item $B_i$개의 정수 $A_{i, 1}$, $A_{i, 2}$, $\cdots$, $A_{i, B_i}$는 서로 다르다.

		\item $1 \le X_k \le N$ ($1 \le k \le Q$).
		\item $1 \le Y_k \le N$ ($1 \le k \le Q$).
		\item $X_k \ne Y_k$ ($1 \le k \le Q$).
	\end{itemize}
	
	
	\SubtaskWithCost{1}{5}
	\begin{itemize}
		\item $N \le 5\ 000$
		\item $Q \le 5\ 000$
		\item $1 \le B_1 + B_2 + \cdots + B_N \le 5\ 000$.
	\end{itemize}

	\SubtaskWithCost{2}{5}
	\begin{itemize}
		\item $N \le 5\ 000$
		\item $1 \le B_1 + B_2 + \cdots + B_N \le 5\ 000$.
	\end{itemize}
	
	\SubtaskWithCost{3}{15}
	\begin{itemize}
		\item $N \le 100\ 000$
		\item $C_i \le 20$ ($1 \le i \le N-1$).
		\item $1 \le A_{i, j} \le 20$ ($1 \le i \le N$, $1 \le j \le B_i$).
	\end{itemize}
	
	\SubtaskWithCost{4}{75}
	
	추가 제한조건이 없다.
	
	\Examples
	
	\begin{example}
		\exmp{
			5
			1 2 3 4
			2 2 3
			1 1
			1 1
			1 3
			1 4
			4
			2 4
			4 2
			1 5
			5 3
		}{%
			YES
			NO
			NO
			YES
		}%
	\end{example}
	
	\begin{itemize}
		\item 첫째 예제에서, JOI군이 방을 2, 1, 2, 3, 4번 방 순서로 방문한다면, 4번 방에 도착할 수 있다.
		\item 둘째 예제에서, JOI군은 3, 4번 방 밖에 방문할 수 없다. 1과 3이 붙어 있는 열쇠 밖에 얻을 수 없으므로, 2번 방에 들어가지 못한다.
		\item 셋째 예제에서, JOI군은 4번 방에서 5번 방으로 가기 위한 종류 4의 열쇠를 얻을 수가 없기 때문에, 5번 방에 들어가지 못한다.
		\item 넷째 예제에서, JOI군이 방을 5, 4, 3번 방 순서로 방문한다면, 4번 방에 도착할 수 있다.
	\end{itemize}
	
	
	\begin{example}
		\exmp{
			5
			2 3 1 3
			1 3
			1 2
			1 1
			1 3
			1 2
			4
			1 3
			3 1
			4 3
			2 5
		}{%
			NO
			YES
			NO
			YES
		}%
		\exmp{
			7
			6 3 4 1 2 5
			1 1
			1 5
			1 1
			1 1
			2 2 3
			1 4
			1 6
			3
			4 1
			5 3
			4 7
		}{%
			YES
			NO
			YES
		}%
	\end{example}
	
	
\end{problem}

