\begin{problem}{불꽃놀이 막대}
	{standard input}{standard output}
	{2초}{256MB}{}
	
	JOI군은 자신을 포함하여 $N$ 명의 친구들과 불꽃놀이 막대를 가지고 놀 것이다. 이번에 사용할 불꽃놀이 막대는 불을 붙이면 정확히 $T$ 초 동안 불이 붙는다.
	
	처음에 JOI군과 친구들은 동서 방향으로 일직선으로 서 있고 한 사람당 하나의 불꽃놀이 막대를 들고 있다. JOI군과 친구들은 각각 1 이상 $N$ 이하의 번호가 붙어있다. $i<j$를 만족하는 $i$와 $j$에 대해서, $i$ 번째 사람은 $j$ 번째 사람의 서쪽에 서 있거나 같은 장소에 서 있다. $i$ 번째 사람과 가장 서쪽에 있는 첫 번째 사람의 거리는 $X_i$ 미터이다. JOI군은 $K$ 번째 사람이다.
	
	불꽃놀이를 시작하려 할 때 라이터의 연료가 충분하지 않다는 사실을 알았다. 오직 하나의 불꽃놀이 막대에만 불을 붙일 수 있다.
	
	그래서 일단 JOI군의 불꽃놀이 막대에 불을 붙이고 타고 있는 불꽃놀이 막대의 불을 옮겨가면서 불을 붙이기로 했다. 불꽃놀이 막대에서 불을 옮길 때는 다음 조건을 만족해야 한다.
	
	\begin{itemize}
		\item 불이 붙지 않은 불꽃놀이 막대를 불을 붙인 지 $T$ 초 이내의 불꽃놀이 막대와 맞닿아야 한다. 불을 붙인 지 정확히 $T$ 초가 지나도 불을 옮길 수 있다.
		\item 불을 붙이려는 불꽃놀이 막대는, 한 번도 불이 붙은 적이 없어야 한다.
		\item 불이 붙지 않은 불꽃놀이 막대와 불이 붙은 불꽃놀이 막대를 가진 사람이 같은 장소에 있어야 한다.
	\end{itemize}
	
	우리는 한 불꽃놀이 막대에서 다른 불꽃놀이 막대로 불이 붙기를 기다리는 시간 등을 무시할 것이다.
	
	JOI군과 친구들이 서로 떨어져 서 있어, 불을 붙이기 위해서는 잘 이동해야 한다. 그들은 임의의 속도로 달릴 수 있지만 불꽃놀이를 하는 중 달리면 위험하므로 속도가 초당 $s$ 미터를 넘지 않게 하고 싶다. 여기서, $s$는 음이 아닌 정수이다.
	
	모든 불꽃놀이 막대에 불을 붙이기 위해서 속도 제한을 어떻게 정하는 게 좋을까?
	
	\InputFile
	
	다음 정보가 표준 입력으로 주어진다.
	
	\begin{itemize}
		\item 첫째 줄에는 공백으로 구분된 세 정수 $N$, $K$, $T$가 주어진다. 이는 $N$ 명의 사람이 있고, JOI군이 $K$ 번째 사람이고, 막대 불꽃놀이에 불을 붙이면 $T$ 초 동안 불이 붙어있다는 의미이다.
		\item 다음 $N$ 개의 줄의 $i$ 번째 ($1 \le i \le N$) 줄에는 공백으로 정수 $X_i$가 주어진다. 이는 $i$ 번째 사람과 가장 서쪽에 있는 첫 번째 사람의 거리는 $X_i$ 미터라는 의미이다.
	\end{itemize}
	
	
	\OutputFile
	
	표준 출력으로 한 개의 줄을 출력하여라. 출력은 모든 불꽃놀이에 불을 붙이기 위한 음이 아닌 정수 속도 제한 $s$이다.
	
	\Constraints
	
	\begin{itemize}
		
		\item $1 \le K \le N \le 100\ 000$.
		\item $1 \le T \le 1\ 000\ 000\ 000$.
		\item $1 \le X_i \le 1\ 000\ 000\ 000$ ($1 \le i \le N$).
		\item $X_1 = 0$
		\item $X_i \le X_j$ ($1 \le i \le N$).	
	\end{itemize}
	
	
	\SubtaskWithCost{1}{30}
	\begin{itemize}
		\item $N \le 20$
	\end{itemize}
	
	\SubtaskWithCost{2}{20}
	\begin{itemize}
		\item $N \le 1\ 000$
	\end{itemize}
	
	\SubtaskWithCost{3}{50}
	
	추가 제한조건이 없다.
	
	\Examples
	
	\begin{example}
		\exmp{
			3 2 50
			0
			200 
			300
		}{%
			2
		}%
	\end{example}
	
	이 예제에서, 속도 제한은 초당 2미터여도 된다.
	
	첫 번째 사람이 동쪽으로, 두 번째와 세 번째 사람이 서쪽으로 움직인다. 속도는 초당 2미터이다. 50초 이후에 두 번째 사람은 첫 번째 사람에게 불을 옮길 수 있다.
	
	그리고 첫 번째 사람이 동쪽으로, 세 번째 사람이 서쪽으로 움직인다. 속도는 초당 2미터이다. 25초 이후에 첫 번째 사람은 세 번째 사람에게 불을 옮길 수 있다.
	
	속도 제한이 1미터였다면 모든 막대에 불을 붙일 수 없다.

	\begin{example}
	\exmp{
		3 2 10
		0
		200 
		300
	}{%
		8
	}%
	\end{example}
	
	이 예제에서, 속도 제한은 초당 8미터여도 된다.
		
	첫 번째와 두 번째 사람이 동쪽으로, 세 번째 사람이 서쪽으로 움직인다. 속도는 초당 8미터이다.
	
	3초 후에 두 번째 사람이 움직임을 멈춘다. 첫 번째와 세 번째 사람은 계속 움직인다.
	
	6.5초가 더 지난 이후에 두 번째 사람과 세 번째 사람이 같은 장소에 모인다. 두 번째 사람과 세 번째 사람이 움직임을 멈춘다. 첫 번째 사람은 계속 움직인다.
	
	0.5초가 더 지난 이후에 두 번째 사람은 세 번째 사람에게 불을 옮긴다. 첫 번째 사람은 계속 움직인다. 세 번째 사람은 서쪽으로 움직인다. 속도는 초당 8미터이다.
	
	9초가 더 지난 이후에 첫 번째 사람과 세 번째 사람이 같은 장소에 모인다. 세 번째 사람은 첫 번째 사람에게 불을 옮긴다.
		
	속도 제한이 7미터였다면 모든 막대에 불을 붙일 수 없다.
	
	\begin{example}
	\exmp{
		20 6 1
		0
		2
		13
		27
		35
		46
		63
		74
		80
		88
		100
		101
		109
		110
		119
		138
		139
		154
		172
		192
	}{%
		6
	}%
	\end{example}
	
	
\end{problem}

