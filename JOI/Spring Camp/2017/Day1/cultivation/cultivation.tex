\begin{problem}{경작}
	{standard input}{standard output}
	{2초}{256MB}{}
	
	21XX년, IOI 행성의 주민들은 최근 발견된 새로운 행성으로 이주하게 되었다.
	
	새로 발견된 행성은 $R$행 $C$열 격자로 이루어진 밭이 있다. 열은 남-북 방향으로, 행은 동-서 방향으로 놓여있다. 북쪽에서 $i$ 번째, 서쪽에서 $j$ 번째 격자 칸의 좌표는 $(i, j)$이다. 가장 북서쪽에 있는 격자 칸의 좌표는 $(1, 1)$이고, 가장 남동쪽에 있는 격자 칸의 좌표는 $(R, C)$이다. 매년 IOI 행성의 주민들은 밭에 불 바람의 방향을 고른다. 방향은 동, 남, 서 혹은 북이다.
	
	새 행성에서는 농업을 활성화하기 위해서, 밭의 모든 격자 칸에 ``JOI 풀"을 심을 것이다. 이주한 첫 연도에는 $N$ 개의 격자 칸에 JOI 풀이 심겨 있다.
	
	JOI 풀은 바람으로 생활권을 늘려간다. 여름마다 JOI 풀의 씨앗이 IOI 행성의 주민들이 고른 바람의 방향으로 날아간다. 씨앗은 원래 JOI 풀이 심겨 있던 곳에서 바람 방향으로 한 칸 움직여 땅에 착지한다. 만약 그 격자 칸에 JOI 풀이 심겨 있지 않다면, 그 격자 칸에는 새로운 JOI 풀이 자란다. 한 격자 칸에 JOI 풀이 심겨 있으면 그 풀은 영구히 자란다.
	
	바람의 방향을 적당히 설정했을 때, 밭의 모든 격자 칸에 JOI 풀을 심을 수 있으려면 최소 몇 년이 걸리는지 구하고 싶다.
	

	\InputFile
	
	다음 정보가 표준 입력으로 주어진다.
	
	\begin{itemize}
		\item 첫째 줄에는 공백으로 구분된 두 정수 $R$, $C$가 주어진다. 이는 격자가 $R$행 $C$열로 이루어져 있다는 의미이다.
		\item 둘째 줄에는 정수 $N$이 주어진다. 이는 이주 첫 연도에 JOI 풀이 심겨 있는 칸의 수가 $N$이라는 의미이다.
		\item 다음 $N$개의 줄의 $i$ 번째 ($1 \le i \le N$) 줄에는 공백으로 구분된 두 정수 $S_i$, $E_i$가 주어진다. 이는 이주 첫 연도에 $(S_i, E_i)$ 좌표의 격자 칸에 JOI 풀이 심겨 있다는 의미이다.
	\end{itemize}

	
	\OutputFile
	
	표준 출력으로 한 개의 줄을 출력하여라. 출력은 우리가 방향을 적당히 설정했을 때, 모든 격자칸에 JOI 풀이 심겨 있기 위해 필요한 연수의 최솟값이다.
	
	\Constraints
	
	\begin{itemize}
	
	\item $1 \le N \le 300$.
	\item $1 \le R \le 1\ 000\ 000\ 000$.
	\item $1 \le C \le 1\ 000\ 000\ 000$.
	\item $1 \le S_i \le R$ ($1 \le i \le N$).
	\item $1 \le E_i \le C$ ($1 \le i \le N$).
	\item 이주 첫 연도에 JOI풀이 심겨 있지 않은 격자가 존재한다.		
	\end{itemize}
	
	
	\SubtaskWithCost{1}{5}
	\begin{itemize}
		\item $R \le 4$
		\item $C \le 4$
	\end{itemize}

	\SubtaskWithCost{2}{10}
	\begin{itemize}
		\item $R \le 40$
		\item $C \le 40$
	\end{itemize}

	\SubtaskWithCost{3}{15}
	\begin{itemize}
		\item $R \le 40$
	\end{itemize}

	\SubtaskWithCost{4}{30}
	\begin{itemize}
		\item $N \le 25$
	\end{itemize}


	\SubtaskWithCost{5}{20}
	\begin{itemize}
		\item $N \le 100$
	\end{itemize}
	
	
	\SubtaskWithCost{6}{20}
	
	추가 제한조건이 없다.
	
	\Examples
		
	\begin{example}
	\exmp{
3 4
3
1 2
1 4
2 3
	}{%
3
	}%
	\end{example}

	이 예제에서, 이주 첫 연도에 다음 격자 칸에 JOI 풀이 심겨 있다.
	
	
	\begin{center}
	
		\begin{tabular}{|l|l|l|l|}
			\hline
		\phantom{0}	& 0 &   & 0 \\ \hline
			&   & 0 &   \\ \hline
			&   &   &   \\ \hline
		\end{tabular}
	
	새 행성의 밭. `0'이 쓰여 있는 격자 칸은 이주 첫 연도에 JOI 풀이 심겨 있다.
	\end{center}
	
	만약에 처음 3년 동안 바람을 서쪽, 남쪽, 남쪽으로 불어가게 만들면, 모든 격자 칸에는 3년 이후에는 JOI 풀이 심겨 있을 것이다. 다음 숫자는 각 격자 칸에 JOI 풀이 심긴 연도를 의미한다. 이는 최솟값이다.
	
	\begin{center}
			\begin{tabular}{|l|l|l|l|}
		\hline
		1 & 0 &  1 & 0 \\ \hline
	2	&  1 & 0 &  2 \\ \hline
	3	& 2  & 2  &   3\\ \hline
	\end{tabular}
	\end{center}

	\begin{example}
	\exmp{
		4 4 
		4
		1 1
		1 4
		4 1
		4 4
	}{%
		4
	}%
	\end{example}
	
	
\end{problem}

