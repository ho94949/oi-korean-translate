\begin{problem}{항구 시설}
	{standard input}{standard output}
	{3.5 seconds}{1024 megabytes}{}
	
	JOI 항구에는 매일 많은 컨테이너가 운반되어 오고, 전국 각지로 트럭을 통해 운반된다.
	
	JOI 항구는 매우 좁아서, 컨테이너를 넣을 수 있는 공간이 2개 밖에 없다. 각각의 공간에는 컨테이너를 수직으로 쌓아서 몇 개든 넣을 수 있다.
	
	안전상의 이유로, 컨테이너가 항구에서 운반되어 오면 두 공간 중 하나에 컨테이너를 놓아야 한다. 만약 이미 컨테이너가 그 위치에 있으면, 이미 있는 컨테이너의 위에 새 컨테이너를 쌓는다. 트럭으로 운반할 때는, 두 공간에 쌓인 컨테이너 중 가장 위에서부터 차례대로 운반해야 한다.
	
	오늘, JOI 항구에는 $N$개의 컨테이너가 배로 운반될 예정이다. 모든 컨테이너는 오늘 내로 트럭으로 운반될 예정이다.
	
	당신은 JOI 항구의 항구시설 관리를 맡고 있어서, 모든 컨테이너가 배로 운반되는 시각과 트럭으로 운반되는 시간을 알고 있다. 컨테이너를 쌓고 가져가는 경우의 수를 1 000 000 007로 나눈 나머지를 구하여라.
	
	\InputFile

다음 정보가 표준 입력으로 주어진다.

\begin{itemize}
	\item 첫째 줄에는 공백으로 정수 $N$이 주어진다. 이는 JOI 항구에 운반될 컨테이너의 수가 $N$이라는 의미이다.
	\item 다음 $N$개의 줄의 $i$ 번째 ($1 \le i \le N$) 줄에는 공백으로 구분된 두 정수 $A_i$, $B_i$가 주어진다. 이는 JOI 항구에 $i$ 번째 컨테이너가 시각 $A_i$에 와서 시각 $B_i$에 트럭으로 운반된다는 의미이다.
\end{itemize}


\OutputFile

표준 출력으로 한 개의 줄을 출력하여라. 출력은 컨테이너를 쌓고 가져가는 경우의 수를 1 000 000 007로 나눈 나머지이다.

\Constraints

\begin{itemize}
	
	\item $1 \le N \le 1\ 000\ 000$.
	\item $1 \le A_i \le 2N$ ($1 \le i \le N$).
	\item $1 \le B_i \le 2N$ ($1 \le i \le N$).
	\item $A_i < B_i$ ($1 \le i \le N$).
	\item $2N$개의 정수 $A_1$, $\cdots$, $A_N$, $B_1$, $\cdots$, $B_N$은 서로 다르다.	
\end{itemize}


\SubtaskWithCost{1}{10}
\begin{itemize}
	\item $N \le 20$
\end{itemize}

\SubtaskWithCost{2}{12}
\begin{itemize}
	\item $N \le 2\ 000$
\end{itemize}

\SubtaskWithCost{3}{56}
\begin{itemize}
	\item $N \le 100\ 000$
\end{itemize}


\SubtaskWithCost{4}{22}

추가 제한조건이 없다.

\Examples

\begin{example}
	\exmp{
		4
		1 3
		2 5
		4 8
		6 7
	}{%
		4
	}%
\end{example}

컨테이너를 놓는 네 가지 방법이 있다. 각 공간을 A, B라고 하자. 다음 방법으로 컨테이너를 놓을 수 있다.

\begin{itemize}
	\item 1, 2, 3, 4번 컨테이너를 각각 A, B, A, A에 놓는다.
	\item 1, 2, 3, 4번 컨테이너를 각각 A, B, A, B에 놓는다.
	\item 1, 2, 3, 4번 컨테이너를 각각 B, A, B, A에 놓는다.
	\item 1, 2, 3, 4번 컨테이너를 각각 B, A, B, B에 놓는다.
\end{itemize}

\begin{example}
	\exmp{
		3
		1 4
		2 5
		3 6
	}{%
		0
	}%
	\exmp{
	5
	1 4
	2 10
	6 9
	7 8
	3 5
}{%
	8
}%
\exmp{
8
1 15
2 5
3 8
4 6
14 16
7 9
10 13
11 12
}{%
16
}%
\end{example}


\end{problem}

