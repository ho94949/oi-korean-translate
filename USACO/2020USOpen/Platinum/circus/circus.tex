\begin{problem}{서커스}
	{circus.in}{circus.out}
	{2초}{256MB}{}
	
	농부 존의 서커스에 참가하는 $N$ 마리 ($1 \le N \le 10^5$) 소는 다음 연기를 준비하고 있다. 연기는 정점에 $1$번부터 $N$번까지 번호가 붙어있는 트리에서 진행된다. ``시작 상태"는 $1 \le K \le N$을 만족하는 정수 $K$와 $1$번부터 $K$번까지의 소가 한 정점당 최대 한 마리씩 트리의 정점에 배치된 것이다.
	
	연기하면서 소는 임의의 수 만큼 ``동작"을 한다. 한 동작에서 한 마리의 소는 비어있는 인접한 정점으로 움직인다. 유한한 수의 동작을 통해 한 시작 상태에서 다른 시작 상태로 갈 수 있다면 두 시작 상태를 ``동등"하다고 한다.
	
	각 $1 \le K \le N$에 대해 동치류의 수, 즉, 어떤 두 시작 상태도 동등하지 않도록 시작 상태를 고를 수 있는 최대 개수를 구하여라. 답이 매우 클 수 있으니 $10^9+7$로 나눈 나머지를 출력하여라.
	
	\InputFile
	
	첫째 줄에는 $N$이 주어진다.
	
	$2 \le i \le N$인 $i$ 번째 줄에는 $a_i$번 정점과 $b_i$번 정점 사이에 간선이 있다는 의미에 $a_i$와 $b_i$가 주어진다.
	
	
	\OutputFile
	
	각 $1 \le i \le N$을 만족하는 $i$에 대해, $i$번째 줄에 $K=i$인 경우에 대한 답을 $10^9+7$로 나눈 나머지를 출력하여라.
	
	
	\Example
		
	\begin{example}
	\exmp{
5
1 2
2 3
3 4
3 5
	}{%
1
1
3
24
120
	}%
	\end{example}
	
	$K=1$과 $K=2$인 경우에, 모든 시작 상태가 동등하다.
	
	$K=3$인 경우에, $c_i$를 $i$번 소의 위치라고 하자. $(c_1, c_2, c_3) = (1, 2, 3)$인 상태는 $(1, 2, 5)$인 상태, $(1, 3, 2)$인 상태와 동등하다. 하지만 $(2, 1, 3)$인 상태와 동등하지는 않다.
	
	\begin{example}
	\exmp{
		8
		1 3
		2 3
		3 4
		4 5
		5 6
		6 7
		6 8
	}{%
		1
		1
		1
		6
		30
		180
		5040
		40320
	}%
	\end{example}
	
	\Scoring
	
	\begin{itemize}
		\item 3--4번 테스트 케이스는 $N \le 8$을 만족한다.
		\item 5--7번 테스트 케이스는 $N \le 16$을 만족한다.
		\item 8--10번 테스트 케이스는 $N \le 100$을 만족하고, 트리는 ``별" 모양이다. 즉, 정점의 차수가 2 이상인 정점이 하나이다.
		\item 11--15번 테스트 케이스는 $N \le 100$을 만족한다.
		\item 16--20번 테스트 케이스는 추가 제한조건이 없다.
	\end{itemize}
	
	
	
	
\end{problem}

