\begin{problem}{열차 추적}
	{}{}
	{2 seconds}{256 megabytes}{}

	오전마다 특급열차는 농장을 출발하여 대도시로 향하고, 오후에는 반대방향으로 시골로 되돌아온다. 오늘, 베시는 시간을 내서 오전과 오후에 열차를 보려고 한다.
	
	소 베시는 열차가 0번부터 $N-1$번까지의 번호가 붙은 $N$개의 ($1 \le N \le 10^6$) 칸으로 되어있는 것을 안다. $i$번째 칸에는 식별번호 $c_i$ ($0 \le c_i \le 10^9$)가 붙어있다. 모든 숫자는 오전과 오후 모두에 볼 수 있고, 각 칸마다 베시는 수를 알 수 있는 두 번의 기회가 있다.
	이는, 오전에 열차가 지나갈 때, $c_0$, $c_1$, $\cdots$, $c_{N-1}$을 차례로 볼 수 있고, 오후에 다시 $c_0$, $c_1$, $\cdots$, $c_{N-1}$을 차례로 볼 수 있다.

	베시는 정수 $K$ ($1 \le K \le N$)을 골라서, 연속된 $K$개 칸의 식별번호의 최솟값들을 모두 알고 싶어 한다. 그는 계산을 할 수 있는 공책이 있지만, 이 공책은 작고 손글씨 (발굽글씨?) 는 크다. 예를 들어, $N+1-K$개의 최솟값을 모두 쓸 수 있는 공간이 없을 수도 있다. 비밀스러운 이유로 인해, 베시는 수를 계산 한 후에 하늘에 계산한 결과를 말하기 때문에, 문제는 아니다.

	열차가 곧 도착한다! 베시가 열차가 두 번 지나는 동안 제한된 공책을 사용하여 $N+1-K$개의 최솟값을 찾는것을 도와주어라. 공책은 5500개의 구역으로 나뉘어 있고, 0번부터 5499번 까지의 번호가 붙어있으며, 각 구역은 $-2^{31}$이상 $2^{31}-1$이하의 정수를 담을 수 있다. 처음에 각 구역은 정수 0이 적혀있다.

	이 문제는 interactive 문제이다. 하지만 표준 입출력 혹은 파일 입출력을 사용하지 않는 문제이다. 당신은 베시가 제한된 공책 공간을 활용하기 위한 다음 함수를 구현해야 한다.
	
	\texttt{void helpBessie(int ID);}
	
	오전과 오후에 기차가 지나갈 때, 함수가 호출 될 것이고 인자는 각 칸에 적힌 식별번호이다.	

	\texttt{helpBessie}함수는 다음 함수를 호출할 수 있다.
	
	\begin{itemize}
	\item \texttt{int get(int index)}: 베시의 공책에 해당하는 \texttt{index}번 칸에 적힌 숫자를 가져온다.
	\item \texttt{int set(int index, int value)}: 베시의 공책에 해당하는 \texttt{index}번 칸에 \texttt{value}를 적는다.
	\item \texttt{void shoutMinimum(int output)}: 베시에게 하늘에 결과 \texttt{output}을 말하라고 한다.
	\item \texttt{int getTrainLength()}: 칸의 갯수 $N$을 반환한다.
	\item \texttt{int getWindowLength()}: 연속으로 봐야 할 칸의 갯수 $K$을 반환한다.
	\item \texttt{int getCurrentCarIndex()}: 현재 보고 있는 열차가 몇번 칸인지를 반환한다.
	\item \texttt{int getCurrentPassIndex()}: 보고 있는 열차가 오전이면 0, 오후면 1을 반환한다.
	\end{itemize}

	코드를 작성하는 것을 돕기 위해서, C/C++과 Java로 작성된 기본 템플릿을 제공했다. Python과 Pascal 제출은 이 문제에 대해서 허용되지 않는다. 	

	연속된 $K$개의 최솟값은 순서대로 출력되어야 한다. (즉 $0$, $1$, $\cdots$, $K-1$번 칸의 식별번호의 최솟값 다음으로 $1$, $2$, $\cdots$, $K$번 칸의 식별번호의 최솟값이 출력되어야 하고, ... 순서로 출력되어야 한다.)
	
	하지만 이 제한과 별개로 함수가 호출의 어떤 때에 결과를 출력해도 상관 없다. 예를 들어, 함수는 몇 호출에는 출력이 없을 수도, 몇 호출에는 여러개의 출력을 할 수도 있다.
	
	베시는 매우 훌륭한 초단기기억력을 가지고 있어서, \texttt{helpBessie}함수 안의 메모리 사용은 256MB 메모리 제한 이외에는 없다. 하지만, 열차의 칸 사이에는 공책에 적히지 않은 그 어떠한 것도 기억할 수 없다. 그래서, 함수 호출 사이에는 \texttt{set}과 \texttt{get}을 제외하고는 어떠한 상태도 남기면 안된다. 
	
	이는:
	
	\textbf{상수가 아닌 어떠한 전역 혹은 스태틱 변수도 선언해서는 안된다. 이럴 경우 이 문제에 대한 제출은 무효처리 될 것이다. 코치는 이 문제의 정신을 잘 따르는지 손으로 각 제출을 확인 할 것이다. 입출력은 필요로하지 않으므로, 이 문제에서 입출력을 하는것도 허용되지 않는다.}
	
	\texttt{set}함수와 \texttt{get}함수의 호출 제한은 각 테스트 케이스 마다 $25 \cdot 10^6$으로 제한된다.

	
	\Examples
	
	\begin{example}
		\exmp{
			10 3
			5 7 9 2 0 1 7 4 3 6
		}{%
			5
			2
			0
			0
			0
			1
			3
			3
		}%
	\end{example}

	
	
\end{problem}

