\begin{problem}{황야의 소}
	{atlarge.in}{atlarge.out}
	{4 seconds}{256 megabytes}{}
	
	궁지에 몰린 베시는 농장이의 지하로 도망갔다. 농장은 $N$개의 ($2 \le N \le 7 \cdot 10^4$) 외양간으로 되어있고, $N-1$개의 농장을 서로 양방향으로 잇는 터널이 있어서, 각 쌍의 외양간을 잇는 유일한 경로가 있다. 단 하나의 터널이 연결되어 있는 외양간을 출구라고 한다. 아침이 밝아오면, 베시는 어떤 외양간으로 나와서 출구로 가려고 노력 할 것이다.
	
	하지만 베시가 어떤 외양간으로 나오는 순간, 위치를 알 수 있게 된다. 그래서 농부들은 출구로 부터 시작해서, 베시를 잡으려고 노력할 것이다. 농부들은 베시와 같은 속도로 움직인다. (한 단위시간 마다, 각 농부는 어떤 외양간에서 다른 외양간으로 갈 수 있다.) 농부들은 베시가 어디있는지 항상 알고 있고, 베시는 농부들이 어디 있는지 항상 알 고 있다. 농부들이 베시를 잡는 순간은 농부 중 하나가 같은 외양간에 있거나, 베시와 같은 터널을 지나던 중에 만난 때 이다. 역으로, 베시는 잡히기 전에 출구 중 하나로 도망치면 된다. 
	
	베시는 어떤 외양간으로 나와야 할 지 확실하지 않다. $N$개의 각 외양간에 베시가 있는 경우에 대해, 농부들을 최적으로 배치 할 때 몇 명이 있어야 베시를 잡을 수 있는지 구하여라.
	
	
	\InputFile
	
	첫째 줄에는 세 정수 $N$이 주어진다. 
	
	
	다음 $N-1$개의 줄에는 1 이상 $N$ 이하의 두 정수가 주어진다. 이는 외양간을 잇는 터널을 의미한다.
	
	
	\OutputFile
	
	$N$개의 줄을 출력하여라. $i$번째 줄은 $i$번째 외양간에서 베시가 나왔을 때 베시를 잡기 위한 최소의 농부의 수이다.
		
	
	\Constraints
		
	\begin{example}
	\exmp{
7
1 2
1 3
3 4
3 5
4 6
5 7
	}{%
3
1
3
3
3
1
1
	}%
	\end{example}
	
	
	
\end{problem}

