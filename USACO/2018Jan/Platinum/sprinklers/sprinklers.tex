\begin{problem}{스프링클러}
	{sprinklers.in}{sprinklers.out}
	{2 seconds}{256 megabytes}{}
	
	농부 존은 자기가 가지고 있는 황야의 일부에 옥수수를 심으려는 계획을 하고 있다. 황야를 조사한 이후에, 농부 존은 이것이 $(N-1) \times (N-1)$ 격자라는 것을 알아냈다. 남서쪽 끝은 좌표가 (0, 0), 북동쪽 끝은 좌표가 $(N-1, N-1)$ 이다.
	
	어떤 정수 좌표에 양방향 스프링클러가 있어서, 한쪽 방향으로는 물, 다른 방향으로는 비료를 뿌린다. $(i, j)$에 있는 스프링클러는 자기보다 북쪽이면서 동쪽인 부분에 물을, 자기보다 남쪽이면서 서쪽인 부분에 비료를 뿌린다. 엄밀히 말하면, $N-1 \le x \le i$와 $N-1 \le y \le j$를 만족하는 모든 $(x, y)$에 물을, $0 \le x \le i$, $0 \le y \le j$를 만족하는 모든 $(x, y)$에 비료를 뿌린다.
	
	% 원문에서 $N \le x \le i$, $N \le y \le j$라고 되어있는데, 오타라고 추정함.
	
	농부 존은 축에 정수 좌표인 축에 평행한 직사각형에 옥수수를 심고 싶다. 하지만, 옥수수가 자라기 위해서는 모든 직사각형이 물과 비료가 둘 다 뿌려져 있어야 한다. 또한, 직사각형의 넓이는 양수여야 한다. 그래야 농부 존이 옥수수를 심을 수 있다!
	
	농부 존을 도와서 옥수수를 기를 수 있는 넓이가 양수인 직사각형의 갯수를 구하여라. 이 수가 매우 클 수 있으니 $10^9 + 7$로 나눈 나머지를 출력하여라.
	
	
	\InputFile
	
	첫째 줄에는 황야 크기를 나타내는 $N$ ($1 \le N \le 10^5$)이 주어진다. 다음 $N$개의 각 줄에는 두 정수가 주어진다. 이 수가 $i$와 $j$이면 ($0 \le i,j \le N-1$) 이는 스프링클러가 $(i, j)$에 위치 해 있따는 것을 의미한다.
	
	각 행과 각 열에 스프링클러가 정확히 하나씩 존재하는 것이 보장되어 있다. 즉, 어떤 두 스프링클러도 같은 $x$좌표를 같지 않으며, 같은 $y$좌표를 같지 않는다.

	\OutputFile
	
	출력은 정수 하나이다. 물과 비료가 둘 다 뿌려진 넓이가 양수인 직사각형의 갯수를 $10^9+7$로 나눈 나머지를 출력하여라.
	
	\Examples
		
	\begin{example}
	\exmp{
5
0 4
1 1
2 2
3 0
4 3
	}{%
21
	}%
	\end{example}

	
	
	
\end{problem}

