\begin{problem}{새 헛간}
	{newbarn.in}{newbarn.out}
	{2 seconds}{256 megabytes}{}

	농부 존은 소들이 너무 가까이 있으면 언쟁을 한다는 사실을 알고, 소들을 멀리 떨어뜨려놓기 위하여 새로운 헛간을 여럿 열려 한다.
	
	농부 존이 새 헛간을 열 때, 다른 헛간과 양방향으로 연결된 길을 최대 한 개 만든다. 소들이 충분히 멀리 떨어져 있다는 것을 보장하기 위해서, 그는 특정한 헛간과 연결된 농장중 가장 거리가 먼 헛간과의 거리를 구하려고 한다. (두 헛간의 거리는 한 헛간에서 다른 헛간으로 가기 위해 거쳐하는 최소한의 길의 수이다.)
	
	농부 존은 총 $Q$개의 질의를 줄 것이고, 각각은 ``건설" 혹은 ``거리" 이다. 건설에 대해서는, 농부 존은 헛간을 짓고, 이미 지어진 헛간과 양방향으로 연결된 길을 최대 한 개 만든다. 거리에 대해서는, 농부 존이 주어진 헛간과 연결된 농장 중 가장 거리가 먼 헛간과의 거리를 구하려고 한다. 주어진 헛간이 존재함은 보장된다. 농부 존을 도와 질의를 대답하자.
	
	
	\InputFile
	
	첫째 줄에는 정수 $Q$가 주어진다. 다음 $Q$개의 줄에는 질의가 들어온다. 각 질의는 ``\texttt{B} $p$" 혹은 ``\texttt{Q} $k$"이고, 각각 헛간을 건설해서 $p$번 헛간과 양방향으로 연결하라는 것과, $k$번 헛간에서 가장 먼 헛간과의 거리를 구하라는 의미이다. $p=-1$인 경우, 새로운 헛간은 다른 어떠한 헛간과도 연결되어있지 않다. 아닌 경우, $p$는 이미 지어진 헛간의 번호이다. 헛간의 번호는 1부터 시작하며, 첫째로 지어진 헛간에 번호는 1, 둘째로 지어진 헛간의 번호는 2, ...와 같이 계속 된다.
	
	\OutputFile
	
	각 거리 질의마다 하나의 줄을 출력하여라. 연결되어 있는 다른 헛간이 없을 경우 답이 0임에 유의하여라.
	
	
	\Examples
	
	\begin{example}
		\exmp{
			7
			B -1
			Q 1
			B 1
			B 2
			Q 3
			B 2
			Q 2
		}{%
			0
			2
			1
		}%
	\end{example}

	\Note
	
	예제의 헛간은 다음과 같은 구조를 가진다.
	
	\begin{verbatim}
  (1) 
   \   
    (2)---(4)
   /
  (3)	
	\end{verbatim}
	
	1번 질의에서, 1번 헛간을 짓는다. 2번 질의에서, 1번 헛간과 연결된 가장 먼 헛간과의 거리를 묻는다. 1번 헛간은 다른 헛간과 연결되지 않았으므로, 답은 0이다. 3번 질의에서 2번 헛간을 짓고, 1번 헛간과 연결한다. 4번 질의에서, 3번 헛간을 짓고, 2번 헛간과 연결한다. 5번 질의에서, 3번 헛간과 연결된 가장 먼 헛간과의 거리를 묻는다. 가장 먼 헛간은 1번 헛간이고, 거리 2만큼 떨어져 있다. 6번 질의에서, 4번 헛간을 짓고, 2번 헛간과 연결한다. 7번 질의에서, 2번 헛간과 연결된 가장 먼 헛간과의 거리를 묻는다. 1번, 3번, 4번 헛간은 모두 거리 1씩 떨어져 있고, 1이 질의에 대한 답이다.
	
	
\end{problem}

