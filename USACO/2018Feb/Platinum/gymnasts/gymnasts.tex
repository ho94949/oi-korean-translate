\begin{problem}{소 체조선수}
	{gymnasts.in}{gymnasts.out}
	{2 seconds}{256 megabytes}{}
	
	농장 생활에 질려서, 소들은 속세의 재산을 모두 팔고 순례공연을 하는 단체에 가입했다. 이제까지, 소들은 횃불로 저글링을 하거나, 외줄타기를 하거나, 외발자전거를 타는 등의 편리한 발굽이 있는 소라면 어렵지 않은 일들만 받았다. 하지만, 연기 지도자는 더 감동적인 공연을 다음에 보여주고 싶었다.
	
	새로운 공연장은 $N$개의 무대가 원형으로 배치되어있다. 각 무대에는 한 마리 이상, $N$ 마리 이하의 소로 탑을 쌓아야 한다. 연기 지도자가 신호를 보내면, 모든 탑에 있는 소들이 시계방향으로 떨어져서, 가장 아래에 있는 소는 움직이지 않고, 그 위에 있는 소가 시계방향으로 무대 하나 만큼 이동하고, 또 그 위에 있는 소는 시계방향으로 무대 둘 만큼 이동하고, ...을 반복한다. 훌륭한 체조선수였던 소들이 이를 하는데에 대한 기술적인 문제는 없을 것이다. 소들이 떨어질 때 서로 간섭하지 않을 것이고, 그래서 모든 소들이 올바른 무대에 착지 할 것이다. 올바른 무대에 착지한 소들은 다시 탑을 만들고, 이는 떨어지지 않는다.
	
	연기 지도자는, 탑이 쓰러진 이후에 새로 쌓인 탑이 원래의 탑과 같은 높이를 가지면 공연이 감동적일 거라고 생각했다. 이런 탑의 높이를 ``마법"의  높이라고 한다. 소들을 위해서 마법의 높이를 가지는 탑의 높이의 경우의 수를 계산 해 주자. 수가 매우 클 수 있으니, $10^9+7$로 나눈 나머지를 출력하여라.
	
	두 경우에서 탑의 높이가 다른 무대가 존재하는 경우, 두 경우는 다르다고 생각한다.
	
	
	
	\InputFile
	
	첫째 줄에는 정수 $N$ ($1 \le N \le 10^{12}$)가 주어진다.
		
	\OutputFile
	
	마법의 높이인 경우의 수를 $10^9+7$로 나눈 나머지를 출력하여라.
	
	\Constraints
		
	\begin{example}
	\exmp{
4
	}{%
6
	}%
	\end{example}
	
	\Note
	
	$N=4$인 경우, 가능한 방법은 (1, 1, 1, 1), (2, 2, 2, 2), (3, 3, 3, 3), (4, 4, 4, 4), (2, 3, 2, 3), (3, 2, 3, 2)이다.	
	
\end{problem}

