\begin{problem}{저울대}
	{balance.in}{balance.out}
	{2 seconds}{256 megabytes}{}
	
	
	새 헛간의 마굿간을 지을 돈을 모으기 위하여, 소 베시는 저울대에서 앞뒤로 움직이면서 균형을 잡는 놀라운 서커스를 하기 시작했다.

	베시가 버는 돈의 양은 최종적으로 저울대에서 점프하는 위치와 고나련 되어있다. 저울대의 위치는 왼쪽에서 오른쪽으로 $0$, $1$, $\cdots$, $N+1$의 번호가 붙어있다. 만약 베시가 $0$ 혹은 $N+1$에 도달하면 저울대에서 떨어져서 돈을 받을 수 없다.

	만약 베시가 $k$번 위치에 있다면, 다음 중 하나를 할 수 있다.

	\begin{enumerate}
		\item 동전을 던진다. 만약 뒷면이 나오면 $k-1$번 위치로 간다. 앞면이 나오면 $k+1$번 위치로 간다. (앞뒷면이 나올 확률은 각각 $\dfrac{1}{2}$이다.)
		\item 저울대에서 점프를 하여 $f(k)$ 만큼의 돈을 번다. ($0 \le f(k) \le 10^9$)
	\end{enumerate}
	
	베시의 움직임은 무작위 동전에 영향을 받기 때문에 어떤 결과가 나올지 보장할 수 없다는 것을 알고 있다. 하지만, 시작하는 위치에 따라서 최적의 결정들을 할 때 벌 수 있는 돈의 기댓값을 구하고 싶다. (``최적의" 라는 뜻은, 돈의 기댓값을 최대로 하는 것이다.) 예를 들어, 전략이 $1/2$의 확률로 10, $1/4$의 확률로 8, $1/4$의 확률로 0만큼의 돈을 번다면, 벌 수 있는 돈의 기댓값은 가중평균인 $10(1/2)+8(1/4)+0(1/4)=7$이다.

	
	\InputFile
	
	첫째 줄에는 정수 $N$이 주어진다. 다음 $N$개의 줄에는 $f(1) \cdots f(N)$이 주어진다.
	
	\OutputFile
	
	$N$개의 줄을 출력하여라. $i$ 번째 줄에는 $i$번 위치에서 시작해서, 최적의 전략을 사용하였을 때 벌 수 있는 돈의 기댓값의 $10^5$배를 가장 가까운 정수로 반올림하여 출력하여라.

	\Examples
	
	\begin{example}
		\exmp{
			2
			1
			3
		}{%
			150000
			300000
		}%
	\end{example}

	
\end{problem}

