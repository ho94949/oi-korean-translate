\begin{problem}{울타리 치기}
	{fencing.in}{fencing.out}
	{2 seconds}{256 megabytes}{}
	
	농부 존은 당신에게 소들의 움직임을 제한하기 위해 지으려는 직선 모양 울타리 위치를 정하기 위한 도움을 요청했다. 그는 몇몇 울타리 위치를 정했고 이 위치들이 유용한지, 즉 모든 소들이 울타리의 한 쪽에 있는지 알려달라는 요청을 했다. 소가 정확히 울타리 위에 있을 경우 위치는 유용하지 않다. 농부 존은 울타리 위치에 관한 몇몇 질문을 하기로 했고, 유용한 위치이면 ``\texttt{YES}", 아니면 ``\texttt{NO}"로 대답해야 한다.
	
	또한, 농부 존은 새로 소를 데리고 올 수 있다. 새로운 소가 들어오면 그 시점 이후에 들어오는 질문에 대해서 새로 들어온 소도 같은 위치에 있어야 한다.

	
	\InputFile
	
	첫째 줄에는 정수 $N$과 $Q$가 공백으로 구분되어 주어진다. ($1 \le N \le 100,000$, $1\le Q \le 100,000$) 이는 각각 처음 소의 수와, 질의의 수를 의미한다.
	
	다음 $N$개의 줄은 처음 소들의 위치를 나타낸다. 각 줄은 소의 위치를 의미하는 공백으로 구분된 두 정수 $x$, $y$가 주어진다.
	
	다음 $Q$개의 줄은 새로운 소가 추가되거나 울타리가 유용한지에 관한 질의를 한다. 입력이 ``1 $x$ $y$" 이면 $(x, y)$에 소가 추가되었다는 것이다. 입력이 ``2 $A$ $B$ $C$" 이면 $Ax+By=C$ 울타리가 유용한지 질문 하는 것을 의미한다.
	
	모든 소들의 위치는 다르며 $-10^9 \le x, y \le 10^9$를 만족한다. 또한, 울타리에 관련된 질문은  $-10^9 \le A, B \le 10^9$, $-10^{18} \le C \le 10^{18}$을 만족한다. 어떤 질문도 $A=B=0$인 경우는 없다.
	
	\OutputFile
	
	각 울타리의 위치를 묻는 질문에 대해 유용한 위치이면 ``\texttt{YES}", 아니면 ``\texttt{NO}"를 출력한다.
	
	\Examples
	
	\begin{example}
		\exmp{
3 4
0 0
0 1
1 0
2 2 2 3
1 1 1
2 2 2 3
2 0 1 1

		}{%
YES
NO
NO

		}%
	\end{example}

	\Notes
	
	직선 $2x+2y=3$를 기준으로 처음 세 마리의 소는 모두 같은 쪽에 있다. 하지만 $(1, 1)$에 있는 소가 추가 된 이후로 더 이상 이 직선에 해당하는 울타리는 유용하지 않다.
	
	직선 $y=1$에 해당하는 울타리는 $(0, 1)$과 $(1, 1)$에 있는 소가 정확히 울타리 위에 있기 때문에 유용하지 않다.
	
	경고: 이 문제의 입력은 매우 크기 때문에, C++ 사용자들은 \texttt{scanf}나 \texttt{ios\_base::sync\_with\_stdio(false)} 줄을 추가하여 입력을 빠르게 받는 것을 고려하십시오. Java 사용자들은 \texttt{java.util.Scanner}를 사용 하는 것을 피해야 합니다. 각 쿼리의 출력마다  (\texttt{std::endl} 등을 사용해서) 버퍼를 비우지 마세요. 

	
\end{problem}

