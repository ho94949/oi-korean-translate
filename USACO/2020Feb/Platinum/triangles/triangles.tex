\begin{problem}{등변삼각형}
	{traingles.in}{traingles.out}
	{2초}{256MB}{}
	
	농부 존의 목장은 $N \times N$ 격자이다. ($1 \le N \le 300$) 모든 $1 \le i, j \le N$을 만족하는 $i$, $j$에 대해 해당하는 격자 칸을 $(i, j)$로 표현한다. 격자의 각 칸에 대해서, 격자 칸에 소가 있으면 입력은 `\texttt{*}'이고, 없으면 `\texttt{.}' 이다. 
	
	농부 존은 이 목장의 아름다움은 서로 같은 거리만큼 떨어져 있는 소 세 마리를 고를 수 있는 가짓수와 비례한다고 생각한다. 유감스럽게도 농부 존은 소들이 정수 좌표에 있다는 사실을 안 지 얼마 안 됐고, 유클리드 거리를 사용한다면 소 세 마리를 고를 방법은 없다! 그러므로 농부 존은 ``맨해튼 거리"를 사용하기로 했다. 두 위치 $(x_0, y_0)$와 $(x_1, y_1)$ 사이의 맨해튼 거리는 $|x_0 - x_1| + |y_0-y_1|$이다.
	
	소들의 위치를 표현한 격자가 주어졌을 때, 서로 같은 거리만큼 떨어져 있는 소 세 마리를 고를 수 있는 가짓수를 출력하여라.
	
	
	\InputFile
	
	첫째 줄에는 $N$이 주어진다.
	
	각 $i$에 ($1 \le i \le N$) 대해, $i+1$ 번째 줄은 `\texttt{*}' 혹은 `\texttt{.}'으로 이루어진 길이 $N$의 문자열이다. $j$ 번째 문자는 $(i, j)$ 격자 칸에 소가 있는지 없는지를 나타낸다.
	
	\OutputFile
	
	답을 출력하여라. 답이 32bit 정수 범위 내에 들어옴이 보장된다.
	
	
	\Example
		
	\begin{example}
	\exmp{
3
*..
.*.
*..
	}{%
1
	}%
	\end{example}

	소는 세 마리 있고, 세 소는 서로 같은 거리인 2 만큼 떨어져 있다.
	
	\Scoring
	
	예제를 제외하고, 각 테스트 케이스의 $N$은 $N \in \{50, 75, 100, 125, 150, 175, 200, 225, 250, 275, 300, 300, 300, 300\}$을 만족한다.
	
	
	
	
	
\end{problem}

