\begin{problem}{항해 이야기}
	{standard input}{standard output}
	{5 seconds}{128 megabytes}{}
	
	Young Bytensson은 항구 선술집에서 항해 이야기를 들려주는 바닷개들의 이야기를 종종 듣는다. 처음에 그는 모든 말을 믿었지만 시간이 지남에 따라 믿을수 없는 이야기를 하는 것 같았다. 시간이 지남에 따라 그는 의심을 가지기 시작했다. 그는 이 기나긴 이야기에 진실이 조금이라도 있는지 판단하는 프로그램을 만들기로 결심했다. Bytensson은 선원들이 폭풍우 속에서 살아남았는지 확인할 방법은 없지만, 여행 일정이 논리적으로 문제가 없는지는 확인 할 수 있다. 하지만 이것은 프로그래머의 일이고 불행하게도 Bytensson은 프로그래머가 아니다. 그를 도와주자!
	
	Bytensson이 선원들에게 들었던 이야기에는 $n$개의 항구와 $m$개의 수로가 있다. 두 항구 사이에 수로가 있다면 한 항구에서 다른 항구로 항해나는 것이 가능하다. 모든 수로는 양방향으로 항해 할 수 있다.
	
	Bytensson은 $k$개의 항해 이야기를 듣게 되었다. 선원은 시작 항구에서 시작하여 몇개의 수로를 거쳐 끝 항구에 도착하는 것으로 여행을 끝냈다. (시작 항구와 끝 항구는 같을 수도 있다.) 이 항해에서는 같은 수로를 여러번 이용 했을 수도 있다.
	
	\InputFile
	
	첫째 줄에는, $n$, $m$, $k$가 공백 하나로 구분되어 들어온다. ($2 \le n \le 5,000$, $1 \le m \le 5,000$, $1 \le k \le 1,000,000$) 이 수들은 각각 항구의 갯수, 수로의 갯수, 항해 이야기의 갯수를 나타낸다.
	
	다음 $m$개의 줄 각각에는 수로를 의미하는 두 정수 $a$와 $b$가 공백 하나로 구분되어 들어온다. ($1 \le a, b \le n$, $a \neq b$). 이 수는 수로의 양 끝에 있는 항구의 번호를 의미한다.
	
	다음 $k$개의 줄은 Bytensson이 들은 항해 이야기를 의미한다. 하나의 이야기는 한 줄에 주어진 공백 하나로 구분된 세 정수 $s$, $t$, $d$로 표현 된다. ($1 \le s, t \le n$, $1 \le d \le 1, 000, 000, 000$). 이것은 이야기의 주인공이 항구 $s$에서 시작하여 $t$로 끝나고 수로 $d$개를 사용했다는 것을 의미한다.
	
	\OutputFile
	
	출력은 $k$줄로 되어 있다. (입력에서 주어진 순서대로) $i$번째 이야기가 가능한 여행 일정이라면 ``\texttt{TAK}"을, 아니면 ``\texttt{NIE}"를 출력하여라. (따옴표는 출력하지 않는다.) 
	
	 
	\Examples
		
	\begin{example}
	\exmp{
	8 7 4
	1 2
	2 3
	3 4
	5 6
	6 7
	7 8
	8 5
	2 3 1
	1 4 1
	5 5 8
	1 8 10
	}{%
	TAK
	NIE
	TAK
	NIE
	}%
	\end{example}
    
\end{problem}

