\begin{problem}{회전}
	{standard input}{standard output}
	{1 second}{128 megabytes}{}
	
	
	범수는 동생 상수에게 1번부터 $n$번까지 번호가 붙은 블럭을 사서 특정한 순서로 섞어서 배치했다. 상수는 숫자들을 1부터 $n$까지 차례로 정렬하고 싶다. 하지만 상수는 다음의 행동 밖에 하지 못한다:
	
	\begin{itemize}
		\item 행동 a: 가장 마지막 블럭을 처음 위치에 가져다 놓는다.
		\item 행동 b: 세번째 블럭을 처음 위치에 가져다 놓는다.
	\end{itemize}
	
	상수가 블럭을 정렬할 수 있도록 도와주자!
	
	
	\InputFile
	
	첫째 줄에는 정수 $n$이 주어진다. ($1 \le n \le 2,000$) 둘째 줄에는 $n$개의 정수가 공백 하나로 구분되어 주어진다. 각 수는 1 이상 $n$이하이고, 중복된 수가 없다. 이 수들은 범수가 상수에게 준 블럭들을 의미한다.
	
	\OutputFile
	
	상수가 행동들을 반복해서 숫자를 증가하는 순서로 정렬할 수 없다면,  ``\texttt{NIE DA SIE}"를 첫째 줄에 출력한다. (따옴표는 출력하지 않는다)

	그렇지 않을 경우 실행의 횟수를 나타내는 $m$을 첫째 줄에 출력해야 한다. ($m\le n^2$) 실행은 같은 행동을 $k$번 하는 것을 의미한다.
	
	$m > 0$이면, 정수 뒤에 \texttt{a}나 \texttt{b}가 붙은 꼴의 문자열 $m$개를 출력해야 한다. $k\texttt{a}$는, 행동 a를 $k$번 반복한 실행을, $k\texttt{b}$는 행동 b를 $k$번 반복한 실행을 나타낸다. ($0 < k < n$)

	그리고 둘째 줄에 등장하는 숫자 뒤에 붙은 문자들은 서로 교대로 나타나야 한다.
	
	답이 여러개인 경우, 아무거나 출력하여도 좋다.
	
	\Examples
		
	\begin{example}
	\exmp{
4
1 3 2 4
	}{%
4
3a 2b 2a 2b
	}%
	\end{example}
	.
        
\end{problem}

